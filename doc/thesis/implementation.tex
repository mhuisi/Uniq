\chapter{Implementation}\label{sec:implementation}

We have implemented a type checker for the type theory described in \Cref{sec:theory} using Lean 4 at \url{https://github.com/mhuisi/Uniq}, targeting the model IR from \Cref{sec:ir}. Note that we have not yet integrated our checker into Lean 4 itself. Unless otherwise stated below, our implementation follows the formal description in \Cref{sec:theory}.

\section{Deviations From Specification}

To represent the global and partial functions $\gamma$, $\delta_\tau$, $\delta$, $\delta_{q_e}$, $\gamma_{*_e}$, as well as $\delta_Q$, we use a map type based on red-black trees. On the other hand, the function $\gamma_*$ that yields unique fields is not represented at all because the codomain is usually infinite, and instead we use isUnique for every escapee in isBorrowed directly.

In order to perform the data flow analysis in \Cref{sec:escapeanalysis}, we use Kosaraju's algorithm \citep{sharir_strong-connectivity_1981} to compute strongly connected components in the call graph of functions $c \in \dom(\delta)$, traverse the resulting graph of strongly-connected components in reverse topological sort and then iteratively compute $\delta_Q(c)$ within each strongly connected component of mutually recursive functions until we reach a fixed point. Kosaraju's algorithm seems to be more suited to compute strongly-connected components in a functional language because it needs to maintain less state than Tarjan's algorithm \citep{tarjan_depth-first_1972}.

We implement the type checker for the type theory in \Cref{sec:checking} as follows:
\begin{itemize}
	\item $Z \in \ZeroedFields$ is implemented as a red-black tree based set, $\Gamma$ as a red-black tree based map from Var to AttrType, not a multiset. Since we know that variables are either unique or shared, all variables in $\Gamma$ can either be duplicated and forgotten freely, or there is only one of them in the context. Hence, we do not need to track the exact amount of each variable in the context, only whether it is unique or shared.
	\item The type rules are implemented by matching on the program and applying the corresponding unique rule.
	\item When checking whether a type $\tau_1$ is applicable to a type $\tau_2$ in any rule, we do not check for equality, but instead a relation $\tau_1 \rightsquigarrow \tau_2$ that determines whether $\tau_1$ can be made equal to $\tau_2$ after applying the \textsc{Downcast}, \textsc{$\blacksquare$-Cast} and \textsc{$\blacksquare$-Erase} rules.
	\item When using \textsc{Downcast}, \textsc{$\blacksquare$-Cast} and \textsc{$\blacksquare$-Erase}, it is important that we replace the old type with the new one, lest we could retain a unique attribute in our context after it has already been made shared. Hence, after checking whether $\tau_1 \rightsquigarrow \tau_2$ for $x : \tau_1 \in \Gamma$ in a rule, we set $\Gamma[x \mapsto \tau_2]$ to update the context. This can be understood as applying \textsc{Downcast}, \textsc{$\blacksquare$-Cast} and \textsc{$\blacksquare$-Erase} as needed just before the application of the rule.
	\item To eliminate \textsc{Duplicate} and \textsc{Forget}, we integrate their application carefully into the other rules by not consuming shared variables in rules that would otherwise consume them.
	\item We infer the constructor type arguments and the constructor return type to decide how to apply \textsc{Let-Ctor-*} and \textsc{Let-Ctor-!} according to our description below in \Cref{sec:ctorinference}.
	\item When applying a vector $[\tau_1]$ to $[\tau_2]$ for $[x : \tau_1] \subseteq \Gamma$, we apply \textsc{Downcast}, \textsc{$\blacksquare$-Cast} and \textsc{$\blacksquare$-Erase} while consuming unique variables and updating the context iteratively, so that e.g.\ \lstinline|c x x| fails for $\icode{x} : *\ \blacksquare$ and $\icode{c} : *\ \blacksquare \to *\ \blacksquare \to ...$ or $\icode{c} :\ !\ \blacksquare \to *\ \blacksquare \to ...$.
\end{itemize}

\section{Constructor Type Inference}\label{sec:ctorinference}
In this section, we will discuss methods of inference in order to decide how and when to apply the \textsc{Let-Ctor-*} and \textsc{Let-Ctor-!} rules in \Cref{sec:checking}. All methods discussed here are approximations, i.e.\ the \textsc{Let-Ctor-*} and \textsc{Let-Ctor-!} rules defined here using inference can lead to a type error when the corresponding rules in \Cref{sec:checking} will not. However, errors of this kind should be easy to understand and easy to resolve locally in user code.

\subsubsection{Type arguments}
As discussed briefly in the description of the \icode{ctor} instruction in \Cref{sec:ir}, we must infer the attributes in a \icode{(A [τ?]).ctorᵢ [y]} call since Lean cannot provide them and because we need them to decide how to apply the \textsc{Let-Ctor-*} and \textsc{Let-Ctor-!} rules in \Cref{sec:typetheory}. 

We will do so in two steps: First, we assign types to type variables based on the types of the arguments provided in \texttt{[y]}. Then, we use the explicit type arguments provided by the user in \icode{[τ?]} to fill the remaining variables that could not be inferred and choose their attributes as strongly as possible, since type arguments that cannot be inferred from the types of \texttt{[y]} are not subject to any uniqueness constraints induced by the vector of constructor arguments. 

Together, this ensures that no attributes for type arguments must be provided explicitly by the user, as they can either be inferred or chosen as strongly as possible.

\begin{alignat*}{3}
	f &\sqcup g : (M \rightharpoonup N) \times (M \rightharpoonup N) \rightharpoonup (\dom(f) \cup \dom(g) \rightharpoonup N) \\
	(f &\sqcup g)(x) = \begin{cases}
		f(x) & x \in \dom(f) \land x \notin \dom(g) \\
		g(x) & x \notin \dom(f) \land x \in \dom(g) \\
		f(x) & x \in \dom(f) \cap \dom(g) \land f(x) = g(x)
	\end{cases}
\end{alignat*}

Note that $(f, g) \notin \dom(\cdot \sqcup \cdot)$ if $\exists x \in \dom(f) \cap \dom(g).\ f(x) \neq g(x)$.

\newcommand{\inferVarsDash}{\mathrm{inferVars'}}

\begin{alignat*}{3}
	&\inferVarsDash &&: \mathrlap{\AttrType \times \AttrType \rightharpoonup (\Var \rightharpoonup \AttrType)} \\
	&\inferVarsDash&&(m\ x^\kappa, \tau) &&= \emptyset \\
	&\inferVarsDash&&(x^\tau, \tau) &&= \{ x^\tau \mapsto \tau \} \\
	&\inferVarsDash&&(m_1\ \blacksquare, m_2\ \blacksquare) &&= \emptyset \\
	&\inferVarsDash&&(m_1\ A\ [\tau_{\arrg_1}], m_2\ A\ [\tau_{\arrg_2}]) &&= \bigsqcup_i \inferVarsDash([\tau_{\arrg_1}]_i, [\tau_{\arrg_2}]_i) \\
	&\inferVarsDash&&(!\ [\tau_{\param_1}] \to \tau_{\ret_1}, !\ [\tau_{\param_2}] \to \tau_{\ret_2}) &&= \\
	&\rebreak{\bigsqcup_i \inferVarsDash([\tau_{\param_1}]_i, [\tau_{\param_2}]_i) \sqcup \inferVarsDash(\tau_{\ret_1}, \tau_{\ret_2})}
\end{alignat*}

\newcommand{\inferVars}{\mathrm{inferVars}}

\begin{alignat*}{3}
	\inferVars &: \mathrlap{[\AttrType] \times [\AttrType] \rightharpoonup (\Var \rightharpoonup \AttrType)} \\
	\inferVars&([], []) &&= \emptyset \\
	\inferVars&(\tau_1 :: \rest_1, \tau_2 :: \rest_2) &&= \inferVarsDash(\tau_1, \tau_2) \sqcup \inferVars(\rest_1, \rest_2) \\
\end{alignat*}

inferVars' takes an expected type with type variables and a provided type and computes an assignment of types to variables s.t.\ the first type becomes the second after substitution. In doing so, it ignores self-variables $m\ x^\kappa$, because the type without type variables has not had its variables substituted yet, while the provided type has had its variables substituted. As a result, inferVars' yielding an assignment by itself cannot guarantee that the first type is equal to the second one after substitution, and we must perform another check after substituting all the variables to obtain this guarantee. 

Note also that if there is a conflicting assignment for any variable in $\inferVarsDash(\tau_1, \tau_2)$ or $\inferVars([\tau_1], [\tau_2])$, then $\cdot \sqcup \cdot$ propagates its partiality to inferVars' and subequently inferVars, i.e.\ $(\tau_1, \tau_2) \notin \dom(\inferVarsDash)$ and $([\tau_1], [\tau_2]) \notin \dom(\inferVars)$.

\newcommand{\pickTypes}{\mathrm{pickTypes}}

\begin{alignat*}{3}
	\pickTypes &: \mathrlap{[\AttrType?] \times [\AttrType?] \rightharpoonup [\AttrType]} \\
	\pickTypes&([]) &&= [] \\
	\pickTypes&(\tau_e? :: \rest_e, \tau_i :: \rest_i) &&= \tau_i :: \pickTypes(\rest_e, \rest_i) \\
	\pickTypes&(\tau_e :: \rest_e, - :: \rest_i) &&= \strengthen(\tau_e) :: \pickTypes(\rest_e, \rest_i) \\
\end{alignat*}

Here, we use $-$ to denote that an $\AttrType?$ is not present.
With pickTypes, we implement the mechanism that inferred types are preferred if they exist, and otherwise an explicit type is used and strengthened. Note that if $\exists i.\ [\tau_1]_i = [\tau_2]_i = -$, then $([\tau_1], [\tau_2]) \notin \dom(\pickTypes)$.

\newcommand{\inferTypeArgs}{\mathrm{inferTypeArgs}}

\begin{alignat*}{3}
	&\inferTypeArgs &&: \mathrlap{\ADT \times \Ctor \times [\AttrType] \times [\AttrType?] \rightharpoonup [\AttrType]} \\
	&\inferTypeArgs&&(a, i, [\tau_\arrg], [\tau_e?]) &&= \pickTypes([\tau_e?], [\mathrm{inferred}(y^\tau)])\\
	&\rebreak{\text{where } a_i = [\tau_\field(x^\kappa_\adt, [y^\tau])] \to *x^\kappa_\adt} \\
	&\rebreak{\text{and } \mathrm{inferred} = \inferVars([\tau_\field(x^\kappa_\adt, [y^\tau])], [\tau_\arrg]])}
\end{alignat*}

Lastly, using inferTypeArgs, we infer type arguments for a constructor $i$ in an ADT $a$ with provided arguments types $[\tau_\arrg]$ and user-provided explicit argument types $[\tau_e?]$. Once again, both inferVars and pickTypes propagate their partiality to inferTypeAgs.

\subsubsection{Return types}
Furthermore, we must infer whether the result of a \icode{(A [τ?]).ctorᵢ [y]} call is used uniquely. If it is, we want the resulting type to be unique, and if it is not, it can be shared. This distinction is important because we do not need to demand unique types for the arguments \texttt{[y]} if the ADT created by the \icode{ctor} call is used only in a shared manner.

\newcommand{\uniqueUses}{\mathrm{uniqueUses}}

\begingroup
\allowdisplaybreaks
\begin{alignat*}{3}
	&\uniqueUses &&: \mathrlap{\FnBody \times \AttrType \to 2^\Var} \\
	&\uniqueUses&&(\icode{ret x}, \tau_\ret) = \begin{cases}
		\{\icode{x}\} & \attr(\tau_\ret) = * \\
		\emptyset & \attr(\tau_\ret) =\ !
	\end{cases} \\
	&\uniqueUses&&(\icode{case x of [F]}, \tau_\ret) = \bigcup_i \uniqueUses(\icode{[F]}_i, \tau_\ret) \\
	&\uniqueUses&&(\icode{case' x of [ctorᵢ [y] ⇒ F]}, \tau_\ret) = \bigcup_i \uniqueUses(\icode{[F]}_i, \tau_\ret) \\
	&\rebreak{\cup \begin{cases}
			\{x\} & \exists \icode{y} \in \icode{[y]}.\ \icode{y} \in \uniqueUses(\icode{[F]}_i, \tau_\ret) \\
			\emptyset & \lnot \exists \icode{y} \in \icode{[y]}.\ \icode{y} \in \uniqueUses(\icode{[F]}_i, \tau_\ret)
	\end{cases}}\\
	&\uniqueUses&&(\icode{let x = c [y]; F}, \tau_\ret) = \uniqueUses(\icode{F}, \tau_\ret)\\
	&\rebreak{\cup \left\{\icode{[y]}_i\ |\ \attr([\tau_\param]_i) = * \land \delta_\tau(c) = ([\tau_\param], \tau_\ret') \right\}}\\
	&\uniqueUses&&(\icode{let x = pap c [y]; F}, \tau_\ret) = \uniqueUses(\icode{F}, \tau_\ret) \\
	&\uniqueUses&&(\icode{let x = y z; F}, \tau_\ret) = \uniqueUses(\icode{F}, \tau_\ret) \\
	&\uniqueUses&&(\icode{let x = (A [τ?]).ctorᵢ [y]; F}, \tau_\ret) = \uniqueUses(\icode{F}, \tau_\ret) \\
	&\rebreak{\cup \begin{cases}
			\left\{\icode{[y]}_j\ |\ \exists x^\tau.\ \gamma(A)_{\icode{i}j} = x^\tau \lor \attr(\gamma(A)_{\icode{i}j}) = *\right\} & \icode{x} \in \uniqueUses(\icode{F}, \tau_\ret) \\
			\emptyset & \icode{x} \notin \uniqueUses(\icode{F}, \tau_\ret)
	\end{cases}} \\
	&\uniqueUses&&(\icode{let x = projᵢⱼ y; F}, \tau_\ret) = \uniqueUses(\icode{F}, \tau_\ret)\\
	&\rebreak{\cup \begin{cases}
			\{\icode{y}\} & \icode{x} \in \uniqueUses(\icode{F}, \tau_\ret) \\
			\emptyset & \icode{x} \notin \uniqueUses(\icode{F}, \tau_\ret)
	\end{cases}}
\end{alignat*}
\endgroup
In the \icode{case' x of ...} case, we denote \texttt{x} as being used uniquely if any of its fields are used uniquely in any of the branches. Arguments to a constructor \icode{ctorᵢ} are regarded as being used uniquely if the constructor itself is used uniquely and the argument type is either unique or a type variable, in which case we assume the argument as being used uniquely. Note that it would be possible to infer this more accurately by inferring which type the type variable will be assigned to from an expected type. If a projection \icode{projᵢⱼ x} is used uniquely, then \texttt{x} is used uniquely as well.

\subsubsection{Adjusted constructor rules}
Using these inference mechanisms, we can define \textsc{Let-Ctor-*} and \textsc{Let-Ctor-!} rules that make it clear which one of the two is applicable and how $[\tau_\arrg']$ are chosen:
\begin{mathpar}
	$\inferrule[Let-Ctor-*]{[\nz(Z, \icode{y})]
		\\ (\gamma(A), \icode{i}, [\tau], \icode{[$\tau_\arrg$?]}) \in \dom(\inferTypeArgs)
		\\ [\tau_\arrg'] = \inferTypeArgs(\gamma(A), \icode{i}, [\tau], \icode{[$\tau_\arrg$?]})
		\\ \gamma(A)\{A, [\tau_\arrg']\}_\icode{i} = [\tau] \to *x^\kappa_\adt
		\\ \icode{x} \in \uniqueUses(\icode{F}, \tret)
		\\ Z; \Gamma, \icode{z} : *\ A\ [\tau_\arrg'] \vdash \icode{F} : \tret
	}
	{Z; \Gamma, [\icode{y} : \tau] \vdash \icode{let x = (A [$\tau_\arrg$?]).ctorᵢ [y]; F} : \tret}$
\end{mathpar}
\begin{mathpar}
	$\inferrule[Let-Ctor-!]{[\nz(Z, \icode{y})]
		\\ (\gamma(A), \icode{i}, [!\tau], \icode{[$\tau_\arrg$?]}) \in \dom(\inferTypeArgs)
		\\ [\tau_\arrg'] = \inferTypeArgs(\gamma(A), \icode{i}, [!\tau], \icode{[$\tau_\arrg$?]})
		\\ \gamma(A)\{A, [\tau_\arrg']\}_\icode{i} = [\tau] \to *x^\kappa_\adt
		\\ \icode{x} \notin \uniqueUses(\icode{F}, \tret)
		\\ Z; \Gamma, \icode{z} :\ !\ A\ [!\tau_\arrg'] \vdash \icode{F} : \tret
	}
	{Z; \Gamma, [\icode{y} :\ !\tau] \vdash \icode{let x = (A [$\tau_\arrg$?]).ctorᵢ [y]; F} : \tret}$
\end{mathpar}
\textsc{Let-Ctor-*} is applicable if \texttt{x} is used in a unique manner. After inferring the type arguments using inferTypeArgs, we substitute them in $\gamma(A)$ and check whether the resulting types for the fields match those of our initial arguments that we used for inference. This is because while $(\gamma(A), i, [\tau], [\tau_\arrg?]) \in \dom(\inferTypeArgs)$ is always true if $\tau$ is applicable to $\tau_\field$, it may still be true if $\tau$ is not applicable to $\tau_\field$, as inferTypeArgs is ignoring the self-variable $x_\adt^\kappa$ in $\gamma(A)$. When applying the rule, we consume the arguments and gain a new unique instance of the ADT type in return. \textsc{Let-Ctor-!} is similar and applicable if \texttt{x} is not used in a unique manner, in which case the new instance of the ADT is created as shared and we only demand the arguments \texttt{[y]} to be shared as well.

\section{Examples}\label{sec:examples}
In this section, we will cover the implementations of a few Lean functions, some of which we have seen previously, and demonstrate features of our type checker using them.

\subsubsection{List.map}
Recall our implementation of \lstinline|List.map| from \Cref{sec:lean4}, augmented with uniqueness annotations where we want types to be unique:\\
\begin{code}
def List.map (f : α → β) : *List α → *List β
  | []      => []
  | x :: xs => f x :: map f xs
\end{code}
\lstinline|List.map| translates to the following IR code:\\
\begin{ifcode}
List.map f xs = case' xs of
  ctor₀ ⇒
    let nil := (List [! $\blacksquare$]).ctor₀;
    ret nil
  ctor₁ hd tl ⇒
    let hd' := f hd;
    let tl' := List.map f tl;
    let r := (List [! $\blacksquare$]).ctor₁ hd' tl';
    ret r
\end{ifcode}

We have the following global context:
\begin{align*}
	\delta &= \{\texttt{List.map} \mapsto \lambda\ \texttt{f}\ \texttt{xs}.\ \texttt{List.map}\ \texttt{f}\ \texttt{xs}\} \\
	\gamma &= \{\List \mapsto \mu\ \List.\ *\List \sep [\alpha, *\List] \to *\List\} \\
	\delta_\tau &= \{\texttt{List.map} \mapsto ([!(!\ \blacksquare \to\ !\ \blacksquare), *\List\ [!\ \blacksquare]] \to *\List\ [!\ \blacksquare])\} \\
	\gamma_{*_e} &= \emptyset \\
	\delta_{q_e} &= \emptyset
\end{align*}

Given this context, we can check $\vdash \delta_\tau$. Since \texttt{List.map} creates a new list, we can also type it using any combination of outer List attributes. The inner input attribute can be made unique, but we cannot retain the uniqueness through the application of the function of type $!(!\ \blacksquare \to\ !\ \blacksquare)$, which is why the inner output attribute must always be shared.

Our escape analysis produces the following output: 
\begin{align*}
	\delta_Q(\icode{List.map}) &= \left\{\icode{f}_{[]}, \icode{f}_{[]} @ [t_1], \icode{xs}_{[10]}, \icode{xs}_{[11]} @ [t_2] \right\} \\
	\text{where } t_1 &= [\#\text{param}\ 0, \#\text{app}, \#\text{case}\ 1, \#\text{const}\ \icode{List.map}] \\
	t_2 &= [\#\text{param}\ 1, \#\text{app}, \#\text{case}\ 1, \#\text{const}\ \icode{List.map}]
\end{align*}
Hence, even if the outer input List attribute was shared, we would not be able to borrow the argument, as our escape analysis approximates that the tail of \texttt{xs} can escape, which would be unique if we were to borrow a unique List to \texttt{xs}.

\subsubsection{List.get?}
We will now cover our implementation of \lstinline|List.get?| from \Cref{sec:lean4} to demonstrate constructor argument type inference and an issue with our implementation of borrowing, augmented with uniqueness annotations where we want types to be unique:\\
\begin{code}
def List.get? : List α → Nat → *Option α
  | [],      _     => Option.none
  | x :: _,  0     => Option.some x
  | _ :: xs, n + 1 => List.get? xs n
\end{code}
It translates to the following IR code:\\
\begin{ifcode}
List.get? xs i = case' of
  ctor₀ ⇒
    let none := (Option [! $\blacksquare$]).ctor₀;
    ret none
  ctor₁ hd tl ⇒
    let zero := Nat.zero;
    let ieqzero := Nat.eq i zero;
    case ieqzero of
      true ⇒
        let predi := Nat.pred i;
        let r := List.get? tl predi;
        ret r
      false ⇒
        let some := (Option [-]).ctor₁ hd;
        ret some
\end{ifcode}

\newcommand{\Option}{\mathrm{Option}}
\newcommand{\Nat}{\mathrm{Nat}}
\newcommand{\Bool}{\mathrm{Bool}}
The program is simply $\delta = \{\texttt{List.get?} \mapsto \lambda\ \texttt{xs}\ \texttt{i}.\ \texttt{List.get?}\ \texttt{xs}\ \texttt{i}\}$. Our types are as follows:
\begin{align*}
	\gamma &= \left\{\begin{aligned}
		\List &\mapsto \mu\ \List.\ &*\List &\sep [\alpha, *\List] \to *\List \\
		\Option &\mapsto \mu\ \Option.\ &*\Option &\sep \alpha \to *\Option
		\end{aligned}\right\}\\
	\delta_\tau &= \left\{\begin{aligned}
		\icode{List.get?} &\mapsto [!\List\ [!\ \blacksquare],\ !\Nat\ []] &&\to *\Option\ [!\ \blacksquare] \\
		\icode{Nat.zero} &\mapsto [] &&\to\ !\Nat\ [] \\
		\icode{Nat.eq} &\mapsto [!\Nat\ [],\ !\Nat\ []] &&\to\ !\Bool\ [] \\
		\icode{Nat.pred} &\mapsto [!\Nat\ []] &&\to\ !\Nat\ []
	\end{aligned}\right\}
\end{align*}
For simplicity, we assume that Nat and Bool are external types. Furthermore, we provide no escapee information:
\begin{align*}
	\gamma_{*_e} &= \emptyset\\
	\delta_{q_e} &= \emptyset
\end{align*}

With these declarations, we can check $\vdash \delta_\tau$. Note that the type for the \texttt{Option.some} call can be inferred. Our escape analysis computes the following information:
\begin{alignat*}{1}
	\delta_Q(\icode{List.get?}) &= \left\{\icode{xs}_{[10]}, \icode{xs}_{[11]} @ [t] \right\} \text{ for some } t
\end{alignat*}
This means that a unique argument that is passed to \icode{xs} cannot be borrowed, as one of its unique fields $\icode{xs}_{[11]}$ is approximated to escape by our escape analysis.

\subsubsection{Array.transpose}
Finally, we will cover a more complex example that shows off nested uniqueness attributes and external types. We intend to check the following function that transposes a two-dimensional square matrix:\\
\begin{code}
def Array.T! (xs : *Array (*Array α)) (i j : Nat) 
  : *Array (*Array α) :=
  let n := Array.size xs
  if i >= n then
    xs
  else if j >= n then
    Array.T! xs (i + 1) (i + 2)
  else
    let (xs, xs_i) := Array.swap xs i #[]
    let (xs, xs_j) := Array.swap xs j #[]
    let x := Array.get! xs_i j
    let x' := Array.get! xs_j i
    let xs_i := Array.set! xs_i j x'
    let xs_j := Array.set! xs_j i x
    let xs := Array.set'! xs i xs_i
    let xs := Array.set'! xs j xs_j
    Array.T! xs i (j + 1)
\end{code}
We assume that \texttt{Array.T!} is always called with a square matrix and $\texttt{i} = 0$ and $\texttt{j} = 1$ at the start. For brevity, we omit the equivalent IR code. The program is simply $\delta = \{\texttt{Array.T!} \mapsto \lambda\ \texttt{xs}\ \texttt{i}\ \texttt{j}.\ \texttt{Array.T!}\ \texttt{xs}\ \texttt{i}\ \texttt{j}\}$.

\newcommand{\Tuple}{\mathrm{Tuple}}
We have the following algebraic data types:
\begin{align*}
	\gamma &= \left\{\begin{aligned}
		\Bool &\mapsto \mu\ \Bool.\ *\Bool \sep *\Bool \\
		\Tuple &\mapsto \mu\ \Tuple.\ [\alpha, \beta] \to *\Tuple
	\end{aligned}\right\}
\end{align*}
The types we assign to constants in the program in the \icode{Array} namespace are as follows:
\begin{align*}
	\delta_\tau(\icode{Array.*}) &= \left\{\begin{aligned}
		\icode{T!} &\mapsto [*\Array\ [*\Array\ [!\ \blacksquare]],\ !\Nat\ [],\ !\Nat\ []]\\
			&\to *\Array\ [*\Array\ [!\ \blacksquare]] \\
		\icode{size} &\mapsto [!\Array\ [!\ \blacksquare]] \to\ !\Nat\ [] \\
		\icode{empty} &\mapsto [] \to *\Array\ [*\ \blacksquare] \\
		\icode{swap} &\mapsto [*\Array\ [*\Array\ [!\ \blacksquare]],\ !\Nat\ [], *\Array\ [!\ \blacksquare]] \\
			&\to *\Tuple\ [*\Array\ [*\Array\ [!\ \blacksquare]], *\Array\ [!\ \blacksquare]] \\
		\icode{get!} &\mapsto [!\Array\ [!\ \blacksquare],\ !\Nat\ []] \to\ !\ \blacksquare \\
		\icode{set!} &\mapsto [*\Array\ [!\ \blacksquare],\ !\Nat\ [],\ !\ \blacksquare] \to *\Array\ [!\ \blacksquare] \\
		\icode{set'!} &\mapsto [*\Array\ [*\ \blacksquare],\ !\Nat\ [], *\ \blacksquare] \to *\Array\ [*\ \blacksquare]
	\end{aligned}\right\}
\end{align*}
For operations on natural numbers, we have:
\begin{align*}
	\delta_\tau(\icode{Nat.*}) &= \left\{\begin{aligned}
		\icode{one} &\mapsto [] &&\to\ !\Nat\ [] \\
		\icode{two} &\mapsto [] &&\to\ !\Nat\ [] \\
		\icode{geq} &\mapsto [!\Nat\ [],\ !\Nat\ []] &&\to\ !\Bool\ [] \\
		\icode{add} &\mapsto [!\Nat\ [],\ !\Nat\ []] &&\to\ !\Nat\ []
	\end{aligned}\right\}
\end{align*}

\texttt{Array} and \texttt{Nat} are external types. For \texttt{Array}, \texttt{size} and \texttt{get!} we provide the following escape information:
\begin{align*}
	\gamma_{*_e} &= \{\Array \mapsto [00](\alpha)\} \\
	\delta_{q_e} &= \left\{\texttt{Array.size} \mapsto \emptyset, \texttt{Array.get!} \mapsto \{\texttt{xs}_{[00]}\}\right\}
\end{align*}
Here, $\alpha$ denotes the first type parameter of \texttt{Array} and \texttt{xs} denotes the first parameter of \texttt{Array.get!}.

Using these definitions, we can check $\vdash \delta_\tau$. The following aspects are noteworthy about this example: 
\begin{itemize}
	\item \texttt{xs} is successfully borrowed in the applications of \texttt{Array.size} and \texttt{Array.get!}. Without borrowing, this definition would not type check, as \texttt{xs} would be consumed on application of these functions.
	\item The \texttt{Array.swap} function ensures that the inner uniqueness is retained by swapping in a unique value that takes the place of the value in the array. When using the \texttt{Array.swap} function, we must be careful to maintain $\icode{i} \neq \icode{j}$, as we would otherwise access the same index twice.
	\item Since we lack polymorphism, we have two separate \texttt{Array.set!} functions, depending on whether the inner attribute is shared or unique.
\end{itemize}