\documentclass[parskip=no,12pt,a4paper,twoside,headings=openright, fleqn]{scrreprt}
% switch to scrbook if you want roman page numbers for the front matter
% however scrbook has no 'abstract' environment!
% if your thesis is in english, use "parskip=no" instead

% binding correction (BCOR) von 1cm für Leimbindung
\KOMAoptions{BCOR=1cm}
\KOMAoptions{draft=yes}

\usepackage[utf8]{inputenc} % encoding of sources
\usepackage{natbib}
\usepackage[T1]{fontenc}
\usepackage{style/studarbeit}
\usepackage{cleveref}
\usepackage{mathpartir}
\usepackage{mathtools}
\usepackage{amssymb}
\usepackage{amsmath}
\usepackage{MnSymbol}
\usepackage{stmaryrd}
\usepackage{bbold}
\usepackage[framemethod=tikz]{mdframed}
\usepackage{placeins}
\usepackage{float}

\usepackage{lineno}
\linenumbers

\makeatletter
\setlength{\@fptop}{0pt}
\setlength{\@fpbot}{0pt plus 1fil}
\makeatother

\makeatletter
\providecommand{\leftsquigarrow}{%
	\mathrel{\mathpalette\reflect@squig\relax}%
}
\newcommand{\reflect@squig}[2]{%
	\reflectbox{$\m@th#1\rightsquigarrow$}%
}
\makeatother
\makeatletter
\newcommand*{\textoverline}[1]{$\overline{\hbox{#1}}\m@th$}
\makeatother

\mdfsetup{
	innertopmargin=10pt,
	innerbottommargin=10pt,
	innerleftmargin=10pt,
	innerrightmargin=5pt
}

\def\lstlanguagefiles{lstlean.tex}
\lstset{language=lean}
\lstdefinelanguage{ir-if}
{
	morekeywords={
		case,
		case',
		of,
		ret,
		proj,
		let,
		ctor,
		pap,
		inc,
		dec,
		reset,
		reuse,
		in,
		nat,
		str,
		var,
		type,
		lit,
		const,
		default,
		def,
		jpdef,
		jmp
	},
	morecomment=[l]{--}, % l is for line comment
}
\lstdefinelanguage{aux}
{
	keywordstyle={\color{black}},
	morekeywords={
		left,
		right,
		case,
		of,
		let,
		in,
		copy,
		as
	},
	morecomment=[l]{--}, % l is for line comment
}

\lstnewenvironment{ifcode}{\minipage{\linewidth}\lstset{xleftmargin=1.48em, language=ir-if}}{\endminipage\hfill\allowbreak}
\lstnewenvironment{code}{\minipage{\linewidth}\lstset{xleftmargin=1.48em}}{\endminipage\hfill\allowbreak}

\definecolor{keywordcolor}{rgb}{0.7, 0.1, 0.1}   % red
\definecolor{commentcolor}{rgb}{0.4, 0.4, 0.4}   % grey
\definecolor{symbolcolor}{rgb}{0.0, 0.1, 0.6}    % blue
\definecolor{sortcolor}{rgb}{0.1, 0.5, 0.1}      % green

\title{Static Uniqueness Analysis for the Lean 4 Theorem Prover}
\author{Marc Huisinga}
\thesistype{Masterarbeit}
\zweitgutachter{Prof.~Dr.~rer.~nat.~Bernhard~Beckert}
\betreuer{M.~Sc.~Sebastian~Ullrich}
%\zweitbetreuer{M.~Sc.~Johannes Fried-Graf}
\coverimage{4665389330_d09f3d6b75_z.jpg}
\abgabedatum{{\year=2023 \month=4 \day=5 \today}}

\begin{document}

\begin{otherlanguage}{ngerman} % Titelseite ist immer auf Deutsch
\mytitlepage
\end{otherlanguage}

\begin{abstract}
\begin{center}\Huge\textbf{\textsf{Abstract}}
\end{center}
\vfill

\begin{otherlanguage}{ngerman}
Der Lean 4 Theorembeweiser, welcher auch als vollwertige Programmiersprache verwendbar ist, implementiert eine auf Referenzzählung basierende Optimierung, die das destruktive Mutieren von pur-funktionalen Werten ermöglicht: Wenn der Referenzzähler eines Werts gleich 1 ist, dann kann dieser sicher an Ort und Stelle mutiert werden. Das Greifen dieser Optimierung ist insbesondere für Arrays unabdingbar, da dort die einzige Alternative zu einer Mutation das vollständige Kopieren des Arrays ist. Um das Greifen dieser Optimierung zu garantieren, mustern wir den Entwurfsraum der substrukturellen Typtheorien und implementieren eine eigene Uniqueness-Typtheorie. Wir implementieren unsere Typtheorie für ein Modell einer Zwischenrepräsentation für Lean 4. Unsere Typtheorie unterstützt Uniqueness-Typen, algebraische Datentypen, gelöschte Typen, externe Deklarationen, nicht-oberflächliches Subtyping von Uniqueness-Attributen und einen Borrowing-Mechanismus, welcher mittels einer Escape-Analyse implementiert ist.
\end{otherlanguage}
\vfill
The Lean 4 theorem prover and programming language implements an optimization based on reference counting that allows for destructively updating purely functional values: If the reference count of a value is equal to 1, it can be safely updated in-place. Especially for arrays, where the only alternative to an in-place update is a full copy of the array, it is essential that this optimization always applies. To ensure this, we survey the design space of substructural type theory and implement a uniqueness type theory of our own. Our type theory targets a model of an intermediate representation for Lean 4. It supports uniqueness types, algebraic data types, erased types, external declarations, non-shallow subtyping for uniqueness attributes and a borrowing mechanism that is implemented using an escape analysis.
\vfill
\end{abstract}

\tableofcontents

\chapter{Introduction}\label{sec:intro}

The Lean 4 theorem prover and programming language \citep{de_moura_lean_2021} features an intermediate representation (IR) language that allows for efficiently mutating values in a purely functional programming language \citep{ullrich_counting_2020}. Garbage collection is implemented via reference counting (RC) and values can be safely and efficiently updated in-place at runtime when the value is unique, i.e. the reference count is equal to 1, a technique known as ``destructive update''. This feature yields fewer memory allocations, faster updates and enables the use of arrays in purely functional programming languages \citep{ullrich_counting_2020}.

However, for programmers it can be difficult to judge whether a value is always unique during program execution, leading to possibly huge disparities in runtime when uniqueness is violated by accident. This issue is especially significant when using arrays or array-derived types, where the fall-back when not being able to update the array in-place is to copy the array in its entirety, bumping the runtime for that specific function call from $\Theta(1)$ to $\Theta(n)$.

The obvious remedy for this issue is to use a type system which ensures that values are unique and issues an error when uniqueness is violated. In addition to aiding debuggability, the information of successfully type-checking a program can also be used for optimization purposes: If, at compile time, we know that a value is always unique, we can eliminate the instructions that check the reference count at runtime and always directly perform the in-place update.

To guarantee uniqueness in this manner, we design and implement a uniqueness type system \citep{sergey_linearity_2022}, a kind of type system pioneered by the Clean programming language \citep{smetsers_guaranteeing_1994}. We implement our type checker in the Lean 4 programming language, targeting a model of Lean 4's IR language.

In \cref{sec:background}, we revisit Lean 4, its IR language and the details related to using reference counting to perform in-place updates, as well as explain and evaluate different techniques for statically ensuring uniqueness. In \cref{sec:designspace}, we explore the different facets and challenges of designing a uniqueness type system. In \cref{sec:theory}, we provide a formal description of our type theory, and in \cref{sec:implementation}, we detail challenges and particularities in implementing the type checker for our type system. In \cref{sec:evaluation}, we re-visit our synopsis of the design space from \cref{sec:designspace} and locate our type system within it. Finally, in \cref{sec:furtherwork}, we describe possible future work, and in \cref{sec:relatedwork}, we compare our approach to other similar type systems with the same or similar goals.
\chapter{Background}\label{sec:background}

\section{Lean 4 Theorem Prover}\label{sec:lean4}
Lean 4 is a proof assistant and programming language developed primarily by Leonardo de Moura at Microsoft Research and Sebastian Ullrich at KIT with numerous open source contributions by other authors.

Up to and including version 3, Lean served only as a proof assistant, i.e. an interactive tool where users can input proofs that are then checked by the proof assistant. Many proof assistants also implement proof-generating automation, so-called ``tactic languages'', to make the task of writing a perfectly formal proof by hand less tedious. In Lean 3, the previous version of Lean, the same term language was used for proofs, theorem statements, definitions, programs, type declarations, specifications, and implementing automation. We will go into some detail on how Lean uses a single unified language for all of these things in \cref{sec:dtt}. For automation, the term language was also evaluated by a separate interpreter for more efficient execution.

Unfortunately, evaluating the term language using a separate interpreter would still yield inadequate performance, both for implementing more demanding automation and for implementing real world programs, which meant that such demanding programs were written in C++ instead and then made available to Lean using a foreign-function interface (FFI) \citep{ullrich_counting_2020}.

To improve on this, Lean 4 now implements its own self-hosted compiler toolchain for both a C backend and a work-in-progress LLVM backend, including its own IR, optimization pipeline and custom garbage collection algorithm. Being almost entirely self-hosted, Lean 4 is now also well capable of being used as a general purpose programming language.

Let us now consider some basic examples of Lean 4 code to get a feeling for the language. It should be noted that all of the following can be expressed more succinctly, but that we have chosen not to do so in order to make these snippets more easy to grasp for readers not familiar with Lean.
\begin{code}
def List.map (f : α → β) : List α → List β
  | []      => []
  | x :: xs => f x :: map f xs
\end{code}
\lstinline|List.map| is implemented by recursion on the second argument. The empty list again yields the empty list. If the list is a cons cell, we map the head of the cons cell using \lstinline|f|, recurse on the tail and then build a new cons cell from the results of both.
\begin{code}
def List.get? : List α → Nat → Option α
  | [],      _     => Option.none
  | x :: _,  0     => Option.some x
  | _ :: xs, n + 1 => List.get? xs n
\end{code}
\lstinline|List.get?| is implemented by recursion both on the provided list and the index provided in the second argument. It yields an \lstinline|Option α|, i.e. either \lstinline|Option.some α| if the index is in the list, or \lstinline|Option.none| otherwise.
\begin{code}
def Array.groupBy (p : α → α → Ordering) (xs : Array α)
  : RBMap α (Array α) p := Id.run do
  let mut result : RBMap α (Array α) p := RBMap.empty
  for x in xs do
    let group := Option.getD (RBMap.find? result x) #[]
    result := RBMap.insert result x (Array.push group x)
  return result
\end{code}
\lstinline|Array.groupBy| is implemented using an imperative domain-specific language (DSL) based on do-notation \citep{ullrich_beyond_2022}. It takes a relation \lstinline|p| that yields an \lstinline|Ordering|, i.e. whether the first argument is greater than, smaller than or equal to the second argument, as well as the array to group the elements of. It returns a red-black map ordered by \lstinline|p|, where the keys are arbitrary representatives of the group and the values denote groups \lstinline|Array α| of \lstinline|p|-equivalent values.

In order to use do-notation, we need to run the code in a monad. Since we do not intend to accumulate any effects and only use do-notation for its imperative domain-specific language (DSL), we use the \lstinline|Id| monad, entering it using \lstinline|Id.run|. First, we initialize a mutable but empty red-black map, denoting our result. Then, we iterate over every element \lstinline|x| in the provided array and look for a group in the current result that is \lstinline|p|-equivalent to \lstinline|x|. If we find such a group, we store it in \lstinline|group|. Otherwise, we allocate a new group for elements \lstinline|p|-equivalent to \lstinline|x| using the call to \lstinline|Option.getD|, which returns the first element if it was equal to \lstinline|some x| and otherwise returns the second element if it was equal to \lstinline|none|. Then, we add \lstinline|x| itself to the group and re-insert it into our mutable \lstinline|result| map. At the end, we return the accumulated \lstinline|result| map.

This piece of imperative code is implemented as a DSL, i.e. the imperative code is translated to a functional equivalent.
One may reasonably wonder why one would ever use a purely functional language to then write imperative code, translate it back to purely functional code, only to then compile the result to imperative machine code. But since Lean 4 is also an interactive theorem prover, the answer is simple: Imperative programming can be convenient, but it is easier to use a purely functional programming language for all the domains of proof, specification and writing programs at once, since all the computational effects are already neatly packed away. In fact, if we were to write a proof about \lstinline|Array.groupBy|, the first thing we would likely do is fold away the syntactical imperative layer in order to uncover the purely functional term representing the program, without ever having to think about loop invariants or state.

One may also wonder whether code written in this imperative manner is about as efficient as real imperative code. Because of the mechanism described in \cref{sec:beans}, this is indeed the case if the code in question is written in such a way that every value is unique, which is usually the case for the kind of code that one would also write in an imperative language. Importantly, functional code will benefit from this as well, which means that we can write compositional, functional code that also updates values in-place instead of making new allocations in every single combinator. It must however be noted that our implementation of \lstinline|Array.groupBy| is actually an example of an imperative implementation that is unexpectedly inefficient. We will resolve this inefficiency in \cref{sec:beans} when it is instrumental to do so.

For most of this thesis, we will not treat Lean 4 as a theorem prover, but instead as a purely functional programming language that implements dependent type theory. For more details on Lean 4 as a programming language, we refer to the book Functional Programming in Lean by \cite{christiansen_functional_2023}.

\section{Dependent Type Theory}\label{sec:dtt}
As described in \cref{sec:lean4}, Lean uses a single language for programming and proving. It accomplishes this by implementing dependent type theory (DTT), a type theory powerful enough to declare mathematical objects, implement programs and write specifications and proofs for both. What follows is only a very brief introduction to some of the important details of Lean's type theory, and we recommend the book Theorem Proving in Lean 4 by \cite{avigad_theorem_2022} for a proper introduction.

The central idea of DTT is that types are allowed to depend on terms. In most type theories, terms and types are entirely different constructs, and while terms have types, terms cannot be used in types. Removing this restriction blurs the line between programs and their static specification.

There are several mechanisms required to make this work: 
\begin{enumerate}
	\item In a quantified type $\forall x.\ \tau(x)$, the variable $x$ is allowed to range over terms of a type (e.g. $x \triangleq n : \mathbb{N}$), not just types themselves as is the case in languages that support polymorphic functions $\forall \alpha.\ \tau(\alpha)$.
	\item Terms in types can be reduced with the usual reduction rules of lambda calculus, e.g. $(\forall x.\ \tau((\lambda y.\ y)\ x)) \equiv (\forall x.\ \tau(x))$.
	\item Support for inductive type families, which are essentially algebraic data types where each constructor creates a term in a type that can be parametrized by other terms. For example, we might declare a type $\lambda \alpha : \mathrm{Type}.\ \lambda n : \mathbb{N}.\ \mathrm{Vec}\ \alpha\ n$ for lists over a type $\alpha$ of size $n$ with constructors $\mathrm{nil} : \mathrm{Vec}\ \alpha\ 0$ and $\mathrm{cons} : \forall n : \mathbb{N}.\ \alpha \to \mathrm{Vec}\ \alpha\ n \to \mathrm{Vec}\ \alpha\ (n+1)$. Each inductive type family also yields a recursion principle that allows for pattern matching and using recursion on the value of a type to compute an accumulate value.
\end{enumerate}

In addition to $\forall x : \tau_1.\ \tau_2(x)$, dependent type theories also always support dependent product types $(x : \tau_1) \times \tau_2(x)$ and dependent sum types $(x : \tau_1) + \tau_2(x)$.

For convenience, Lean's type theory also supports a separate type universe of propositions $\mathbb{P}$, the terms of which are types $p : \mathbb{P}$ with proof terms $h : p$ witnessing the truth of the proposition $p$. For example, if $\mathrm{refl} : \forall x.\ x = x$, we have $(\mathrm{refl}\ n) : (n = n) : \mathbb{P}$ and $(\mathrm{refl}\ n) : (n + 1 - 1 = n) : \mathbb{P}$ for $n : \mathbb{N}$, as $n + 1 - 1 \equiv n$. Meanwhile, there is no term $h : (n = n + 1) : \mathbb{P}$. For less trivial propositions, we use the recursion principles of inductive type families, which become induction principles if the accumulated value is a proposition $p : \mathbb{P}$. 

What distinguishes propositions $p : \mathbb{P}$ from other types is that $\mathbb{P}$ is impredicative and proof-irrelevant, i.e. whenever we quantify over a proposition $p : \mathbb{P}$, the resulting type is again a proposition, and for proofs $h_1, h_2 : p$, we have $h_1 = h_2$. In other words, propositions are contained to $\mathbb{P}$ and all proofs of a proposition are considered equal, i.e. only their existence is relevant, not the concrete content of the proof. Hence, proofs are inherently non-computational; since the content of a proof is irrelevant, it can be erased. These two features allow Lean to introduce additional non-computational classical axioms into its universe of propositions $\mathbb{P}$, most prominently the axiom of choice.

Putting all of these mechanisms together yields a type theory powerful enough to declare types like $\mathrm{Vec}\ \alpha\ n$ and all the usual objects that are used in mathematics, as well as logical operators like $\cdot \land \cdot$, $\exists x.\ p(x)$, $\cdot = \cdot$ and even well-founded recursion. Additionally, the term language is strong enough to write arbitrary programs, as well as classical proofs within $\mathbb{P}$.

For a detailed formal description of Lean's type theory, we refer to \citep{carneiro_type_2019}.

\section{Intermediate Representations for Lean 4}\label{sec:irs}

\section{Counting Immutable Beans}\label{sec:beans}

\section{Linear Type Theory}\label{sec:ltt}
Type systems typically guarantee certain functional or extensional properties for a given program using a typing relation $\Gamma \vdash e : \tau$, where $\Gamma$ is a set of type judgements $x : \tau'$. The following \textsc{Exchange}, \textsc{Weaken} and \textsc{Contract} rules are usually assumed implicitly:
\begin{mathpar}
	$\inferrule[Exchange]{\Gamma_1, y : \tau_2, \Gamma_2, x : \tau_1, \Gamma_3 \vdash e : \tau}{\Gamma_1, x : \tau_1, \Gamma_2, y : \tau_2, \Gamma_3 \vdash e : \tau}$ \hspace{1.5em}
	$\inferrule[Weaken]{\Gamma \vdash e : \tau}{\Gamma, x : \tau' \vdash e : \tau}$ \hspace{1.5em}
	$\inferrule[Contract]{\Gamma, x : \tau', x : \tau' \vdash e : \tau}{\Gamma, x : \tau' \vdash e : \tau}$
\end{mathpar} 
In other words, judgements in the context can be reordered, discarded and duplicated freely.

In the formal descriptions of dependent type theories, like the one described in \cref{sec:dtt}, \textsc{Exchange} is inhibited, because $\tau_2$ may depend on $x$, which induces an order of declaration on variables in the context. But if we want to guarantee only extensional properties for $e$, \textsc{Weaken} and \textsc{Contract} can always be freely assumed, as there is nothing to gain from retaining the exact count of every judgement in the context. 

However, if we wish to guarantee non-functional or intensional properties for a given program, then discarding the \textsc{Weaken} and \textsc{Contract} rules can be useful. And indeed doing just that constitutes the core idea of substructural type theories: By retaining the exact number of each judgement in the context, we can use typing rules to count objects in our program to ensure various kinds of intensional properties, like for example information flows \citep{choudhury_dependent_2022}.

However, in most substructural type theories, the extra detail in the context is used to count variable uses. \cite{girard_linear_1987} was the first to notice that not assuming \textsc{Weaken} and \textsc{Contract} allows one to define so-called linear logics with dualities that allow reasoning about resource usage, and \cite{wadler_linear_1990} transports this idea to form a linear type theory with a set of entirely separate linear and non-linear types prefixed by $!$, even on the term level. In both, discarding \textsc{Weaken} and \textsc{Contract} is used to enforce the invariant that linear variables can only be used exactly once. \cite{wadler_is_1991} then goes on to define \textsc{Dereliction} and \textsc{Promotion} rules that allow coercing a non-linear type to a linear type, as well as a ``steadfast'' type system where there is no coercion between linear and non-linear types, but both use the same term language. 

With \textsc{Promotion} and \textsc{Dereliction}, linear types make a guarantee for the future: We do not know whether this variable has always been linear, but now that it is, we guarantee that it will be used exactly once. Meanwhile, in Wadler's ``steadfast'' version of the type system, as there is no coercion from non-linear to linear types, a linear variable has and will always be used exactly once. In a type system with \textsc{Promotion} and \textsc{Dereliction}, linearity is hence unsuited to guarantee the uniqueness of a variable, as it may have been duplicated in the past, an issue that is acknowledged by \cite{wadler_is_1991}. For linear type systems where linearity is guaranteed from construction onwards, \cite{chirimar_reference_1996} prove that the linearity of a variable implies the uniqueness of the associated reference.

Future linear type systems \citep{goos_observers_1992}\citep{atkey_syntax_2018}\citep{bernardy_linear_2018}\citep{brady_idris_2021}\citep{choudhury_graded_2021} \citep{li_linear_2022}\citep{spiwack_linearly_2022} always adopt one of these two approaches and either allow for a non-linear to linear coercion while attempting to guarantee linearity from the construction of a value onwards, or no coercion at all, thus cementing the idea of ``linearity'' referring to either ``always linear'' or ``linear from this point onwards''.

One common refinement of linear type theory is to allow the use of \textsc{Weaken}, but not \textsc{Contract}, specifically when the type theory only wants to guarantee that a reference is unique, but not that it is used. These type theories are also knows as ``affine'' \citep{tov_practical_2011}, though the term is often conflated with ``linear''.

Amongst others, the following types are commonly present in linear type theories:
\begin{itemize}
	\item $\tau_1 \multimap \tau_2$ for the type of linear functions that consume an argument of type $\tau_1$ on application
	\item $(!\tau_1) \multimap \tau_2$ for the type of non-linear functions that consume a non-linear argument of type $\tau_1$ on application and yield a $\tau_2$ that can be made non-linear through dereliction
	\item $\tau_1 \otimes \tau_2$ for the type of multiplicative linear products, where both arguments of type $\tau_1$ and $\tau_2$ are consumed on construction, and then re-obtained through pattern matching on the resulting value
	\item $\tau_1 \oplus \tau_2$ for the type of linear sums, where one of the arguments of type $\tau_1$ and $\tau_2$ is consumed depending on which constructor is used, and then re-obtained through pattern matching on the resulting value
\end{itemize}

The various applications of linear type theory include the following:
\begin{itemize}
	\item Threading a functional program and enforcing an execution order to replace the use of monads for I/O in functional languages with a linear equivalent \citep{vries_making_2009}\citep{bernardy_linear_2018}\citep{brady_idris_2021}
	\item Ensuring resource- and memory-safety so that resources and memory cannot be freed multiple times \citep{weiss_oxide_2021}
	\item Performing efficient in-place updates and enabling the use of arrays in functional languages \citep{vries_making_2009}\citep{bernardy_linear_2018}
	\item Specifying usage protocols for types \citep{brady_idris_2021}
	\item Guiding program synthesis \citep{brady_idris_2021}
	\item Inverting the computation of functions \citep{matsuda_sparcl_2020}
\end{itemize}

\section{Quantitative Type Theory}
Quantitative type theory (QTT) applies the idea of linear type theory to dependent type theory. 

In linear type theories, the \textsc{App} rule is typically stated in a manner similar to the following in order to distribute the needed amount of judgements to both expressions:
\begin{mathpar}
	$\inferrule[App]{\Gamma_1 \vdash e_1 : \tau_1 \multimap \tau_2 \\ \Gamma_2 \vdash e_2 : \tau_1}{\Gamma_1, \Gamma_2 \vdash e_1\ e_2 : \tau_2}$
\end{mathpar}
When omitting the \textsc{Exchange} rule in dependent type theory, it is not clear that splitting the context into $\Gamma_1$ and $\Gamma_2$ is always possible, since types in $\Gamma_2$ may depend on variables in $\Gamma_1$. This seemingly technical issue induces a semantic problem as well: Which occurrences of a variable constitute a use? For example, what about occurrences in types?

After this issue was left unsolved for a long time, \cite{lindley_i_2016} resolved it by introducing a third kind of type to linear type theory, resulting in the three kinds of type ``linear'' (denoted as $1$), ``non-linear'' (denoted as $\omega$) and ``erased'' (denoted as $0$). ``erased'' specifies that a type or a term within DTT is not computationally relevant and will be erased by the compiler. Erased terms or types can only be used in other erased terms or types, and types are always erased. Finally, when a linear variable is used, it becomes erased, which justifies the use of $1$ to denote linear types and $0$ to denote erased types. \textsc{Exchange} is only omitted for variables of type $0$. In \textsc{App}, since we do not need to count the uses of variables of type $0$, we can freely distribute variables of type $1$ and $\omega$ to either $\Gamma_1$ or $\Gamma_2$, but duplicate all variables of type $0$ in their given order to $\Gamma_1$ and $\Gamma_2$. In other words, both the technical and the semantic issue are resolved at once. Quantitative type theory also retains the spirit of linear typing in that non-linear types can be coerced to linear types, which can subsequently be erased by using them in a non-erased term.

The implementations of QTT by \cite{lindley_i_2016} and \cite{atkey_syntax_2018} use an additional trick to make the description of the type theory more compact. As seen in \cref{sec:ltt}, while linear type theories usually denote linearity on the type $\tau$, quantitative type theory instead denotes it on the binder $:$ in $x : \tau$, which leads to $0$, $1$ and $\omega$ not simply being kinds of types, but ``quantities'' on the binders $x :_q \tau$ for $q \in \{0, 1, \omega\}$ in the context. Instead, there are only linear types, and the quantity on the binder determines the substructural restrictions on the variable. The typing judgement $\Gamma \vdash e :_\sigma \tau$ is quantified as well to track whether $e$ is being checked in a computationally relevant or irrelevant context, and $\sigma$ is restricted to $\{0, 1\}$ to ensure admissibility of substitution \citep{atkey_syntax_2018}. The types described in \cref{sec:ltt} now become the following dependent types:
\begin{itemize}
	\item $((x :_1 \tau_1) \to \tau_2) \triangleeq (\tau_1 \multimap \tau_2)$
	\item $((x :_1 \tau_1) \otimes (y :_1 \tau_2) \otimes \mathbb{1}) \triangleeq (\tau_1 \otimes \tau_2)$
	\item $((x :_1 \tau_1) \oplus (y :_1 \tau_2) \oplus \mathbb{0}) \triangleeq (\tau_1 \oplus \tau_2)$
\end{itemize}
It is worth pointing out that both \cite{lindley_i_2016} and \cite{atkey_syntax_2018} specify a generic framework for adding additional quantities to the type theory, allowing, for example, the additional introduction of a $\leq\hspace{-0.3em}1$ quantity representing affinity, or more accurate accounting of uses $n > 1$.

Finally, note that by denoting the linear quantity on the binder in $(x :_1 \tau_1) \to \tau_2$, we cannot specify a quantity for $\tau_2$ any more, as we could in linear type theory using $!$. Instead, in checking an application $f\ (g\ e) :_1 \tau_3$ for $f :_1 (x :_q \tau_2) \to \tau_3$, $g :_1 (x :_1 \tau_1) \to \tau_2$ and $e :_1 \tau_1$, we demand $q$ instances of the resources $\Gamma$ required to check $g\ e :_1 \tau_2$. In other words, if we need $q$ instances of a return value that requires $\Gamma$ resources to produce, we instead demand $q \cdot \Gamma$ resources in our context, pretending that we applied the function $q$ times to obtain $q$ instances of the return value. This way, quantities need not be specified on return values, and resources for the arguments are consumed based on the required amount of the return value.

Quantitative type theory combines the benefits of linear type theory and dependent type theory, leading to far greater capabilities when specifying protocols for types \citep{brady_idris_2021}. The introduction of an erasure quantity to combine the two type theories also allows for finer-grained specification of terms that are not computationally relevant but would otherwise be expensive to compute \citep{brady_idris_2021}.

\section{Uniqueness Type Theory}\label{sec:uniqueness}
\cite{smetsers_guaranteeing_1994} and \cite{barendsen_uniqueness_1996} introduce a type system for the Clean programming language with the goal of guaranteeing referential uniqueness to enable many of the applications described in \cref{sec:ltt}. The core idea is to use the linear and non-linear types of linear type systems, but instead of keeping them entirely separate or allowing for a coercion from non-linear types to linear types, the coercion is inverted, allowing for the conversion of linear types to non-linear types. The motivation for this idea is that in a linear type system, using the coercion from non-linear to linear precludes linearity from guaranteeing the referential uniqueness of a variables.

Respectively, as long as there is a coercion from linear types to non-linear types, uniqueness type systems refer to linear types as ``unique'' and non-linear types as ``non-unique'' or ``shared''. The temporal guarantee becomes inverted: While linear types ensure that a variable is always used exactly once in the future, uniqueness types ensure that a variable has always been used exactly once in the past \citep{vries_making_2009}\citep{sergey_linearity_2022}.

Unfortunately, the original type system was formulated in terms of graph rewriting and not lambda calculus, which meant that advances by the rest of the type theory research community were difficult to transfer over to uniqueness typing, a deficit that was only resolved much later by \cite{vries_making_2009}. In his thesis, de Vries first provides a uniqueness type system based on lambda calculus resembling that of Clean and then iteratively refines it with the goal of introducing higher-rank polymorphism \citep{peyton_jones_practical_2007}.

In uniqueness type systems, there are a number of challenges that do not appear in linear type systems. The first is that types are inherently less composable. In linear type theory, if \textsc{Dereliction} and \textsc{Promotion} are assumed, non-linear and linear types can be mixed freely, as long the provided resources are present to create them. For example, $!(\tau_1 \otimes \tau_2)$ is just a non-linear product, whereas $(!\tau_1) \otimes \tau_2$ is a linear product where the first type is non-linear. If $\tau_1$ is linear and \textsc{Dereliction} and \textsc{Promotion} are not assumed, the former example $!(\tau_1 \otimes \tau_2)$ is malformed, as deconstructing the product and obtaining the linear $\tau_1$ will not actually yield us a guarantee that the corresponding value has not been shared in the past, as e.g. the product could have been shared. The same is true for uniqueness type systems.

\section{Borrowing}\label{sec:borrowingbackground}

\section{Abstract Interpretation}
\chapter{Design Space Exploration}\label{sec:designspace}
In this chapter, we will explore possible design decisions with the goal of guaranteeing efficient in-place updates. First, we will evaluate linear type theory, quantitative type theory and uniqueness type theory, and argue why we decide for the latter. Second, we will discuss the dimensions in the design space of uniqueness type systems.

\section{Substructural Framework}
Let us first quickly repeat some of the important points from \Cref{sec:ltt}, \Cref{sec:qtt} and \Cref{sec:uniqueness} that are relevant to this section:
\begin{enumerate}
	\item Properly linear types make a guarantee for the usage of a variable in the future, but not the past.
	\item Uniqueness type theory makes a guarantee for the usage of a variable in the past, but not the future.
	\item Invariably unique types are always unique and can never discard their uniqueness.
	\item Properly linear types can make a guarantee for the past on particular types if all constructors of the type return linear values.
	\item Properly linear types can discard the linearity of particular types with special library functions, but are not capable of doing so in constant time for nested linear types.
	\item Quantitative type theory introduces an ``erasure'' quantity to specify computationally relevant terms in dependent types and count only those in the substructural portion of the type system.
	\item Quantitative type theory inherits the properties of proper linearity.
	\item Both quantitative and properly linear type theories can use linear functions that return values of linear type to produce values of non-linear type by demanding inputs of non-linear type to the function.
	\item Allowing the use of linear functions to produce non-linear outputs forces constructors of unique types to be formulated in continuation-passing style.
\end{enumerate}

\subsubsection{Properly linear types}
Because of 1.\ and despite 4.\ and 5., we believe that properly linear types are simply not the right tool for the job of ensuring uniqueness for destructive updates. 

Although they can be made to ensure uniqueness with careful library design, the tricks involved often seem to go against the grain of the initial decision to allow for a coercion from non-linear to linear types and are never without sizeable limitations, like the discarding of linearity only being available for a couple of types and nesting of such unique types being complicated. 

While \cite{spiwack_linearly_2022} manage to remove some of these limitations, they also add an entirely separate system of linear capabilities on top of the linear type system.

\subsubsection{Invariably unique types}
Meanwhile, due to 3., we think that invariably unique types are too restrictive for safe destructive updates. 

In all linear and uniqueness type systems known to us, there are significant limitations in expressivity, where it can occasionally happen that the type system cannot ensure the uniqueness of a value that is indeed referenced uniquely at runtime. Some of these limitations can be mitigated by choosing a different program encoding, while in other cases it is extremely difficult to do so. 

Ideally, as we are only concerned with efficient updates, we would like our type system to be fairly non-invasive, so as not to bother users. Hence, since the reference counting system described in \Cref{sec:beans} is already suitable to implement safe destructive updates at runtime, we think that it is absolutely crucial that users can fall back to this manual type-less mode of ensuring uniqueness if they cannot get the type system to guarantee the invariants that they need. This is especially pressing because we intend to add a substructural type system to the existing eco-system of Lean, where none of the code is annotated with linearity or uniqueness annotations. As per 2., uniqueness type theory has the desired property.

\subsubsection{Term language}
Finally, we must decide what the term language of our type system will be. 

Due to 7., 8.\ and 9., existing formulations of quantitative type theory are not readily suited for uniqueness. Graded modal dependent type theory \citep{moon_graded_2021} is still an active area of research and implementing something along these lines would be far beyond the scope of this thesis. Transferring the core idea of QTT from 6.\ and removing all the components that make QTT ``quantitative'' may be possible, but would both increase the complexity of the resulting system due to the extra erasure attribute and require additional adjustments to the existing compiler toolchain to respect erasure. 

Because of this, we decide against combining dependent type theory with uniqueness type theory and instead focus on a derivate of the LCNF language described in \Cref{sec:irs} in the later stages of the compiler toolchain when type dependencies have been erased. Type erasure will lose us some analysis precision, but in turn integrating the resulting type system with the Lean compiler toolchain should be much easier.

\section{Uniqueness Type Systems}
In \Cref{sec:uniqueness} and \Cref{sec:borrowing}, we describe two key challenges that come up when designing a uniqueness type system. The first is that non-unique values, including closures of non-unique functions, cannot be allowed to contain unique values. The second is that functions which use an argument in a read-only manner still consume it, thus losing the reference in the process.

\subsection{Higher-Order Functions}\label{sec:hof}
As discussed in \Cref{sec:uniqueness}, invariably unique type systems ensure that shared containers cannot contain unique values during the construction of a value, whereas Clean ensures it during the deconstruction of a value by disallowing the deconstruction of a shared value if it contains unique values, as types can lose their uniqueness. Higher-order functions complicate this matter because function types typically do not reveal the types of the values in their closure and because implementations usually commit to one particular function pointer when the higher order function is created. 

To see why this is an issue, consider a function $f : *(*\alpha \to *(!\beta \to *\alpha))$, using the notation introduced in \Cref{sec:uniqueness} where $*$ represents a unique type and $!$ represents a non-unique type. There are two ways in which the uniqueness of the first argument can be leveraged: In the construction of the return value of type $*\alpha$, and in the resulting code for the function $f$ where the first argument may get updated destructively. If we partially apply the first argument, the type becomes $f\ a : *(!\beta \to *\alpha))$ and the information that the first argument was unique is lost, despite $a$ being in the closure of $f\ a$. Now, if we discard the uniqueness of $f$ and apply it twice, then the return type may incorrectly suggest that the return value is unique, and we may even accidentally destructively update the first argument to the function, despite the fact that it is shared between the two function invocations.

There are several possible solutions to this problem, most of which have previously been covered by \cite{de_vries_making_2009}. 

\subsubsection{``Necessarily unique''}
The first is to take the approach that Clean uses and disallow discarding the uniqueness of function types altogether. Clean ensures this by introducing another kind of type into its type system, so called ``necessarily unique'' types. In practice, ``necessarily unique'' is the same thing that we have been calling ``invariably unique'' so far: the value is unique when it has been constructed and can never lose its uniqueness. Unfortunately, having support for ``necessarily unique'' values only in functions creates two additional problems.

First, in a type system with polymorphic types, the fact that functions in particular can be invariably unique also precludes type variables from discarding their uniqueness, as they could be substituted for functions.

And second, if an invariably unique function is stored in a unique container that then discards its uniqueness, we must make the function unavailable during the deconstruction of the value, lest it could have been shared. As argued in \Cref{sec:uniqueness}, for other unique values in non-unique containers, we do not have to be this restrictive and could instead also discard the uniqueness of the contained unique values during deconstruction.

Furthermore, a perspective put forward by \cite{sergey_linearity_2022} is that Clean's ``necessarily unique'' is too restrictive: For functions with unique values in their closure, we typically do not care about the uniqueness of the function itself, just that the function does not duplicate values in its closure. Hence, functions could really be properly linear, not invariably unique, allowing for some greater flexibility when joining two code paths where one yields a function with a unique value in its closure, whereas the other does not.

\subsubsection{Closure typing}
The second solution is what de Vries calls ``closure typing''---instead of disallowing the discarding of uniqueness of a function altogether, an additional attribute is added to every function type that denotes whether the function contains a unique value in its closure. Then, when applying the function, the information of whether the closure contains a unique value is not lost, and applying a shared function with a unique value in its closure becomes disallowed.

\subsubsection{Higher-rank polymorphism}
The third solution outlined by de Vries is to attempt to do away with the coercion altogether and replace it with careful library design, though of another nature than the freeze function in Linear Haskell. 

First, de Vries argues that constructors of functions should leverage polymorphism in order to create values of polymorphic kind, e.g.\ $\mathrm{mkArray} :\ !\mathbb{N} \to *\mathrm{Array}\ \alpha$ becomes $\mathrm{mkArray} : \forall u.\ !\mathbb{N} \to u(\mathrm{Array}\ \alpha)$, where $u \in \{*, !\}$. However, this is not equivalent to the approach with the coercion, as we must commit to a concrete $u$ when constructing the array, and two code paths cannot use the same array in two separate ways anymore. 

To mitigate this, de Vries suggests using higher-rank types so that the constructor is typed as $\mathrm{mkArray} : \ !\mathbb{N} \to (\forall u.\ u(\mathrm{Array}\ \alpha))$ and two different code paths can instantiate $(\forall u.\ u(\mathrm{Array}\ \alpha))$ in two separate ways. Unfortunately, this is still not equivalent to being able to discard uniqueness: Two code paths can now instantiate the array in two separate ways, but after they instantiate it as needed, they cannot be joined anymore, as the concrete instantiations $*\mathrm{Array}\ \alpha$ and $!\mathrm{Array}\ \alpha$ are not compatible.

\subsubsection{Deleveraging uniqueness}
Finally, the fourth solution outlined by de Vries, originally due to \cite{harrington_uniqueness_2006}, is to allow the application of shared functions with a unique value in their closure, but to degrade the return type of the function to non-unique instead. 

As de Vries correctly points out, if this idea is used in a programming language, then ways of leveraging the uniqueness of the function argument other than the uniqueness of the return value must be addressed as well, e.g.\ the presence of destructive updates in the function. For destructive updates in particular, when creating a higher-order function, one possible implementation might yield two function pointers for the higher-order function: one for if the function remains unique in which destructive updates are used, and another for if the function becomes non-unique in which no destructive updates are used. Then, when the unique function is forced to discard its uniqueness, the function pointer with destructive updates can be swapped out for a function pointer without destructive updates, and the now violated uniqueness of the function argument cannot be leveraged anymore.

Since Clean uses uniqueness not just for destructive updates, but for threading I/O as well, de Vries argues that the approach is inadequate, as side-effecting functions like $\mathrm{closeFile} : *\mathrm{File} \to\ !\mathbb{B}$ cannot be prevented from leveraging the uniqueness of the $*\mathrm{File}$ argument, and closing a file twice will always be an error. However, we believe that uniqueness types are simply the wrong tool to handle I/O, as it is never desirable to make I/O values shared, and that Clean should instead use invariably unique types for I/O. Uniqueness types are better suited for situations in which the sharedness of a value is still admissible, like destructive updates in memory, where we can always copy a value if it is shared.

\subsubsection{Conclusion}
We feel that the first approach is overkill. If one introduces linearity into a uniqueness type system for functions, then one should also introduce linearity for all other types, so that linear functions can be nested in linear structures. This is essentially the approach of \cite{sergey_linearity_2022}, though as stated in \Cref{sec:uniqueness}, their type system does not support nesting of unique types. This creates lots of extra work for users, and it seems excessive if all we want to do is guarantee safe destructive updates and handle higher-order functions correctly.

The second approach of closure typing also adds plenty of notational overhead to the type theory, as functions now have two separate type annotations, and the idea of not being able to apply a function that is present in the context seems unintuitive to us.

We think that the third approach introduces a lot of complexity in requiring higher-rank types to work, and even then, it does not fully replace the notion of discarding uniqueness.

Despite de Vries' criticism of it, we think that the last approach is the most viable if all we care about are destructive updates of values in memory. However, this approach would require a hefty change to the runtime of Lean, as every unique higher-order function needs to carry two function pointers around, so that we can switch to the one without destructive updates when the function becomes shared. Because of this, we decided not to implement this approach yet, and will instead require all higher-order functions and the types within them to be non-unique. As this precludes us from making guarantees for monadic code, type classes and idioms using higher-order functions like the one in \Cref{sec:beans}, this is a considerable limitation that we hope to resolve in the future.

\subsection{Implicit Coercion}\label{sec:coercions}
In most descriptions of linear and uniqueness type theory discussed so far, the coercion between non-linear/non-unique and linear/unique types has usually been implicit, e.g.\ passing a unique value to a shared parameter works, but will discard the uniqueness in the process. For linear type theories this is not an issue, as the coercion from non-linear types to linear types adds structure. However, for uniqueness type theories, the coercion from unique to shared discards the guarantee that the value is unique, and so the user may not want it to happen implicitly. 

Additionally, there is a question of when unique types are implicitly coerced. In Clean, this is only possible at function boundaries, and if one wishes to share a variable, then one must choose the respective parameter type accordingly. An explicit coercion operator would give users greater control over when coercion happens.

While the type theory that we will describe in \Cref{sec:theory} also uses an implicit coercion, we think it is best to make it explicit when integrating the type theory with Lean 4.

\subsection{Uniqueness Propagation}
As discussed in \Cref{sec:hof}, shared containers cannot be allowed to contain unique values. We have already explored possible approaches for how to ensure this for higher-order functions, but we are still lacking a more general framework for other types.

\subsubsection{Subtyping}
In Clean, a shared product is allowed to contain unique values, but upon deconstruction or projection, the type system checks that the projected values cannot be unique if the outer value is shared. In other words, if we discard the uniqueness of a product, then we cannot access its fields anymore. This is rather limiting, because we could instead also discard the uniqueness of the fields when we attempt to access them. 

Relatedly, there is a question of how deep the subtyping relation induced by the coercion from unique to non-unique types is. In Clean, it is very shallow and only allows discarding the uniqueness of the outer layer, i.e.\ we cannot pass a value of type $*(*\alpha \times *\beta)$ to a parameter of type $!(!\alpha\ \times\ !\beta)$, only a parameter of type $!(*\alpha \times *\beta)$, the fields of which cannot be accessed anymore. Similarly, passing a value of type $*(*\alpha \times *\beta)$ to a parameter of type $*(!\alpha \times *\beta)$ is not possible either, though doing so is sound. A less shallow subtyping relation would resolve these issues.

If we make the subtyping relation less shallow, then coercions should also propagate the change in the outer attribute to the values contained within, so that $!(*\alpha \times *\beta)$ and $!(!\alpha\ \times\ !\beta)$ are not differently annotated but equivalent types, and $!(*\alpha \times *\beta)$ becomes unrepresentable. For types like $*(*\alpha \times *\beta)$ where the attributes are floated out, this is straight-forward, as we can propagate the outer sharedness annotation to the inner components of the type.

\subsubsection{Algebraic data types}
For algebraic data types (ADTs), this is not as straight-forward, as there is a question of what to do with the attributes associated with fields of the ADT. One possibility would be to float out every attribute within the ADT, so that fields use an attribute variable $m$ to refer to a concrete attribute passed to the ADT, after which $!$ can be propagated directly in the arguments of the ADT, much like in $*(*\alpha \times *\beta)$. However, for large ADTs, this will very quickly accumulate dozens of attribute arguments, leading to a huge notational overhead.

Instead, for our type system described in \Cref{sec:theory}, we decide that fields with a $*$ attribute are ``unique if the outer value is unique''. In other words, for the uniqueness annotations within ADTs, sharedness is not propagated directly, only when deconstructing the value and accessing the fields. Since these attributes are not floated out, delaying the propagation does not make us end up with differently annotated types that are equivalent, like it would be the case for $*(*\alpha \times *\beta)$.

\subsection{Borrowing}
As discussed in \Cref{sec:borrowingbackground}, most implementations of borrowing use a form of type-driven escape analysis. We think that introducing extra attributes like observer-types into the type system that users need to keep accurate track of in order to be able to access borrowing is too much of an annotational burden.

Instead, we will make use of the fact that escape analyses are very local in nature; whether a variable escapes or not can be decided by following the data flow of the variable from the start of the function to the return value. Hence, we implement a data flow analysis \cite{allen_program_1976} in \Cref{sec:borrowing}. Due to the inherent locality of the analysis, we will not need any type information to run it.

While we will not touch on it further, it is worth pointing out that adding extra annotations describing whether a function parameter is borrowed may still be a good idea for maintenance purposes, despite the fact that we can compute this information. When integrating our type system with Lean 4, it may hence be a good idea to add these annotations as well.
\chapter{Formal description}\label{sec:theory}

In this chapter, we will provide a formal description of our type theory and all the associated mechanisms required to make it work. \Cref{sec:types} introduces the syntactical material for our types and declares a number of commonly useful utility functions. \Cref{sec:ir} defines the syntax of the IR. In \cref{sec:escapeanalysis}, we specify an abstract interpretation based escape analysis in order to implement the borrowing mechanism described in \cref{sec:borrowing}. Finally, \cref{sec:checking} provides the rules of our type theory.

\newcommand{\sep}{\ \ |\ \ }
\newcommand{\icode}[1]{\textrm{\lstinline[language=ir-if]|#1|}}

\section{Types}\label{sec:types}
In all of the following sections, we use $[x]$ to denote a vector of elements $x$, otherwise commonly written as $\overline{x}$. We will often lift these brackets over an operation; e.g. the functional code $\mathrm{map}(\oplus, \mathrm{zip}([x], [y]))$ is written as $[x \oplus y]$ for vectors $[x]$ and $[y]$. In derivation rules, we also use $[x]$ for $x \in \mathbb{B}$ to mean $\forall x \in [x].\ x$.

\subsection{Syntax}

\newcommand{\dom}{\mathrm{dom}}
\newcommand{\Var}{\mathrm{Var}}
\newcommand{\Ctor}{\mathrm{Ctor}}
\newcommand{\Proj}{\mathrm{Proj}}
\newcommand{\Const}{\mathrm{Const}}
\newcommand{\Attr}{\mathrm{Attr}}
\newcommand{\ADT}{\mathrm{ADT}}
\newcommand{\adt}{\mathrm{adt}}
\newcommand{\field}{\mathrm{field}}
\newcommand{\ADTConst}{\mathrm{ADTConst}}
\newcommand{\AttrType}{\mathrm{AttrType}}
\newcommand{\arrg}{\mathrm{arg}}
\newcommand{\param}{\mathrm{param}}
\newcommand{\ret}{\mathrm{ret}}
\newcommand{\ADTDecls}{\mathrm{ADTDecls}}
\newcommand{\FunTypes}{\mathrm{FunTypes}}

\begin{alignat*}{2}
  x, y, z &\in \Var \\
  i &\in \Ctor \\
  j &\in \Proj \\
  c &\in \Const \\
  m &\in \Attr &\Coloneqq&\ ! \sep * \\
  a &\in \ADT &\Coloneqq&\ \mu\ x^\kappa_\adt.\ [[\tau_\field(x^\kappa_\adt, [y^\tau])] \to *x^\kappa_\adt] \\
  A &\in \mathrlap{\ADTConst} \\
  \gamma &\in \ADTDecls &=&\ \ADTConst \rightharpoonup \ADT \\
  \tau &\in \AttrType &\Coloneqq&\ m\ x^\kappa \sep x^\tau \sep m\ \blacksquare \sep m\ A\ [\tau_\arrg] \sep !\ [\tau_\param] \to \tau_\ret \\
  \delta_\tau &\in \FunTypes &=&\ \Const \rightharpoonup [\AttrType] \times \AttrType
\end{alignat*}

Ctor and Proj denote the constructors and fields within a constructor, respectively. Const designates function names. Attr contains the attributes that are the main subject of our type theory; shared (!) and unique ($*$). 

\sloppy Since the Lean 4 compiler erases type dependencies\footnote{\todo{Remember inserting examples into the IR section in the background}}, we will limit ourselves to types that look like potentially recursive algebraic data types. In $\mu\ x^\kappa_\adt.\ [[\tau_\field(x^\kappa_\adt, [y^\tau])] \to *x^\kappa_\adt]$, $x^\kappa_\adt$ is the variable we use to refer back to the ADT itself, $[[\tau_\field(x^\kappa_\adt, [y^\tau])] \to *x^\kappa_\adt]$ is a vector of constructors and $[\tau_\field(x^\kappa_\adt, [y^\tau])]$ denotes the types of the fields of the constructor, where $\tau_\field$ is parametrized by the variable $x^\kappa_\adt$ representing the ADT itself, as well as a vector $[y^\tau]$ of type parameters to the ADT. Constructors, projections and type parameters are assumed to be enumerated by intervals $[0, n)$, and so we write $(\mu\ x^\kappa_\adt.\ [[\tau_\field(x^\kappa_\adt, [y^\tau])] \to *x^\kappa_\adt])_i \coloneqq [[\tau_\field(x^\kappa_\adt, [y^\tau])] \to *x^\kappa_\adt]_i$ and $([\tau_\field(x^\kappa_\adt, [y^\tau])] \to *x^\kappa_\adt)_j \coloneqq [\tau_\field(x^\kappa_\adt, [y^\tau])]_j$, as well as $[y^\tau]_x \coloneqq x$. As Lean 4 code commonly interacts with external types and external code via its foreign function interface (FFI), we cannot assume that we can access an ADT declaration for every type. To deal with this, ADTs are instead identified by an ADTConst, the mapping of which is maintained in a global and partial function $\gamma \in \ADTDecls$. ADTConsts $A \notin \dom(\gamma)$ that appear in the program are regarded as external. Lastly, we demand that all $A \in \dom(\gamma)$ are fully propagated, i.e. that $\forall i\ j.\ \mathrm{propagate}(\gamma(A)_{ij}) = \gamma(A)_{ij}$ for the definition of propagate below.

AttrType contains our types. $m\ x^\kappa$ and $x^\tau$ are the two kinds of variables that can occur only within an ADT; self-referring variables $x^\kappa$ have an associated (fixed) attribute and the variable only represents the parameterless portion of a type, while variables $x^\tau$ can denote any type parameter $\tau \in \AttrType$. $m\ \blacksquare$ is an erased type, $m\ A\ [\tau_\arrg]$ is an ADT (or external type) $A$ parametrized by type arguments $[\tau_\arrg]$, and $!\ [\tau_\param \to \tau_\ret]$ is the type of a higher-order function. Finally, $\delta_\tau$ provides the parameter- and return types for all functions in the program, including external ones. This is a reasonable assumption because we can simply assign a type $[!\ \kappa_\param] \to \ !\ \kappa_\ret$ for Lean 4 functions with an unattributed function type $[\kappa_\param] \to \ \kappa_\ret$. For $\delta_\tau(c) = ([\tau_\param], \tau_\ret)$, we also demand that all the types are fully propagated, i.e. that $\forall \tau_\param \in [\tau_\param].\ \mathrm{propagate}(\tau_\param) = \tau_\param \land \mathrm{propagate}(\tau_\ret) = \tau_\ret$ for the definition of propagate below.

\subsection{Propagation}

\newcommand{\propagateWithinShared}{\mathrm{propagateWithinShared}}
\newcommand{\rebreak}[1]{\mathrlap{\qquad#1}}

\begin{alignat*}{3}
  &\propagateWithinShared &&: \mathrlap{\AttrType \to \AttrType} \\
  &\propagateWithinShared&&(m\ x^\kappa) &&=\ !\ x^\kappa \\
  &\propagateWithinShared&&(x^\tau) &&= x^\tau \\
  &\propagateWithinShared&&(m\ \blacksquare) &&=\ !\ \blacksquare \\
  &\propagateWithinShared&&(m\ A\ [\tau_\arrg]) &&=\ !\ A\ [\propagateWithinShared(\tau_\arrg)] \\
  &\propagateWithinShared&&\mathrlap{(!\ [\tau_\param] \to \tau_\ret)} \\
  &\rebreak{= \ !\ [\propagateWithinShared(\tau_\param)] \to \propagateWithinShared(\tau_\ret)}
\end{alignat*}

\newcommand{\propagate}{\mathrm{propagate}}

\begin{alignat*}{3}
  &\propagate &&: \mathrlap{\AttrType \to \AttrType} \\
  &\propagate&&(m\ x^\kappa) &&= m\ x^\kappa \\
  &\propagate&&(x^\tau) &&= x^\tau \\
  &\propagate&&(m\ \blacksquare) &&= m\ \blacksquare \\
  &\propagate&&(*\ A\ [\tau_\arrg]) &&= *\ A\ [\propagate(\tau_\arrg)] \\
  &\propagate&&(!\ A\ [\tau_\arrg]) &&= \ !\ A\ [\propagateWithinShared(\tau_\arrg)] \\
  &\propagate&&\mathrlap{(!\ [\tau_\param] \to \tau_\ret)} \\
  &\rebreak{= \ !\ [\propagateWithinShared(\tau_\param)] \to \propagateWithinShared(\tau_\ret)}
\end{alignat*}

propagate ensures that unique types are made shared if they are contained within a shared type, since a value within another value cannot be guaranteed to be unique if the outer value is already shared. 

We use the following notation for substitution in $a = \mu\ x^\kappa_\adt.\ [[\tau_\field(x^\kappa_\adt, [y^\tau])] \to *x^\kappa_\adt]$:
\begin{alignat*}{3}
	a\{A, [\tau]\}\ &&\coloneqq&\ \mu\ x^\kappa_\adt.\ [[\propagate(\tau_\field[A\ [\tau]/x^\kappa_\adt][[\tau]/[y^\tau]])] \to *x^\kappa_\adt]
\end{alignat*}

\subsection{Definitional Nuances}

It is worth pointing out a number of semantic nuances in both the definitions of our types and propagate above:
\begin{itemize}
	\item If we know that a type is unique, we can always throw away this guarantee and make it shared, as described in \cref{sec:uniqueness}.
	\item In $m\ \blacksquare$, $\blacksquare$ could be any other type, potentially parametrized by any other attributed type if $\blacksquare$ used to be $A\ [\tau_\arrg]$. We must ensure that our type theory can deal with this kind of erasure.
	\item We must always ensure that types remain fully propagated.
	\item Within an ADT declaration $a$, we do not know how to propagate $x^\tau$, as it depends on the concrete parameter type. Instead, we ensure that type parameters become fully propagated when substituting the type variables for type parameters using our definition of $a\{A, [\tau]\}$.
	\item While we can propagate within a given ADT field or within any other given type, we cannot propagate from an outer $!\ A\ [\tau_\arrg]$ into the fields within $\gamma(A)$, as not all the attributes in $\gamma(A)$ are floated to the outer $!\ A\ [\tau_\arrg]$, only those in the type parameters $[\tau_\arrg]$. To alleviate this issue, we take an attribute $*$ in a field within $\gamma(A)$ to mean ``Unique if the outer value is unique'' and enforce this property in our type rules for projections on $m\ A\ [\tau_\arrg]$.
	\item Higher-order functions are always shared, so we do not need to worry about covariance or contravariance in either propagate or type parameters. This is a considerable limitation: Lean 4 code uses higher-order functions very liberally to encode type classes, monads, as well as some performance idioms related to the Counting Immutable Beans optimization described in \cref{sec:beans}. See \footnote{\todo{add refs}} for possible approaches to alleviate this issue in future work.
	\item Since external types have no associated declaration, if we want to gather information about the type, we must rely on auxiliary information provided by users at the FFI. We will need this kind of auxiliary information in \cref{sec:escapeanalysis} and \cref{sec:borrowing}.
\end{itemize}

\subsection{Utilities}

We will now proceed to declare some convenient auxiliary functions. 

\newcommand{\weaken}{\mathrm{weaken}}

\begin{alignat*}{3}
  \weaken &: \mathrlap{\AttrType \to \AttrType} \\
  \weaken&(\tau) &&= \propagateWithinShared(\tau) \\
  !&(\tau) &&= \weaken(\tau)
\end{alignat*}

weaken makes the type argument $\tau$ shared and then propagates the attribute through the type. We will need this function whenever we have to make a type shared and we will avoid using it for the types of fields that have not been substituted yet.

\newcommand{\weakenInner}{\mathrm{weakenInner}}

\begin{alignat*}{3}
	\weakenInner &: \mathrlap{\AttrType \to \AttrType} \\
	\weakenInner&(m\ x^\kappa) &&= m\ x^\kappa \\
	\weakenInner&(x^\tau) &&= x^\tau \\
	\weakenInner&(m\ \blacksquare) &&= m\ \blacksquare \\
	\weakenInner&(m\ A\ [\tau_\arrg]) &&= m\ A\ [!(\tau_\arrg)] \\
	\weakenInner&(!\ [\tau_\param] \to \tau_\ret) &&=\ !\ [\tau_\param] \to \tau_\ret \\
\end{alignat*}

weakenInner leaves the outer attribute intact but weakens every inner type. This will be useful when dealing with erased types $m\ \blacksquare$: When casting $m\ \blacksquare$ to another type $\tau$, we want $\tau$ to retain the outer attribute $m$, but we cannot make any guarantees for the inner attributes, and so we weaken them. We will avoid using it for the types of fields that have not been substituted yet.

\newcommand{\strengthen}{\mathrm{strengthen}}

\begin{alignat*}{3}
  \strengthen &: \mathrlap{\AttrType \to \AttrType} \\
  \strengthen&(m\ x^\kappa) &&= *\ x^\kappa \\
  \strengthen&(x^\tau) &&= x^\tau \\
  \strengthen&(m\ \blacksquare) &&= *\ \blacksquare \\
  \strengthen&(m\ A\ [\tau_\arrg]) &&= *\ A\ [\strengthen(\tau_\arrg)] \\
  \strengthen&(!\ [\tau_\param] \to \tau_\ret) &&=\ !\ [\tau_\param] \to \tau_\ret \\
\end{alignat*}

strengthen makes every attribute within a type unique which can be made unique. We will use this function for inferring the type parameters of $m\ A\ [\tau_\arrg]$ at construction: If a type parameter variable is not assigned by any constructor argument, we can strengthen it. We will also avoid using it for the types of fields that have not been substituted yet.

\newcommand{\attr}{\mathrm{attr}}

\begin{alignat*}{3}
  \attr &: \mathrlap{\AttrType \rightharpoonup \Attr} \\
  \attr&(m\ x^\kappa) &&= m \\
  \attr&(m\ \blacksquare) &&= m \\
  \attr&(m\ A\ [\tau_\arrg]) &&= m \\
  \attr&(!\ [\tau_\param] \to \tau_\ret) &&=\ ! \\
\end{alignat*}

attr simply yields the outer attribute of any $\tau \neq x^\tau$.

\subsection{Subtyping}

Finally, whenever we pass a type $\tau_1$ to a type $\tau_2$, we must ask ourselves whether $\tau_1$ can be applied to $\tau_2$. The type structure must be the same, but it should be possible to throw away the uniqueness attribute of types within $\tau_1$. Hence, we use $m_1 \succcurlyeq_m m_2 :\Leftrightarrow m_1 = * \lor m_1 =\ ! \land m_2 =\ !$ to denote attribute subtyping and define a subtyping relation $\succcurlyeq$ for fully propagated types $\tau_1$ and $\tau_2$ as follows:
\begin{mathpar}
	\boxed{\tau_1 \succcurlyeq \tau_2} \hspace{1.5em}
	$\inferrule{m_1 \succcurlyeq_m m_2}{m_1\ \blacksquare \succcurlyeq m_2\ \blacksquare}$ \hspace{1.5em}
	$\inferrule{m_1 \succcurlyeq_m m_2}{m_1\ x^\kappa \succcurlyeq m_2\ x^\kappa}$ \hspace{1.5em}
	$\inferrule{ }{x^\tau \succcurlyeq x^\tau}$
\end{mathpar}
\begin{mathpar}
	$\inferrule{ }{!\ [\tau_\param] \to \tau_\ret \succcurlyeq\ !\ [\tau_\param] \to \tau_\ret}$ \hspace{1.5em}
	$\inferrule{m_1 \succcurlyeq_m m_2 \\ [\tau_{\arrg_1} \succcurlyeq \tau_{\arrg_2}]}{m_1\ A\ [\tau_{\arrg_1}] \succcurlyeq m_2\ A\ [\tau_{\arrg_2}]}$
\end{mathpar}
Note that if higher-order functions could be unique, we would have to account for covariance and contravariance in this definition.

\section{Intermediate Representation}\label{sec:ir}
For our model of the IR, we use a mixture of the IR described by \cite{ullrich_counting_2020} and the newly implemented LCNF, both detailed in \cref{sec:irs}. 

\subsection{Syntax}

\newcommand{\Expr}{\mathrm{Expr}}
\newcommand{\FnBody}{\mathrm{FnBody}}
\newcommand{\Fn}{\mathrm{Fn}}
\newcommand{\Program}{\mathrm{Program}}

\begin{alignat*}{3}
  \icode{e} &\in \Expr &\Coloneqq&\ \icode{c [y]}
    \sep \icode{pap c [y]}
    \sep \icode{x y}
    \sep \icode{(A [τ?]).ctorᵢ [y]} \\
    &&&\enspace\sep \icode{projᵢⱼ y} \\
  \icode{F} &\in \FnBody &\Coloneqq&\ \icode{ret x}
    \sep \icode{let x := e; F}
    \sep \icode{case x of [F]} \\
    &&&\enspace\sep \icode{case' x of [ctorᵢ [y] ⇒ F]}\\
  f &\in \Fn &\Coloneqq&\ \lambda\ \icode{[y]}.\ \icode{F} \\
  \delta &\in \Program &=&\ \Const \rightharpoonup \Fn
\end{alignat*}

Expr and FnBody are similar to Lean's IR, except for our definition of \icode{proj} and \icode{ctor}, as well as the addition of a new instruction \icode{case'}.

\icode{proj} is provided not just with the projection \icode{j} as in Lean's IR, but also the constructor \icode{i}. As the code generation ensures that \icode{proj} calls always occur after \icode{case} within the same function if the type has multiple constructors or on its own if the type only has a single constructor, we can easily compute \icode{i} by walking back from the \icode{projⱼ y} call either to the start of the function to set $\icode{i} = 0$ or to a \icode{case x of [...]} instruction, where we choose \icode{i} as the index of the branch that we are walking back from.

\icode{ctor} takes an additional vector of explicit attributed type arguments \icode{[τ?]}, where \icode{?} refers to each explicit argument being optional. Since users do not provide them, Lean can provide us with type arguments \icode{[κ]} for the constructor call, but not any of the attributes, and so we must infer them from the types of arguments provided in \icode{[y]}. But since there may be type arguments to \icode{A} that occur only in the other constructors for \icode{A}, we cannot infer all of them, and so they must be provided explicitly. However, type arguments that do not occur in \icode{[y]} are also not subject to any uniqueness constraints, and so we can instantiate them as strongly as possible. Subsequently, the attributes in \icode{[τ?]} can be chosen arbitrarily: If the type argument occurs in \icode{[y]}, we can infer the type together with its attributes, and if it does not occur in \icode{[y]}, the type $\tau_e$ must be provided in \icode{[τ?]}, but the corresponding attributes can be chosen as given by $\strengthen(\tau_e)$.

The \icode{case} and \icode{proj} combination turns out to be unwieldy for substructural type systems: When we use \icode{let z := projᵢⱼ y; F} on a unique value to obtain another unique value, the contained value now exists both in \icode{z} and in \icode{y}, i.e. uniqueness is violated. The solution to this issue would be that \icode{projᵢⱼ y} consumes our unique value \icode{y} so that it is not available in \icode{F} any more. However, it is very common that we would like to access multiple fields of \icode{y} in succession, which we will not be able to do now that \icode{y} is consumed. So, instead, the typical solution to this issue in substructural type systems is not to access fields via projections, but using a single destructuring pattern match that yields all fields of the type in one go and consumes the variable associated with the type. This is exactly what \icode{case'} does as well, and while the instruction does not exist in Lean's IR, it does exist in Lean 4's LCNF. Regardless, even in LCNF, structures are still accessed via projections and not using a destructuring pattern match. To alleviate this final issue, we implement a compromise in \cref{sec:checking} which ensures that we can use multiple projections on \icode{y}, but not use it in any other manner.

As in Lean's IR, a global and partial $\delta \in \Program$ assigns function declarations to constants. All $c \notin \dom(\delta)$ that occur in the program are assumed to be external functions.

Finally, there are a number of omissions from our IR compared to the IR implemented in Lean. Most notably, there are instructions to work with join points, which we could implement as functions in our IR. However, it is worth noting that join points, like auxiliary functions $c$ generated by the Lean compiler, do not necessarily have an associated user-provided type $\delta_\gamma(c)$. Unfortunately, we will not touch on the topic of type inference in this thesis.

\newcommand{\Tag}{\mathrm{Tag}}
\newcommand{\Escapee}{\mathrm{Escapee}}
\newcommand{\ExternFunEscapees}{\mathrm{ExternFunEscapees}}

\section{Escape Analysis}\label{sec:escapeanalysis}
As described in \cref{sec:borrowingbackground}, borrowing in functional languages is closely related to escape analysis; if nothing within a shared parameter escapes, then we do not have to make a unique argument to that parameter shared, as the caller is guaranteed to still hold the only reference to the object in question when the called function returns. Instead of unloading this additional burden of tracking the data flow of variables and fields to the user, we implement a data flow analysis, i.e. an instance of abstract interpretation.\footnote{\todo{add note that explicit attribution might nonetheless be helpful}}

\subsection{Syntax}

\begin{alignat*}{3}
  n, m &\in \mathbb{N} \\
  s, t, v &\in \Tag &\Coloneqq&\ \#\textrm{const c} \sep \#\textrm{case i} \sep \#\textrm{app} \sep \#\textrm{param n} \\
  q &\in \Escapee\ &\Coloneqq&\ x_{[ij]} @ [t]? \\
  \delta_{q_e} &\in \ExternFunEscapees &=&\ \Const \rightharpoonup 2^{\Escapee}
\end{alignat*}

Escapees are the subject of our escape analysis and represent the elements of the sets that we compute. Each escapee has an associated variable $x$, a field index $[ij]$ represented by a vector of $\Ctor \times \Proj$ tuples and a vector of tags that describes the path to the parameter an escapee was spawned from, if the escapee came from a function call. The need for the vector of tags will become obvious later, and until then it can just be understood as an identifier that identifies the location in the code where escapees from function calls were spawned. Since external funtions do not have a function body that we can analyze, a global and partial function $\delta_{q_e}$ allows specifying the set of all escapees for these functions. For external functions $c \notin \dom(\delta_{q_e})$ we will assume all parameters and all fields thereof to escape.

\subsection{Computing Escapees}

\newcommand{\ecp}[2]{\llbracket {#1} \rrbracket_Q \left( {#2} \right)}

Using abstract interpretation, we compute a least fixed point of the following mutually recursive equations $\ecp{\cdot}{\cdot}$ and $\delta_Q$, which we will explain in detail along the way. The first parameter of $\ecp{\cdot}{\cdot}$ is the portion of the function body that we want to compute the escapees for, the second parameter denotes the vector of tags thus far from the start of the function to this portion of the function body.

\begingroup
\allowdisplaybreaks
\begin{align*}
  &\ecp{\cdot}{\cdot} : \mathrlap{\FnBody \times [\Tag] \to 2^{\Escapee}} \\
  &\ecp{\icode{ret x}}{[t]} =
    \left\{\icode{x}_{[]}\right\} \\
  &\ecp{\icode{case x of [F]}}{[t]}\ =\
    \bigcup_n \ecp{\icode{[F]}_n}{\#\text{case n} :: [t]} \\
  &\ecp{\icode{case' x of [ctorᵢ [y] ⇒ F]}}{[t]}\ =\
    \bigcup_n \ecp{\icode{[F]}_n}{\#\text{case n} :: [t]} \\
    &\rebreak{\cup \left\{\icode{x}_{nm :: [kj]} @ [s]?
    \ \ |\ \ (\icode{[y]}_m)_{[kj]} @ [s]? \in \ecp{\icode{[F]}_n}{\#\text{case n} :: [t]}\ \land\ m \in [0, |\icode{[y]}|) \right\}} \\
  &\ecp{\icode{let x = c [y]; F}}{[t]} = Q_F \\
  &\rebreak{\cup \begin{cases}
  	Q & \exists [nm].\ \icode{x}_{[nm]} @ [v]? \in Q_F \land \icode{c} \in \dom(\delta_Q) \\
    \left\{\icode{y}_{[]} \ | \ \icode{y} \in \icode{[y]}\right\} & \exists [nm].\ \icode{x}_{[nm]} @ [v]? \in Q_F \land \icode{c} \notin \dom(\delta_Q) \land \icode{c} \notin \dom(\delta) \\
  	\emptyset & \exists [nm].\ \icode{x}_{[nm]} @ [v]? \in Q_F \land \icode{c} \notin \dom(\delta_Q) \land \icode{c} \in \dom(\delta) \\
  	\emptyset & \lnot \exists [nm].\ \icode{x}_{[nm]} @ [v]? \in Q_F
  \end{cases}} \\
  &\rebreak{\qquad\text{where } Q := \left\{(\icode{[y]}_z)_{[ij]} @ (\#\text{param z} :: [t]) \ |\ z_{[ij]} @ [s]? \in \delta_Q(\icode{c}) \right\}}\\
  &\rebreak{\qquad\text{ and } Q_F := \ecp{\icode{F}}{\#\text{app}::[t]}}\\
  &\ecp{\icode{let x = pap c [y]; F}}{[t]} = \ecp{\icode{F}}{[t]} \\
  &\rebreak{\cup \begin{cases}
  	\left\{\icode{y}_{[]} \ | \ \icode{y} \in \icode{[y]}\right\} & \exists [nm].\ \icode{x}_{[nm]} @ [v]? \in \ecp{\icode{F}}{[t]} \\
  	\emptyset & \lnot \exists [nm].\ \icode{x}_{[nm]} @ [v]? \in \ecp{\icode{F}}{[t]}
  \end{cases}}\\
  &\ecp{\icode{let x = y z; F}}{[t]} = \ecp{\icode{F}}{[t]} \\
  &\rebreak{\cup \begin{cases}
      \left\{\icode{y}_{[]}, \icode{z}_{[]}\right\} & \exists [nm].\ \icode{x}_{[nm]} @ [v]? \in \ecp{\icode{F}}{[t]} \\
      \emptyset & \lnot \exists [nm].\ \icode{x}_{[nm]} @ [v]? \in \ecp{\icode{F}}{[t]}
  	\end{cases}}\\
  &\ecp{\icode{let x = (A [τ?]).ctorᵢ [y]; F}}{[t]} = \ecp{\icode{F}}{[t]} \\
  &\rebreak{\cup \begin{cases}
     	\left\{\icode{y}_{[]} \ | \ \icode{y} \in \icode{[y]}\right\} & \icode{x}_{[]} @ [v]? \in \ecp{\icode{F}}{[t]} \\
     	\left\{(\icode{[y]}_j)_{[nm]} @ [s]? \ | \ \icode{x}_{\icode{i}j :: [nm]} @ [s]? \in \ecp{\icode{F}}{[t]} \right\} & \icode{x}_{[]} @ [v]? \notin \ecp{\icode{F}}{[t]}
  \end{cases}} \\
  &\ecp{\icode{let x = projᵢⱼ y; F}}{[t]} = \ecp{F}{[t]} \\
  &\rebreak{\cup \begin{cases}
    \left\{\icode{y}_{\icode{ij} :: [kl]} @ [s]? \ | \ \icode{x}_{[kl]} @ [s]? \in \ecp{\icode{F}}{[t]} \right\} & \exists [nm].\ \icode{x}_{[nm]} @ [v] \in \ecp{\icode{F}}{[t]} \\
    \emptyset & \lnot \exists [nm].\ \icode{x}_{[nm]} @ [v]? \in \ecp{\icode{F}}{[t]}
  \end{cases}}
\end{align*}
\endgroup
In \icode{ret x}, only \icode{x} itself escapes. For \icode{case x of [F]}, we determine the escapees of each $\icode{F} \in \icode{[F]}$ and compute the resulting union of all escapees. In \icode{case' x of [ctorᵢ [y] ⇒ F]}, we use the same idea as for \icode{case}, but must also transfer escapees concerning \icode{[y]} over to \icode{x}: each escapee $(\icode{[y]}_m)_{[kj]} @ [s]?$ in branch $n$ corresponding to constructor $n$ is converted to an escapee $\icode{x}_{nm :: [kj]} @ [s]?$. 

For all \icode{let x = e; F} function bodies, we will always have a case stating that if \icode{x} does not escape in \icode{F}, then neither do we need to compute any additional escapees for \icode{e}.

Application \icode{let x = c [y]; F} is the most tricky since it is the spot where our analysis recurses with escapees for \icode{c}. If we have already computed escapees for \icode{c} or they are specified in $\delta_{q_e}$, i.e. $\icode{c} \in \dom(\delta_Q)$, we take all escapees $z_{[ij]}@[s]?$ for parameters $z$ from $\delta_Q$ and rename them to the corresponding arguments $\icode{[y]}_z$. If $\icode{c} \notin \dom(\delta_{q_e})$ is external, all \icode{[y]} are assumed to escape. Finally, if we have not already computed the escapees for \icode{c} but are expected to do so in the future because \icode{c} is not external, we yield the bottom element of our lattice $\bot = \emptyset$.

When creating a higher-order function using \icode{pap c [y]}, we assume that all \icode{[y]} escape if the resulting higher-order function escapes. The same is true for higher-order function application \icode{y z}: If the result escapes, then so may \icode{y} and \icode{z}. Note that creating an escapee for \icode{y}, i.e. the higher-order function itself, is important, because if the higher-order function containing all the previously-applied arguments or the return value of the function escapes, we need to know that the previously-applied arguments may escape too, and so we propagate this bit of information backwards using the escapee for \icode{y}.

\icode{(A [τ?]).ctorᵢ [y]} and \icode{projᵢⱼ y} are once again fairly straight-forward. If the ADT resulting from a constructor call escapes, then so do all of its fields, and if only particular fields of constructor \icode{i} escape, then the respective escapees $\icode{x}_{\icode{i}j :: [nm]} @ [s]?$ must be translated to escapees for $\icode{[y]}_j$ by removing the $\icode{i}j$ field. Other escapees $\icode{x}_{kj :: [nm]} @ [s]?$ for $k \neq \icode{i}$ do not need to be translated.
For \icode{projᵢⱼ y}, we use the same idea as for \icode{case'} and translate the escapees for the projection to ones for \icode{y}.

Next, we will define a couple of post-processing functions to make our application of abstract interpretation to the mutually recursive $\ecp{\cdot}{\cdot}$ and $\delta_Q$ terminate. 

\newcommand{\fd}{\mathrm{fd}}
\newcommand{\ct}{\mathrm{ct}}
\newcommand{\fs}{\mathrm{fs}}

\newcommand{\collapse}{\mathrm{collapse}}

\begin{alignat*}{3}
	\fd &: \mathrlap{\mathbb{N} \times 2^{\Escapee} \to 2^{\Escapee}} \\
	\fd&(\mathrm{arity}, Q) &&= \left\{x_{[ij]}@[t]? \ |\ x_{[ij]}@[t]? \in Q \land x \in [0, \mathrm{arity}) \right\}
\end{alignat*}
fd removes all dead escapees for a function of a particular arity by keeping only those that correspond to function parameters. This is mainly useful for performance because $\ecp{\cdot}{\cdot}$ accumulates escapees for all variables in a function, even local ones.

\newcommand{\concat}{\mathrm{++}}

\begin{alignat*}{3}
	\fs &: \mathrlap{2^{\Escapee} \to 2^{\Escapee}} \\
	\fs&(Q) &&= \left\{q \ |\ q \in Q \land \lnot \exists q' \in Q.\ q \neq q' \land q' \subset q\right\}
\end{alignat*}
Here, $x_{[i_1j_1]} @ [t_1]? \subset x_{[i_1j_1]\concat[i_2j_2]} @ [t_2]?$ asserts that $x_{[i_1j_1]} @ [t_1]?$ subsumes $x_{[i_1j_1]\concat[i_2j_2]} @ [t_2]?$. Hence, fs removes all escapees which are subsumed by another escapee.

\begin{alignat*}{1}
	&\equiv_t : \Escapee \times \Escapee \to \mathbb{B} \\
	&x_{[ij]} @ [s]? \equiv_t y_{[kl]} @ [v]? :\Leftrightarrow [s?] = [v?]
\end{alignat*}
\begin{align*}
	\collapse &: \Escapee \times \Escapee \rightharpoonup \Escapee \\
	\collapse&(x_{[i_1j_1]} @ [t]?, x_{[i_2j_2]} @ [t]?) = x_{lcp([i_1j_1], [i_2j_2])} @ [t]?
\end{align*}
\begin{alignat*}{3}
	\ct &: \mathrlap{2^{\Escapee} \rightharpoonup 2^{\Escapee}} \\
	\ct&(Q) &&= \left\{\mathrm{fold}(\collapse, [x_{[ij]} @ [s]])\ |\ [x_{[ij]} @ [s]] \in Q/{\equiv_t} \right\}
\end{alignat*}
ct is the key post-processing function that makes our escape analysis terminate and finally makes use of the tags that we have been keeping track of in $\ecp{\cdot}{\cdot}$. The key idea is that we take escapees with the same vector of tags, i.e. equivalence classes in $Q/{\equiv_t}$, and collapse them so that we get an escapee with a field that is the longest common prefix of all the fields of escapees in an equivalence class. Note that in such an equivalence class, as all escapees have been created from the same parameter at the same call site, all these escapees must use the same variable.

Without ct, the corresponding lattice over $2^{\Escapee}$ does not have finite height: In an escapee $x_{[ij]} @ [t]?$, we can bound $x$ by all the variables that are possible in the program and $t?$ by every single call site in the program, but the field $[ij]$ may diverge, e.g. when attempting to run the abstract interpretation on a recursive function with a \icode{List}-type, in which case we will keep prepending fields to the respective escapee, and they will not subsume one another. 

Collapsing the escapees with the same tag effectively bounds the lattice: Because there are only finitely many call sites, if the abstract interpretation is executed on a program where it would otherwise diverge, it must necessarily eventually visit the same call site twice and yield an escapee with the same variable but a different field. Collapsing all these escapees from the same call site thus computes a more general escapee that subsumes all the previous escapees from that call site, ensuring that we can iteratively reduce the field in which we were diverging up to a bound of $[]$, where we are guaranteed to terminate.

\begin{align*}
	\delta_Q &: \mathrlap{\Const \rightharpoonup 2^{\Escapee}} \\
	\delta_Q&(c) = \begin{cases}
		\fs(|\icode{[y]}|, \ct(\fd(\ecp{\icode{F}}{[\#\text{const c}]}))) & c \in \dom(\delta) \land \delta(c) = \lambda\ \icode{[y]}.\ \icode{F} \\
		\delta_{q_e}(c) & c \notin \dom(\delta) \land c \in \dom(\delta_{q_e})
	\end{cases}
\end{align*}

Finally, $\delta_Q$ computes the escapees of every function in the program and and uses $\delta_{q_e}$ to obtain escapees for some external functions.

\section{Borrowing}\label{sec:borrowing}
When checking whether a parameter can be borrowed, we do not need to check whether fields that are always shared escape, only fields that are unique if their outer value is unique. In this section, we will define the function $\delta_\mathbb{B}$ that tells us which parameters of a function can be borrowed when the function is applied. Henceforth, we will use $[x] \leq_+ [y]$ to denote that $[x]$ is a prefix of $[y]$.

\newcommand{\ExternUniqueFieldResult}{\mathrm{ExternUniqueFieldResult}}
\newcommand{\ExternUniqueField}{\mathrm{ExternUniqueField}}
\newcommand{\ExternUniqueFields}{\mathrm{ExternUniqueFields}}

\subsection{Syntax}

\begin{alignat*}{3}
	r_{*_e} &\in \ExternUniqueFieldResult\ &\Coloneqq& *_r \sep !_r \sep ?_r\ x^\tau\ [ij] \\
	f_{*_e} &\in \ExternUniqueField\ &\Coloneqq&\ [ij] (x^\tau)? \\
	\gamma_{*_e} &\in \ExternUniqueFields\ &=&\ \ADTConst \rightharpoonup 2^{\ExternUniqueField}
\end{alignat*}

For external types $A \notin \dom(\gamma)$, we cannot compute which fields are unique if their outer value is unique. Obtaining this information even for external types is useful because it allows us to specify which escapees are the relevant ones for a given external type and automatically check the adherence to this specification for the escapees provided for external functions, as well as handle fields for which uniqueness depends on a type parameter.

Elements of ExternUniqueFieldResult denote the result for queries that ask whether a specific field is unique: It can be unique ($*_r$), shared ($!_r$) or its uniqueness can depend on a type parameter $x^\tau$ with auxiliary information about the field $[ij]$ within the type parameter for which we would like to determine whether it is unique ($?_r\ x^\tau\ [ij]$). It will be used in the function eu defined below. ExternUniqueField is used to specify that a specific field $[ij]$ is unique, with the caveat that its uniqueness may depend on a type parameter $x^\tau$. Finally, a global and partial function $\gamma_{*_e}$ specifies the full and nonempty tree of unique fields for a given ADTConst by its leafs, i.e. there are no $[ij] (x^\tau)?,\ [kl] (y^\tau)? \in \gamma_{*_e}(A)$ s.t. $[ij] \neq [kl]$ but $[ij] \leq_+ [kl]$ or $[kl] \leq_+ [ij]$. The tree must include the unique fields of all dependencies. For external types $A \notin \dom(\gamma_{*_e})$, we assume that all fields are unique.

\subsection{Unique Fields}

\newcommand{\eu}{\mathrm{eu}}
\newcommand{\paath}{\mathrm{path}}

\begin{alignat*}{3}
	&\eu &&: \mathrlap{\ADTConst \times [\Ctor \times \Proj] \rightharpoonup \ExternUniqueFieldResult} \\
	&\eu&&(A, \paath) &&= \\
	&\rebreak{\begin{cases}
		?_r\ x^\tau\ [kl]	& A \in \dom(\gamma_{*_e}) \land \exists [ij] (x^\tau) \in \gamma_{*_e}(A).\ \exists [kl].\ [ij] \concat [kl] = \paath \\
		*_r	& A \in \dom(\gamma_{*_e}) \land \exists [ij] (x^\tau) \in \gamma_{*_e}(A).\ \paath \leq_+ [ij] \land \paath \neq [ij] \\
		*_r	& A \in \dom(\gamma_{*_e}) \land \exists [ij] \in \gamma_{*_e}(A).\ \paath \leq_+ [ij] \lor [ij] \leq_+ \paath \\
		!_r	& A \in \dom(\gamma_{*_e}) \land \lnot \exists [ij] (x^\tau)? \in \gamma_{*_e}(A).\ \paath \leq_+ [ij] \lor [ij] \leq_+ \paath \\
		*_r & A \notin \dom(\gamma_{*_e})
	\end{cases}}
\end{alignat*}

eu (``external unique'') computes the ExternUniqueFieldResult for a given external type and field. If the field is somewhere on the interior of the tree induced by $\gamma_{*_e}$, we assume that the field is unique. Otherwise, if the field is a leaf or points to a field within a leaf, there are two cases: Either the uniqueness of the field depends on a type parameter, in which case we yield $?_r\ x^\tau\ [kl]$ with $[kl]$ being the remaining path within the leaf, or it does not, in which case we return that the field is unique. Only if the field is not within the tree or within one of the leafs do we return that the field is shared.

\newcommand{\isUnique}{\mathrm{isUnique}}
\newcommand{\rest}{\mathrm{rest}}

\begin{alignat*}{3}
	&\isUnique &&: \mathrlap{\AttrType \times [\Ctor \times \Proj] \rightharpoonup \mathbb{B}} \\
	&\isUnique&&(*\ \blacksquare, \paath) &&= \top \\
	&\isUnique&&(*\ A\ [\tau_\arrg], []) &&= \top \\
	&\isUnique&&(*\ A\ [\tau_\arrg], \paath@((i, j)::\rest)) &&= \\
	&\rebreak{\begin{cases}
		\isUnique(\gamma(A)\{A, [\tau_\arrg]\}_{ij}, \rest)& A \in \dom(\gamma) \\
		\top & A \notin \dom(\gamma) \land \eu(A, \paath) = *_r \\
		\bot & A \notin \dom(\gamma) \land \eu(A, \paath) =\ !_r \\
		\isUnique([\tau_\arrg]_{x^\tau}, [kl]) & A \notin \dom(\gamma) \land eu(A, \paath) =\ ?_r\ x^\tau\ [kl]
	\end{cases}} \\
	&\isUnique&&(!\ \blacksquare, \paath) &&= \bot \\
	&\isUnique&&(!\ A\ [\tau_\arrg], \paath) &&= \bot \\
	&\isUnique&&(!\ [\tau_\param] \to \tau_\ret, \paath) &&= \bot
\end{alignat*}

With isUnique, we compute whether a given field is unique in a given type. If the path points into an erased type, we assume that it is always unique, as we do not know anything about the type in question. Otherwise, if an attribute is shared, then every field within the type in question must also be shared. Finally, for ADTDecls $A$ there are several cases: If $A$ is an ADT, we can proceed with the field denoted in the path after substitution eliminates variables in the ADT. If it is an external type, we use the information by eu and proceed with $[kl]$ in $[\tau_\arrg]_{x^\tau}$ when the result is $?_r\ x^\tau\ [kl]$, bouncing back and forth between external types and type parameters.

\begin{alignat*}{3}
	\gamma_* &: \mathrlap{\ADTConst \times [\AttrType] \to 2^{[\Ctor \times \Proj]}} \\
	\gamma_*&(A, [\tau_\arrg]) &&= \left\{ p \ |\ \isUnique(*\ A\ [\tau_\arrg], p) = \top \land p \neq [] \right\}
\end{alignat*}

In $\gamma_*$, we accumulate all the unique fields of a given type with a given vector of type arguments.

\subsection{Borrowed Parameters}

\newcommand{\isBorrowed}{\mathrm{isBorrowed}}

\begin{alignat*}{3}
	\isBorrowed &: \mathrlap{\mathbb{N} \times \AttrType \times 2^{\Escapee} \to \mathbb{B}} \\
	\isBorrowed&(x, !\ A\ [\tau_\arrg], Q) &&= x_{[]}@[t]? \notin Q \land \forall x_{[ij]}@[t]? \in Q.\ [ij] \notin \gamma_*(A, \tau_\arrg) \\
	\isBorrowed&(x, \tau_\param, Q) &&= x_{[]}@[t]? \notin Q \land \attr(\tau_\param) =\ ! \qquad\text{otherwise}
\end{alignat*}

isBorrowed combines our escape analysis and the information we have gathered about unique fields in order to check whether a function parameter with a specific type and a specific set of escapees for the function can be borrowed. First, the type must always be shared so that the function cannot make use of uniqueness. Secondly, the parameter itself cannot escape. And lastly, if the type is an ADT or an external type, none of the fields of the escapees for the parameter may be unique.

\begin{alignat*}{3}
	&\delta_\mathbb{B} &&: \mathrlap{\Const \to 2^\mathbb{N}} \\
	&\delta_\mathbb{B}&&(c) &&= \begin{cases}
		\left\{x \ |\ \isBorrowed(x, [\tau_\param]_x, \delta_Q(c)_x) \right\} & c \in \dom(\delta_Q) \\
		\emptyset & c \notin \dom(\delta_Q)
	\end{cases}\\
	&\rebreak{\text{where}\ \delta_\tau(c) = ([\tau_\param], \tau_\ret)} \\
	&\rebreak{\text{and}\ \delta_Q(c)_x \coloneqq \left\{y_{[ij]}@[t]? \ |\ y_{[ij]}@[t]? \in \delta_Q(c) \land y = x \right\}}
\end{alignat*}

With $\delta_\mathbb{B}$, we obtain a total function that tells us which parameters can be borrowed for each function in the program. If we have no escape information for a function, then no parameter can be borrowed.

\section{Type Checking}\label{sec:checking}

In this section we will finally define the rules of our type theory.

\subsection{Syntax}

\newcommand{\ZeroedFields}{\mathrm{ZeroedFields}}
\newcommand{\Context}{\mathrm{Context}}

\begin{alignat*}{3}
	Z &\in \ZeroedFields\ &=&\ \Var \times \Ctor \times \Proj \to \mathbb{B} \\
	\Gamma &\in \Context  &\Coloneqq&\ [] \sep \Gamma, x : \tau
\end{alignat*}

ZeroedFields will be used for the mechanism described in the discussion of the \icode{case'} instruction in \cref{sec:ir}. As we want to enable the use of multiple projections on the same variable within a function, we must track which fields have already been projected, disallow repeated projections of the same field and disallow using the variable in any way other than projecting from it. $Z \in \ZeroedFields$ will be used to track this information. For convenience we also declare the following functions that check whether a variable has no zeroed fields and zero a field if the corresponding inner attribute is unique. 

\newcommand{\nz}{\mathrm{nz}}

\begin{alignat*}{3}
	\nz &: \mathrlap{\ZeroedFields \times \Var \to \mathbb{B}} \\
	\nz&(Z, x) &&= \lnot \exists i\ j.\ Z(x, i, j) = \top
\end{alignat*}

\newcommand{\zeroo}{\mathrm{zero}}

\begin{alignat*}{3}
	\zeroo &: \mathrlap{\ZeroedFields \times \Attr \times Var \times Ctor \times Proj \to \ZeroedFields} \\
	\zeroo&(Z, m, x, i, j) &&= \begin{cases}
		Z[(x, i, j) \mapsto \top] & m = * \\
		Z & m =\ !
	\end{cases}
\end{alignat*}

We assume $\Gamma$ to be a multiset, i.e. we track duplicate judgements, but not the order of the context. Note that the latter would be required in dependent type theory, as the order of type dependencies must be retained.

\subsection{Constructor Type Parameter Inference}

As discussed briefly in the description of the \icode{ctor} instruction in \cref{sec:ir}, we must infer the attributes in a \icode{(A [τ?]).ctorᵢ [y]} call since Lean cannot provide them. We will do so in two steps: First, we assign types to type variables based on the types of the arguments provided in \icode{[y]}. Then, we use the explicit type arguments provided by the user in \icode{[τ?]} to fill the remaining variables that could not be inferred and choose their attributes as strongly as possible, since type arguments that cannot be inferred from the types of \icode{[y]} are not subject to any uniqueness constraints induced by the vector of constructor arguments. Together, this ensures that no attributes for type arguments must be provided explicitly by the user, as they can either be inferred or chosen as strongly as possible.

\begin{alignat*}{3}
	f &\sqcup g : (M \rightharpoonup N) \times (M \rightharpoonup N) \rightharpoonup (\dom(f) \cup \dom(g) \rightharpoonup N) \\
	(f &\sqcup g)(x) = \begin{cases}
		f(x) & x \in \dom(f) \land x \notin \dom(g) \\
		g(x) & x \notin \dom(f) \land x \in \dom(g) \\
		f(x) & x \in \dom(f) \cap \dom(g) \land f(x) = g(x)
	\end{cases}
\end{alignat*}

Note that $(f, g) \notin \dom(\cdot \sqcup \cdot)$ if $\exists x \in \dom(f) \cap \dom(g).\ f(x) \neq g(x)$.

\newcommand{\inferVarsDash}{\mathrm{inferVars'}}

\begin{alignat*}{3}
	&\inferVarsDash &&: \mathrlap{\AttrType \times \AttrType \rightharpoonup (\Var \rightharpoonup \AttrType)} \\
	&\inferVarsDash&&(m\ x^\kappa, \tau) &&= \emptyset \\
	&\inferVarsDash&&(x^\tau, \tau) &&= \{ x^\tau \mapsto \tau \} \\
	&\inferVarsDash&&(m\ \blacksquare, m\ \blacksquare) &&= \emptyset \\
	&\inferVarsDash&&(m\ A\ [\tau_{\arrg_1}], m\ A\ [\tau_{\arrg_2}]) &&= \bigsqcup_i \inferVarsDash([\tau_{\arrg_1}]_i, [\tau_{\arrg_2}]_i) \\
	&\inferVarsDash&&(!\ [\tau_{\param_1}] \to \tau_{\ret_1}, !\ [\tau_{\param_2}] \to \tau_{\ret_2}) &&= \\
	&\rebreak{\bigsqcup_i \inferVarsDash([\tau_{\param_1}]_i, [\tau_{\param_2}]_i) \sqcup \inferVarsDash(\tau_{\ret_1}, \tau_{\ret_2})}
\end{alignat*}

\newcommand{\inferVars}{\mathrm{inferVars}}

\begin{alignat*}{3}
	\inferVars &: \mathrlap{[\AttrType] \times [\AttrType] \rightharpoonup (\Var \rightharpoonup \AttrType)} \\
	\inferVars&([], []) &&= \emptyset \\
	\inferVars&(\tau_1 :: \rest_1, \tau_2 :: \rest_2) &&= \inferVarsDash(\tau_1, \tau_2) \sqcup \inferVars(\rest_1, \rest_2) \\
\end{alignat*}

inferVars' takes an expected type with type variables and a provided type and computes an assignment of types to variables s.t. the first type becomes the second after substitution. In doing so, it ignores self-variables $m\ x^\kappa$, because the type without type variables has not had its variables substituted yet, while the provided type has had its variables substituted. As a result, inferVars' yielding an assignment by itself cannot guarantee that the first type is equal to the second one after substitution, and we must perform another check after substituting all the variables to obtain this guarantee. Note also that if there is a conflicting assignment for any variable in $\inferVarsDash(\tau_1, \tau_2)$ or $\inferVars([\tau_1], [\tau_2])$, then $\cdot \sqcup \cdot$ propagates its partiality to inferVars' and subequently inferVars, i.e. $(\tau_1, \tau_2) \notin \dom(\inferVarsDash)$ and $([\tau_1], [\tau_2]) \notin \dom(\inferVars)$.

\newcommand{\pickTypes}{\mathrm{pickTypes}}

\begin{alignat*}{3}
	\pickTypes &: \mathrlap{[\AttrType?] \times [\AttrType?] \rightharpoonup [\AttrType]} \\
	\pickTypes&([]) &&= [] \\
	\pickTypes&(\tau_e? :: \rest_e, \tau_i :: \rest_i) &&= \tau_i :: \pickTypes(\rest_e, \rest_i) \\
	\pickTypes&(\tau_e :: \rest_e, - :: \rest_i) &&= \strengthen(\tau_e) :: \pickTypes(\rest_e, \rest_i) \\
\end{alignat*}

Here, we use $-$ to denote that an $\AttrType?$ is not present.
With pickTypes, we implement the mechanism that inferred types are preferred if they exist, and otherwise an explicit type is used and strengthened. Note that if $\exists i.\ [\tau_1]_i = [\tau_2]_i = -$, then $([\tau_1], [\tau_2]) \notin \dom(\pickTypes)$.

\newcommand{\inferTypeArgs}{\mathrm{inferTypeArgs}}

\begin{alignat*}{3}
	&\inferTypeArgs &&: \mathrlap{\ADT \times Ctor \times [\AttrType] \times [\AttrType?] \rightharpoonup [\AttrType]} \\
	&\inferTypeArgs&&(a, i, [\tau_\arrg], [\tau_e?]) &&= \pickTypes([\tau_e?], [\mathrm{inferred}(y^\tau)])\\
		&\rebreak{\text{where } a_i = [\tau_\field(x^\kappa_\adt, [y^\tau])] \to *x^\kappa_\adt} \\
		&\rebreak{\text{and } \mathrm{inferred} = \inferVars([\tau_\field(x^\kappa_\adt, [y^\tau])], [\tau_\arrg]])}
\end{alignat*}

Lastly, using inferTypeArgs, we infer type arguments for a constructor $i$ in an ADT $a$ with provided arguments types $[\tau_\arrg]$ and user-provided explicit argument types $[\tau_e?]$. Once again, both inferVars and pickTypes propagate their partiality to inferTypeAgs.

\subsection{Type Theory}

In the following, we will progressively introduce the rules of our type theory and explain them along the way. Whenever a variable \icode{x} is used in any meaningful way other than projection, we demand $\nz(Z, \icode{x})$ so that it cannot be used if any field has been projected in the past.

\begin{mathpar}
	\boxed{\vdash \delta_\tau} \hspace{1.5em}
	$\inferrule[Program]{\forall c \in \dom(\delta_\tau) \cap \dom(\delta) \text{ s.t. } \delta(c) = \lambda\ \icode{[y]}.\ \icode{F} \land \delta_\tau(\tau) = ([\tau_\param], \tau_\ret).\\ 
		\emptyset; [\icode{y} : \tau_\param] \vdash \icode{F} : \tau_\ret}
	{\vdash \delta_\tau}$
\end{mathpar}

The \textsc{Program} rule states that in order to check a program $\delta_\tau$, we check that each function $c \in \dom(\delta)$ adheres to its function type with no zeroed fields in any variable at the start.

\newcommand{\tret}{\tau_\text{ret'}}

\begin{mathpar}
	\boxed{Z; \Gamma \vdash \texttt{F} : \tau}
\end{mathpar}
\begin{mathpar}
	$\inferrule[Duplicate]{Z; \Gamma, \icode{x} :\ !\tau, \icode{x} :\ !\tau \vdash \icode{F} : \tret}{Z; \Gamma, \icode{x} :\ !\tau \vdash \icode{F} : \tret}$ \hspace{1.5em}
	$\inferrule[Forget]{Z; \Gamma \vdash \icode{F} : \tret}{Z; \Gamma, \icode{x} :\ !\tau \vdash \icode{F} : \tret}$
\end{mathpar}
\begin{mathpar}
	$\inferrule[Downcast]{\tau \succcurlyeq \tau' \\ \nz(Z, \icode{x}) \\ Z; \Gamma, \icode{x} : \tau' \vdash \icode{F} : \tret}{Z; \Gamma, \icode{x} : \tau \vdash \icode{F} : \tret}$
\end{mathpar}
\textsc{Duplicate} and \textsc{Weaken} allow manipulating variables of shared type in the context as if the context was a set and structural. Note that $!\tau$ is an application of the $!(\cdot)$ function defined in \cref{sec:types} and that unique variables cannot be manipulated in this manner; their exact amount in the context needs to be tracked. Meanwhile, the \textsc{Downcast} rule allows applying the subtyping relation also defined in \cref{sec:types}.

\begin{mathpar}
	$\inferrule[$\blacksquare$-Cast]{Z; \Gamma, \icode{x} : \weakenInner(\tau) \vdash \icode{F} : \tret}{Z; \Gamma, \icode{x} : \attr(\tau)\ \blacksquare \vdash \icode{F} : \tret}$ \hspace{1.5em}
	$\inferrule[$\blacksquare$-Erase]{Z; \Gamma, \icode{x} : \attr(\tau)\ \blacksquare \vdash \icode{F} : \tret}{Z; \Gamma, \icode{x} : \tau \vdash \icode{F} : \tret}$
\end{mathpar}
\textsc{$\blacksquare$-Cast} and \textsc{$\blacksquare$-Erase} enable us to work with erased types: We can cast to and from any type while retaining the outer attribute, but have to make all the inner attributes shared in the process.

\begin{mathpar}
	$\inferrule[Ret]{\nz(Z, \icode{x})}{Z; \Gamma, \icode{x} : \tret \vdash \icode{ret x} : \tret}$ \hspace{1.5em}
	$\inferrule[Case]{\nz(Z, \icode{x}) \\ [Z; \Gamma, \icode{x} : m\ A\ [\tau_\arrg] \vdash \icode{F} : \tret]}{Z; \Gamma, \icode{x} : m\ A\ [\tau_\arrg] \vdash \icode{case x of [F]} : \tret}$
\end{mathpar}
The \textsc{Ret} and \textsc{Case} rules are straight-forward: For \textsc{Ret}, we need a matching variable in our context, and for \textsc{Case}, we check every branch. It is worth pointing out that in \textsc{Ret}, there is no issue with throwing away the rest of the context $\Gamma$, as all variables can always be made shared and then discarded using \textsc{Weaken}, and that in \textsc{Case}, \icode{x} does not need to be consumed as \icode{case} is read-only.

\begin{mathpar}
	$\inferrule[Case'-!]
		{\nz(Z, \icode{x}) 
				\\ A \in \dom(\gamma)
				\\ \gamma(A)\{A, [\tau_\arrg]\} = \mu\ x^\kappa_\adt.\ [[\tau_\field] \to *x^\kappa_\adt]
				\\ [Z; \Gamma, [\icode{y} :\ !\tau_\field] \vdash \icode{F} : \tret]}
		{Z; \Gamma, \icode{x} : \ !\ A\ [\tau_\arrg] \vdash \icode{case' x of [ctorᵢ [y] ⇒ F]} :  \tret}$
\end{mathpar}
\begin{mathpar}
	$\inferrule[Case'-*]
	{\nz(Z, \icode{x}) 
		\\ A \in \dom(\gamma)
		\\ \gamma(A)\{A, [\tau_\arrg]\} = \mu\ x^\kappa_\adt.\ [[\tau_\field] \to *x^\kappa_\adt]
		\\ [Z; \Gamma, [\icode{y} : \tau_\field] \vdash \icode{F} : \tret]}
	{Z; \Gamma, \icode{x} : *\ A\ [\tau_\arrg] \vdash \icode{case' x of [ctorᵢ [y] ⇒ F]} :  \tret}$
\end{mathpar}
The \textsc{Case'}-rules work similar to \textsc{case}, except that the variable that is being matched on is consumed and that we need to add the variables associated with the constructor in a specific branch to the context. If the value we are matching on is shared, then the newly created variables must be shared as well. This is the essence of $*$ in ADTs meaning ``unique if the outer value is unique''.

\begin{mathpar}
	$\inferrule[Let-App]{[\nz(Z, \icode{y})] 
		\\ \delta_\tau(\icode{c}) = ([\tau_\param], \tau_\ret)
		\\ Z; \Gamma, \{ [\icode{y} : \tau_\param]_x \ |\ x \in \delta_\mathbb{B}(\icode{c}) \}, \icode{z} : \tau_\ret \vdash \icode{F} : \tret
	}
	{Z; \Gamma, [\icode{y} : \tau_\param] \vdash \icode{let z := c [y]; F} : \tret}$
\end{mathpar}
In \textsc{Let-App}, our argument types need to match the parameter types and we obtain the result of the function call in our new context. Non-borrowed arguments are consumed, borrowed ones are retained.

\begin{mathpar}
	$\inferrule[Let-Pap-Full]{[\nz(Z, \icode{y})] 
		\\ \delta_\tau(\icode{c}) = ([\tau_\param], \tau_\ret)
		\\ |\icode{[y]}| = |[\tau_\param]|
		\\ Z; \Gamma, \icode{}z :\ !\tau_\ret \vdash \icode{F} : \tret
	}
	{Z; \Gamma, [\icode{y} :\ !\tau_\param] \vdash \icode{let z := pap c [y]; F} : \tret}$
\end{mathpar}
\begin{mathpar}
	$\inferrule[Let-Pap-Part]{[\nz(Z, \icode{y})] 
		\\ \delta_\tau(\icode{c}) = ([\tau_\param], \tau_\ret)
		\\ |\icode{[y]}| = |[\tau_{\param_1}]| < |[\tau_\param]|
		\\ [\tau_{\param_1}] \concat [\tau_{\param_2}] = [\tau_\param]
		\\ Z; \Gamma, \icode{z} :\ !\ [\tau_{\param_2}] \to \tau_\ret \vdash \icode{F} : \tret
	}
	{Z; \Gamma, [\icode{y} :\ !\tau_{\param_1}] \vdash \icode{let z := pap c [y]; F} : \tret}$
\end{mathpar}
For \icode{pap}, there are two separate rules, one for a \icode{pap} call that is effectively a full application, and one for a proper partial application. As our higher-order functions are always shared, our context needs to contain matching shared arguments that are consumed in the process. Depending on the rule, we either obtain a new higher-order function or the shared return value of the function. Note that to implement these rules within Lean, we need to ensure that the result of \icode{pap c [y]} does not rely on the uniqueness of its arguments and must generate a variant of \icode{c} that does not rely on uniqueness instead.

\begin{mathpar}
	$\inferrule[Let-VarApp-Full]{\nz(Z, \icode{y}) 
		\\ Z; \Gamma, \icode{z} :\ !\tau_\ret \vdash \icode{F} : \tret
	}
	{Z; \Gamma, \icode{x} :\ !\ \tau_\param \to \tau_\ret, \icode{y} :\ ! \tau_\param \vdash \icode{let z := x y; F} : \tret}$
\end{mathpar}
\begin{mathpar}
	$\inferrule[Let-VarApp-Part]{\nz(Z, \icode{y}) 
		\\ |[\tau_{\text{param'}}]| \geq 1
		\\ Z; \Gamma, \icode{z} :\ !\ [\tau_{\text{param'}}] \to \tau_\ret \vdash \icode{F} : \tret
	}
	{Z; \Gamma, \icode{x} :\ !\ (\tau_\param :: [\tau_{\text{param'}}]) \to \tau_\ret, \icode{y} :\ ! \tau_\param \vdash \icode{let z := x y; F} : \tret}$
\end{mathpar}
For \textsc{Let-VarApp}, there is a similar split as for \textsc{Let-Pap}: Depending on whether we have applied all the arguments of a higher-order function, we either get a new higher-order function with the argument applied or the shared return value of the function. 

\begin{mathpar}
	$\inferrule[Let-Ctor]{[\nz(Z, \icode{y})]
		\\ (\gamma(A), \icode{i}, [\tau], \icode{[$\tau_\arrg$?]}) \in \dom(\inferTypeArgs)
		\\ [\tau_\arrg'] = \inferTypeArgs(\gamma(A), \icode{i}, [\tau], \icode{[$\tau_\arrg$?]})
		\\ \gamma(A)\{A, [\tau_\arrg']\}_\icode{i} = [\tau_\field] \to *x^\kappa_\adt
		\\ [\tau_\field] = [\tau]
		\\ Z; \Gamma, \icode{z} : *\ A\ [\tau_\arrg'] \vdash \icode{F} : \tret
	}
	{Z; \Gamma, [\icode{y} : \tau] \vdash \icode{let x = (A [$\tau_\arrg$?]).ctorᵢ [y]; F} : \tret}$
\end{mathpar}
\textsc{Let-Ctor} is the most complex rule because it needs to infer the attributes of its type arguments using inferTypeArgs. After inferring the type arguments, we substitute them in $\gamma(A)$ and check whether the resulting types for the fields match those of our initial arguments that we used for inference. This is because while $(\gamma(A), i, [\tau], [\tau_\arrg?]) \in \dom(\inferTypeArgs)$ is always true if $\tau$ is applicable to $\tau_\field$, it may still be true if $\tau$ is not applicable to $\tau_\field$, as inferTypeArgs is ignoring the self-variable $x_\adt^\kappa$ in $\gamma(A)$. In applying the rule, we consume the arguments and gain a new unique instance of the ADT type in return.

\begin{mathpar}
	$\inferrule[Let-Proj-*]{Z(\icode{y}, \icode{i}, \icode{j}) = \bot
		\\ \tau_\field = \gamma(A)\{A, [\tau_{\arrg}]\}_{\icode{ij}}
		\\ \zeroo(Z, \attr(\tau_\field), \icode{y}, \icode{i}, \icode{j}); \Gamma, \icode{y} : *\ A\ [\tau_{\arrg}], \icode{z} : \tau_\field \vdash \icode{F} : \tret
	}
	{Z; \Gamma, \icode{y} : *\ A\ [\tau_{\arrg}] \vdash \icode{let z = projᵢⱼ y; F} : \tret}$
\end{mathpar}
\begin{mathpar}
	$\inferrule[Let-Proj-!]{Z(\icode{y}, \icode{i}, \icode{j}) = \bot
		\\ Z; \Gamma, \icode{y} :\ !\ A\ [\tau_{\arrg}], \icode{z} :\ !\gamma(A)\{A, [\tau_{\arrg}]\}_{\icode{ij}} \vdash \icode{F} : \tret
	}
	{Z; \Gamma, \icode{y} :\ !\ A\ [\tau_{\arrg}] \vdash \icode{let z = projᵢⱼ y; F} : \tret}$
\end{mathpar}
Finally, our \textsc{Let-Proj} rules are only applicable if the specific field has not been projected from yet, but retain the variable that we project from in return. We also obtain the field in our new context. If the ADT is shared, then we need to make the new field shared as well, but do not have to zero the projected field, whereas otherwise, we obtain the field with its actual attribute, but have to zero it.
\chapter{Implementation}\label{sec:implementation}
\chapter{Future Work}\label{sec:futurework}
In this chapter, we will quickly cover work that is still left to be done in order to integrate our type theory with Lean 4, as well as work that would greatly improve the user-experience of working with the type theory.

\section{Type Inference}\label{sec:inference}
We have not covered the topic of type inference at all so far, but some amount of type inference is necessary to integrate our type theory with Lean 4. During the compilation process, the Lean 4 compiler toolchain may create additional auxiliary functions that users cannot provide type annotations for: 
\begin{itemize}
	\item In monadic control flow, Lean may create join points and other auxiliary functions to represent complex control flow (\lstinline|break|, \lstinline|continue|, early \lstinline|return|, \lstinline|if| blocks).
	\item As argued in \cref{sec:ir}, functions that act as join points may get replaced by dedicated join point instructions that callers can directly jump to.
	\item If a function does not use all of its parameters, then an auxiliary function using only live parameters is created. Note that this may occur often because of LCNF's monomorphization step, where type parameters and other instances of dependent types are erased.
\end{itemize}
Fortunately, in all of these cases, we have access to a lot of surrounding context to determine what the respective uniqueness attributes for an auxiliary function should be.

Nonetheless, it would be nice to have some amount of type inference for the uniqueness attributes of user-written functions as well. Generally, we want parameters that can be borrowed in the sense of \cref{sec:borrowing} to be shared. On the other hand, for parameters that escape, there is no principal type: Depending on whether callers have a unique or shared object at hand, they would want the parameter type to match the given attribute that they have available.

Hence, without polymorphism, one possible approach would be to infer the strongest possible type for a given function and then automatically create auxiliary functions that represent possible weakenings of this strongest possible type, the most obvious weakening being a function where all parameters are shared and the return type is shared as well.

With polymorphism, principal types may again become available and could directly be inferred if the inference algorithm is sufficiently sophisticated.

\section{Integration With Lean 4}
In \cref{sec:implementation}, we implement a type checker for our type theory and a model implementation of the IR described in \cref{sec:ir}, but do not integrate it with Lean 4 itself. In the future, the following steps need to be taken to finish the integration:
\begin{enumerate}
	\item A form of type inference that can deal with the kinds of auxiliary functions described in \cref{sec:inference} needs to be implemented.
	\item A notation to provide uniqueness annotations in types needs to be implemented. Types annotated with $*$ are unique, types without an annotation are shared. The notation should set the \lstinline|mdata| field of Lean 4's \lstinline|Expr|, which can be used to store auxiliary data.
	\item The \lstinline|mdata| field needs to be handled correctly in the compiler toolchain so that it is preserved until the type checker runs at the end of the pipeline. As of the writing of this thesis, the function that converts Lean types to LCNF types erases expressions annotated with \lstinline|mdata|.
	\item The functions $\delta_{q_e}$ from \cref{sec:escapeanalysis} and $\gamma_{*_e}$ from \cref{sec:borrowing} need to be provided using a custom annotation in Lean 4's annotation mechanism, where declarations, including stubs for external functions and types, can be provided with auxiliary data by users.
	\item Lean's LCNF must be translated to our model IR and our model type system while preserving information about the mapping.
	\item Our type checker must be adjusted to produce reasonable error codes. As of now, it only yields a boolean indicating whether a given environment type-checks. Then, if there is a uniqueness type error, using the aforementioned mapping, the error for our model IR must be translated to a corresponding LCNF error.
	\item In addition to checking whether a given environment type-checks, our type checker must accumulate information about the scope in which a variable is unique. After translating this information back to LCNF, it can be used in subsequent optimizations, for example eliminating the reference count check described in \cref{sec:beans}.
\end{enumerate}

\section{Higher-Order Functions}
We have discussed the topic of higher-order functions in uniqueness type systems at length in \cref{sec:uniqueness} and \cref{sec:hof}, but chose to disregard uniqueness of functions for now in \cref{sec:types} by assuming every higher-order function as shared. 

This situation is far from optimal because Lean code uses higher-order functions for type classes, monadic code, as well as tricks that guarantee uniqueness for unique values within other unique values, like the \lstinline|update| trick discussed in \cref{sec:beans}. 

If we could use uniqueness annotations in type classes, defining the following classes that provide extra control over uniqueness would be possible:

\begin{code}
class Copyable (α) where
  copy : *α → *(*α × *α)
  
class LawfulCopyable (α) extends Copyable α where
  law : copy = (fun x => (x, x))
\end{code}

As discussed in \cref{sec:hof}, we believe that the approach that deleverages the uniqueness of objects in a function closure is the most viable, though it requires a change to the runtime in that higher-order functions need to store two function pointers as opposed to just one. Additionally, several other components in \cref{sec:theory} must be adjusted as well, as subtyping must now account for co- and contravariant type parameters and propagation must be capable of propagating through higher-order functions. Similarly, borrowing most be adjusted with unique functions in mind.

\section{Polymorphism}
Another topic we have not touched on at all is attribute polymorphism. Part of the reason for this is that polymorphism results in somewhat unintuitive behaviour if the coercion between unique and shared values is implicit, as in our system in \cref{sec:theory}, since polymorphic functions may both silently downcast a shared value and then propagate the sharedness through the rest of the code. In this case, using different function names would unveil the mistake early on.

However, if the coercion is made explicit, then mistakes would always be spotted early on, as a unique value can never be passed to a shared parameter unless users acknowledge it with an explicit instruction. Then, shared and unique functions would not need to be named differently, and polymorphism would be a useful thing to have.

What follows are some brief and incomplete thoughts on what would be required to make polymorphism work:
\begin{itemize}
	\item In order to be substitutable for both a unique and a shared attribute, variables that are polymorphic in their attribute can only be passed to polymorphic parameters, but must be used uniquely.
	\item Variables polymorphic in their attribute can be updated destructively.
	\item Borrowed parameters should always be shared, not polymorphic, since both linear and unique values can be passed to the parameter, but it is still allowed to share the parameter within the function body, as long as it does not escape.
	\item There needs to be a mechanism to connect the uniqueness of two attribute variables and state ``this component of the return type can be unique if this parameter is unique'', similar to Clean's $\leq$ operator or the boolean connectives discussed by \cite{vries_making_2009}.
	\item Uniqueness attributes in ADTs should propagate the attribute variable of the outer value when the respective field is accessed.
	\item Ideally, the fact that shared types cannot contain unique ones should not have to be explicitly reflected everywhere in the type annotation of a function; it should be assumed implicitly, or at least the annotational clutter that is present in de Vries' implementations of polymorphism should be reduced somehow.
\end{itemize}

An alternative to polymorphism would be to provide facilities that generate function declarations for all valid attribute annotations after monomorphization. Since there are only two attributes and users likely do not care much about the concrete uniqueness annotations except that they should make their code type check, generating functions for all possible annotations may be a worthwhile compromise, especially as systems that support attribute polymorphism tend to produce complex annotations that are difficult to grasp.

\section{Proof of Soundness}
In all of the previous sections, we have only argued informally about the soundness of our type system, i.e. that variables of unique type are indeed referenced uniquely. A formal proof of soundness of our system is still left to be done.
\chapter{Related Work}\label{sec:relatedwork}
In this chapter, we will briefly cover other approaches that are similar to ours.

\paragraph{Linear Haskell} is a lazy functional language with support for properly linear types, but no borrowing. It supports an \lstinline|MArray| type with efficient destructive updates and a function \lstinline|freeze| that can convert a linear \lstinline|MArray| to a non-linear \lstinline|Array|. As described in \Cref{sec:qtt}, \lstinline|MArray| constructors use continuation passing. Additionally, an \lstinline|MArray| itself is not allowed to contain linear values. Linear Haskell also has experimental support for multiplicity polymorphism and multiplicity inference.

\cite{spiwack_linearly_2022} resolve some of these shortcomings by introducing a language of linear capabilities on top of Linear Haskell. The uniqueness of a type is still a library design decision, but constructors do not need to be stated in continuation-passing style anymore. Arrays are allowed to contain other unique types and there is a borrowing mechanism both for unique types within unique types and unique types that are used in a read-only manner in functions. 

Unfortunately, linear constraints still have to be unpacked explicitly in the term language and the resulting system is quite complex. We do however think that the general idea of decoupling values from their linear capabilities and enabling libraries to provide complex capabilities of their own may prove to be a fruitful avenue of research.

\paragraph{Idris 2 \citep{brady_idris_2021}} is a dependently typed functional language with support for quantitative type theory. As such, its substructural type system has the same properties as that of Linear Haskell, but also supports an erasure quantity. Dependent type theory enables some additional applications, like specifying linear usage protocols and simulating session types \citep{honda_types_1993}.

Idris, the precursor to Idris 2, has support for what the authors call ``uniqueness types'' as well \citep{brady_type-driven_2017}. Idris' ``uniqueness types'' are what we call ``invariably unique types'' in \Cref{sec:ltt}, though Idris' linear types need not be consumed and are hence affine. Compound data types are either declared to be inherently unique or non-unique, and there is no coercion between the two. It also supports a restrictive notion of borrowing that allows pattern matching on borrowed values, but no further inspection. Due to the lack of an erasure quantity, the combination of dependent and linear types is limited.

\paragraph{ATS \citep{shi_linear_2013}} is a dependently typed language with support for invariably unique types to ensure safe resource-usage and an extensive foreign function interface. Pointers to resources consist of two components: the reference to the resource, as well as a linear value that witnesses that the resource has not been freed yet. 

There does not appear to be a safe borrowing mechanism or a safe coercion between unique and shared types, but library functions can assert that a reference is not consumed by a function. There also seems to be support for syntactic sugar that reduces the amount of explicit linear values being passed around by implicitly managing a context of linear values, similar in spirit to the apporach of \cite{spiwack_linearly_2022}.

\paragraph{Clean \citep{smetsers_guaranteeing_1994}} is the functional language that introduced the concept of uniqueness typing. It supports uniqueness types, polymorphism for uniqueness attributes, as well as type inference. \cite{de_vries_making_2009} provides a type theory description for Clean that uses lambda calculus as its term language. 

As previously discussed in \Cref{sec:designspace}, Clean uses invariably unique types for higher-order functions. Its borrowing mechanism follows \cite{wadler_linear_1990} and only allows borrowing in expressions that produce primitive types, so that borrowed variables are guaranteed not to escape. In terms of applications, Clean uses uniqueness for destructive updates, as well as safe I/O.

\paragraph{Cogent \citep{oconnor_cogent_2021}} is a non-recursive functional language designed for systems programming with extensive FFI capabilities, delegating I/O and recursive programs to C. It features a certifying compiler that produces proofs that the generated program is a refinement of the original Cogent program using an Isabelle/HOL \citep{nipkow_isabellehol_2002} embedding. With these certificates, it is possible to prove programs correct within Isabelle/HOL. 

It also features invariably unique types to ensure safe resource usage and enable destructive updates, as well as a borrowing mechanism based on observer types \citep{odersky_observers_1992}, polymorphism and type inference. Since it uses uniqueness to ensure safe resource use, there is no coercion from unique to non-unique types. Higher-order functions are always fully applied and cannot capture variables in their closure.

\paragraph{Granule \citep{orchard_quantitative_2019}} is a functional programming language that acts as a framework for linear types, indexed types and graded modal types. \cite{sergey_linearity_2022} add uniqueness types to Granule as a separate modality and use them for efficient destructive updates. Their type system does not support nested uniqueness types, borrowing, type inference or uniqueness polymorphism. Higher-order functions are implemented using properly linear types. 

\paragraph{Futhark \citep{henriksen_futhark_2017}} is a functional data-parallel language intended for GPU code. It uses an alias analysis in order to support efficient in-place updates on arrays only: If an array has been aliased in the past, then at least all of its aliases may not be used after the in-place update. This allows for some greater flexibility, where arrays can become temporarily aliased. Across function boundaries, invariably unique types are used to ensure the uniqueness of arrays, not the full aliasing information. 

There are dedicated type system rules for common array operations and variable uses are classified as ``consuming'' or ``observing''. This allows Futhark to implement a form of borrowing where observing and consuming operations are used in an alternating manner, such that an array can only be observed arbitrarily often before it is consumed, after which it cannot be observed or consumed anymore.

There is no notion of uniqueness polymorphism or type inference. There is also no support for letting users implement custom consumers of higher-order functions. The built-in higher-order operators always fully apply the higher-order functions that they are passed and arrays in the closure of the higher-order function are not allowed to be consumed within the function.

\paragraph{Affe \citep{radanne_kindly_2020}} is an impure functional language with support for properly linear and affine types, borrowing, polymorphism and type inference. Higher-order functions that contain linear values in their closure are again linear.

It supports the notion of an ``explicit borrow'' where the borrowee is allowed to update the borrowed value, but is still required to use it linearly. This is necessary because Affe is impure; in a purely functional language, we would instead just return the updated value regardless of whether we are using a substructural type system or not.

For a detailed comparison of Affe with other ML-like languages that support linear types, such as System F° \citep{mazurak_lightweight_2010}, Alms \citep{tov_practical_2011}, Quill \citep{morris_best_2016} and Mezzo \citep{balabonski_design_2016}, as well as the imperative programming languages Rust \citep{weiss_oxide_2021}, Vault \citep{deline_enforcing_2001} and Plaid \citep{garcia_foundations_2014}, we refer to \citep{radanne_kindly_2020}.

\paragraph{Linear Dafny \citep{li_linear_2022}} is an imperative language with invariably unique types and support for an SMT backend \citep{barrett_satisfiability_2018} that is leveraged to reason about general program properties and aliasing when the linear type system cannot provide the required guarantees. 

It supports a traditional borrowing mechanism in the style of observer types \citep{odersky_observers_1992} where observation propagates outwards, as well as arbitrarily mixing unique and shared types, the latter of which is ensured to be safe by delegating an aliasing proof obligation to the SMT backend.

Since Lean is also a general proof language, we consider the approach of Linear Dafny very relevant to Lean as well. In order to implement something similar for Lean, uniqueness types would have to be made compatible with Lean's dependent type theory and we would have to embed a model of Lean that allows reasoning about the aliasing of Lean values within Lean itself. \cite{niu_cost-aware_2022} may provide some guidance for augmenting dependently typed languages with cost functions.
\chapter{Conclusion}\label{sec:conclusion}

\bibliographystyle{plainnat}
\bibliography{bib}

\begin{otherlanguage}{ngerman}
\chapter*{Erklärung}
\pagestyle{empty}

  \vspace{20mm}
  Hiermit erkläre ich, \theauthor, dass ich die vorliegende Masterarbeit selbst\-ständig
verfasst habe und keine anderen als die angegebenen Quellen und Hilfsmittel
benutzt habe, die wörtlich oder inhaltlich übernommenen Stellen als solche kenntlich gemacht und
die Satzung des KIT zur Sicherung guter wissenschaftlicher Praxis beachtet habe.
  \vspace{20mm}
  \begin{tabbing}
  \rule{7cm}{.4pt}\hspace{1cm} \= \rule{6.8cm}{.4pt} \\
 Ort, Datum \> Unterschrift
  \end{tabbing}
\end{otherlanguage}

\chapter*{Acknowledgements}
\pagestyle{empty}

\end{document}
