\chapter{Conclusion}\label{sec:conclusion}
We have provided an overview for the design space of substructural type theory with the goal of ensuring destructive updates in the Lean 4 theorem prover and evaluated existing approaches, concluding that uniqueness type theory is most suited for ensuring destructive updates. 

Using our evaluation, we have designed a uniqueness type theory of our own and implemented a type checker for it using the Lean 4 programming language at \url{https://github.com/mhuisi/Uniq}. Our type theory targets a combined model of Lean 4's intermediate representations. It supports uniqueness types, algebraic data types, erased types, external declarations, non-shallow subtyping for uniqueness attributes and a notion of borrowing. To implement the latter, we have designed and implemented an escape data flow analysis that computes an over-approximation of both parameters and fields of parameters that escape in a given function.

Our type theory lacks support for uniqueness attributes in higher-order functions, type inference, as well as support for polymorphism. Our escape analysis produces non-satisfactory results on recursive functions over recursive types, inhibiting the borrowing of arguments to such functions. Integrating our type theory with the Lean 4 compiler is future work. We have not formally evaluated the soundness of our type theory.

For higher-order functions, we have evaluated all existing approaches known to us and made a recommendation for an approach that we think is the most suitable one to implement in the future. For borrowing, we have made suggestions to improve the implementation provided in this thesis. For the topics of type inference, polymorphism and Lean 4 integration, we have outlined steps that need to be taken in order to complete the implementation thereof.