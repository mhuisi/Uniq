\chapter{Formal description}\label{sec:theory}

In this chapter, we will provide a formal description of our type theory and all the associated mechanisms required to make it work. \Cref{sec:types} introduces the syntactical material for our types and declares a number of commonly useful utility functions. \Cref{sec:ir} defines the syntax of the IR. In \cref{sec:escapeanalysis}, we specify an abstract interpretation based escape analysis in order to implement the borrowing mechanism described in \cref{sec:borrowing}. Finally, \cref{sec:checking} provides the rules of our type theory.

\newcommand{\sep}{\ \ |\ \ }
\newcommand{\icode}[1]{\textrm{\lstinline[language=ir-if]|#1|}}

\section{Types}\label{sec:types}
In all of the following sections, we use $[x]$ to denote a vector of elements $x$, otherwise commonly written as $\overline{x}$. We will often lift these brackets over an operation; e.g. the functional code $\mathrm{map}(\oplus, \mathrm{zip}([x], [y]))$ is written as $[x \oplus y]$ for vectors $[x]$ and $[y]$. In derivation rules, we also use $[x]$ for $x \in \mathbb{B}$ to mean $\forall x \in [x].\ x$.

\newcommand{\dom}{\mathrm{dom}}
\newcommand{\Var}{\mathrm{Var}}
\newcommand{\Ctor}{\mathrm{Ctor}}
\newcommand{\Proj}{\mathrm{Proj}}
\newcommand{\Const}{\mathrm{Const}}
\newcommand{\Attr}{\mathrm{Attr}}
\newcommand{\ADT}{\mathrm{ADT}}
\newcommand{\adt}{\mathrm{adt}}
\newcommand{\field}{\mathrm{field}}
\newcommand{\ADTConst}{\mathrm{ADTConst}}
\newcommand{\AttrType}{\mathrm{AttrType}}
\newcommand{\arrg}{\mathrm{arg}}
\newcommand{\param}{\mathrm{param}}
\newcommand{\ret}{\mathrm{ret}}
\newcommand{\ADTDecls}{\mathrm{ADTDecls}}
\newcommand{\FunTypes}{\mathrm{FunTypes}}

\begin{alignat*}{2}
  x, y, z &\in \Var \\
  i &\in \Ctor \\
  j &\in \Proj \\
  c &\in \Const \\
  m &\in \Attr &\Coloneqq&\ ! \sep * \\
  a &\in \ADT &\Coloneqq&\ \mu\ x^\kappa_\adt.\ [[\tau_\field(x^\kappa_\adt, [y^\tau])] \to *x^\kappa_\adt] \\
  A &\in \mathrlap{\ADTConst} \\
  \gamma &\in \ADTDecls &=&\ \ADTConst \rightharpoonup \ADT \\
  \tau &\in \AttrType &\Coloneqq&\ m\ x^\kappa \sep x^\tau \sep m\ \blacksquare \sep m\ A\ [\tau_\arrg] \sep !\ [\tau_\param] \to \tau_\ret \\
  \delta_\tau &\in \FunTypes &=&\ \Const \rightharpoonup [\AttrType] \times \AttrType
\end{alignat*}

Ctor and Proj denote the constructors and fields within a constructor, respectively. Const designates function names. Attr contains the attributes that are the main subject of our type theory; shared (!) and unique ($*$). Since the Lean 4 compiler erases type dependencies, we will limit ourselves to types that look like potentially recursive abstract data types.

\sloppy In $\mu\ x^\kappa_\adt.\ [[\tau_\field(x^\kappa_\adt, [y^\tau])] \to *x^\kappa_\adt]$, $x^\kappa_\adt$ is the variable we use to refer back to the ADT itself, $[[\tau_\field(x^\kappa_\adt, [y^\tau])] \to *x^\kappa_\adt]$ is a vector of constructors and $[\tau_\field(x^\kappa_\adt, [y^\tau])]$ denotes the types of the fields of the constructor, where $\tau_\field$ is parametrized by the variable $x^\kappa_\adt$ representing the ADT itself, as well as a vector $[y^\tau]$ of type parameters to the ADT. Constructors, projections and type parameters are assumed to be enumerated by intervals $[0, n)$, and so we write $(\mu\ x^\kappa_\adt.\ [[\tau_\field(x^\kappa_\adt, [y^\tau])] \to *x^\kappa_\adt])_i \coloneqq [[\tau_\field(x^\kappa_\adt, [y^\tau])] \to *x^\kappa_\adt]_i$ and $([\tau_\field(x^\kappa_\adt, [y^\tau])] \to *x^\kappa_\adt)_j \coloneqq [\tau_\field(x^\kappa_\adt, [y^\tau])]_j$, as well as $[y^\tau]_x \coloneqq x$. As Lean 4 code commonly interacts with external types and external code via its foreign function interface (FFI), we cannot assume that we can access an ADT declaration for every type. To deal with this, ADTs are instead identified by an ADTConst, the mapping of which is maintained in a global and partial function $\gamma \in \ADTDecls$. ADTConsts $A \notin \dom(\gamma)$ that appear in the program are regarded as external. Lastly, we demand that all $A \in \dom(\gamma)$ are fully propagated, i.e. that $\forall i\ j.\ \mathrm{propagate}(\gamma(A)_{ij}) = \gamma(A)_{ij}$ for the definition of propagate below.

AttrType contains our types. $m\ x^\kappa$ and $x^\tau$ are the two kinds of variables that can occur only within an ADT; self-referring variables $x^\kappa$ have an associated (fixed) attribute and the variable only represents a parameterless type suffix, while variables $x^\tau$ can denote any type parameter $\tau \in \AttrType$. $m\ \blacksquare$ is an erased type, $m\ A\ [\tau_\arrg]$ is an ADT (or external type) $A$ parametrized by type parameters $[\tau_\arrg]$, and $!\ [\tau_\param \to \tau_\ret]$ is the type of a higher-order function. Finally, $\delta_\tau$ provides the parameter- and return types for all functions in the program, including external ones. This is a reasonable assumption because we can simply assign a type $[!\ \kappa_\param] \to \ !\ \kappa_\ret$ for Lean 4 functions with an unattributed function type $[\kappa_\param] \to \ \kappa_\ret$, or the most general and useless type $[!\ \blacksquare] \to \ !\ \blacksquare$ to all external functions for which we have no type information at all. For $\delta_\tau(c) = ([\tau_\param], \tau_\ret)$, we also demand that all the types are fully propagated, i.e. that $\forall \tau_\param \in [\tau_\param].\ \mathrm{propagate}(\tau_\param) = \tau_\param \land \mathrm{propagate}(\tau_\ret) = \tau_\ret$ for the definition of propagate below.

\newcommand{\propagateWithinShared}{\mathrm{propagateWithinShared}}

\begin{alignat*}{3}
  \propagateWithinShared &: \mathrlap{\AttrType \to \AttrType} \\
  \propagateWithinShared&(m\ x^\kappa) &&=\ !\ x^\kappa \\
  \propagateWithinShared&(x^\tau) &&= x^\tau \\
  \propagateWithinShared&(m\ \blacksquare) &&=\ !\ \blacksquare \\
  \propagateWithinShared&(m\ A\ [\tau_\arrg]) &&=\ !\ A\ [\propagateWithinShared(\tau_\arrg)] \\
  \propagateWithinShared&\mathrlap{(!\ [\tau_\param] \to \tau_\ret)} \\
  \mathclap{\hspace{12em}= \ !\ [\propagateWithinShared(\tau_\param)] \to \propagateWithinShared(\tau_\ret)}
\end{alignat*}

\newcommand{\propagate}{\mathrm{propagate}}

\begin{alignat*}{3}
  \propagate &: \mathrlap{\AttrType \to \AttrType} \\
  \propagate&(m\ x^\kappa) &&= m\ x^\kappa \\
  \propagate&(x^\tau) &&= x^\tau \\
  \propagate&(m\ \blacksquare) &&= m\ \blacksquare \\
  \propagate&(*\ A\ [\tau_\arrg]) &&= *\ A\ [\propagate(\tau_\arrg)] \\
  \propagate&(!\ A\ [\tau_\arrg]) &&= \ !\ A\ [\propagateWithinShared(\tau_\arrg)] \\
  \propagate&\mathrlap{(!\ [\tau_\param] \to \tau_\ret)} \\
  \mathclap{\hspace{25em}= \ !\ [\propagateWithinShared(\tau_\param)] \to \propagateWithinShared(\tau_\ret)}
\end{alignat*}

propagate ensures that unique types are made shared if they are contained within a shared type, since a value within another value cannot be guaranteed to be unique if the outer value is already shared. 

We use the following notation for substitution:
\begin{alignat*}{3}
	a\{A, [\tau]\}\ &&\coloneqq&\ \mu\ x^\kappa_\adt.\ [[\propagate(\tau_\field[A\ [\tau]/x^\kappa_\adt][[\tau]/[y^\tau]])] \to *x^\kappa_\adt]
\end{alignat*}

It is worth pointing out a number of semantic nuances in both the definitions of our types and propagate above:
\begin{itemize}
	\item If we know that a type is unique, we can always throw away this guarantee and make it shared, as described in \cref{sec:uniqueness}.
	\item In $m\ \blacksquare$, $\blacksquare$ could be any other type, potentially parametrized by any other attributed type if $\blacksquare = A\ [\tau_\arrg]$. We must ensure that our type theory can deal with this kind of erasure.
	\item We must always ensure that types remain fully propagated.
	\item Within an ADT declaration $a$, we do not know how to propagate $x^\tau$, as it depends on the concrete parameter type. Instead, we ensure that type parameters become fully propagated when substituting the type variables for type parameters using our definition of $a\{A, [\tau]\}$.
	\item While we can propagate within a given ADT field or within any other given type, we cannot propagate from an outer $!\ A\ [\tau_\arrg]$ into the fields within $\gamma(A)$, as not all the attributes in $\gamma(A)$ are floated to the outer $!\ A\ [\tau_\arrg]$, only those in the type parameters $[\tau_\arrg]$. To alleviate this issue, we take an attribute $*$ in a field within $\gamma(A)$ to mean ``Unique if the outer value is unique'' and enforce this property in our type rules for projections on $m\ A\ [\tau_\arrg]$.
	\item Higher-order functions are always shared, so we do not need to worry about covariance or contravariance in either propagate or type parameters. This is a considerable limitation: Lean 4 code uses higher-order functions very liberally to encode type classes, monads, as well as some performance idioms related to the Counting Immutable Beans optimization described in \cref{sec:beans}. See \footnote{\todo{add refs}} for possible approaches to alleviate this issue in future work.
	\item Since external types have no associated declaration, if we want to gather information about the type, we must rely on auxiliary information provided by users at the FFI. We will need this kind of auxiliary information in \cref{sec:escapeanalysis} and \cref{sec:borrowing}.
\end{itemize}

We will now proceed to declare some convenient auxiliary functions. 

\newcommand{\weaken}{\mathrm{weaken}}

\begin{alignat*}{3}
  \weaken &: \mathrlap{\AttrType \to \AttrType} \\
  \weaken&(\tau) &&= \propagateWithinShared(\tau) \\
  !&(\tau) &&= \weaken(\tau)
\end{alignat*}

weaken makes the type argument $\tau$ shared and then propagates the attribute through the type. We will need this function whenever we have to make a type shared and we will avoid using it for the types of fields that have not been substituted yet.

\newcommand{\weakenInner}{\mathrm{weakenInner}}

\begin{alignat*}{3}
	\weakenInner &: \mathrlap{\AttrType \to \AttrType} \\
	\weakenInner&(m\ x^\kappa) &&= m\ x^\kappa \\
	\weakenInner&(x^\tau) &&= x^\tau \\
	\weakenInner&(m\ \blacksquare) &&= m\ \blacksquare \\
	\weakenInner&(m\ A\ [\tau_\arrg]) &&= m\ A\ [!(\tau_\arrg)] \\
	\weakenInner&(!\ [\tau_\param] \to \tau_\ret) &&=\ !\ [\tau_\param] \to \tau_\ret \\
\end{alignat*}

weakenInner leaves the outer attribute intact but weakens every inner type. This will be useful when dealing with erased types $m\ \blacksquare$: When casting $m\ \blacksquare$ to another type $\tau$, we want $\tau$ to retain the outer attribute $m$, but we cannot make any guarantees for the inner attributes, and so we weaken them. We will avoid using it for the types of fields that have not been substituted yet.

\newcommand{\strengthen}{\mathrm{strengthen}}

\begin{alignat*}{3}
  \strengthen &: \mathrlap{\AttrType \to \AttrType} \\
  \strengthen&(m\ x^\kappa) &&= *\ x^\kappa \\
  \strengthen&(x^\tau) &&= x^\tau \\
  \strengthen&(m\ \blacksquare) &&= *\ \blacksquare \\
  \strengthen&(m\ A\ [\tau_\arrg]) &&= *\ A\ [\strengthen(\tau_\arrg)] \\
  \strengthen&(!\ [\tau_\param] \to \tau_\ret) &&=\ !\ [\tau_\param] \to \tau_\ret \\
\end{alignat*}

strengthen makes every attribute within a type unique which can be made unique. We will use this function for inferring the type parameters of $m\ A\ [\tau_\arrg]$ at construction: If a type parameter variable is not assigned by any constructor argument, we can strengthen it. We will also avoid using it for the types of fields that have not been substituted yet.

\newcommand{\attr}{\mathrm{attr}}

\begin{alignat*}{3}
  \attr &: \mathrlap{\AttrType \rightharpoonup \Attr} \\
  \attr&(m\ x^\kappa) &&= m \\
  \attr&(m\ \blacksquare) &&= m \\
  \attr&(m\ A\ [\tau_\arrg]) &&= m \\
  \attr&(!\ [\tau_\param] \to \tau_\ret) &&=\ ! \\
\end{alignat*}

attr simply yields the outer attribute of any $\tau \neq x^\tau$.

Finally, whenever we pass a type $\tau_1$ to a type $\tau_2$, we must ask ourselves whether $\tau_1$ can be applied to $\tau_2$. The type structure must be the same, but it should be possible to throw away the uniqueness attribute of types within $\tau_1$. Hence, we use $m_1 \succcurlyeq_m m_2 :\Leftrightarrow m_1 = * \lor m_1 =\ ! \land m_2 =\ !$ to denote attribute subtyping and define a subtyping relation $\succcurlyeq$ for fully propagated types $\tau_1$ and $\tau_2$ as follows:
\begin{mathpar}
	\boxed{\tau_1 \succcurlyeq \tau_2} \hspace{1.5em}
	$\inferrule{m_1 \succcurlyeq_m m_2}{m_1\ \blacksquare \succcurlyeq m_2\ \blacksquare}$ \hspace{1.5em}
	$\inferrule{m_1 \succcurlyeq_m m_2}{m_1\ x^\kappa \succcurlyeq m_2\ x^\kappa}$ \hspace{1.5em}
	$\inferrule{ }{x^\tau \succcurlyeq x^\tau}$
\end{mathpar}
\begin{mathpar}
	$\inferrule{ }{!\ [\tau_\param] \to \tau_\ret \succcurlyeq\ !\ [\tau_\param] \to \tau_\ret}$ \hspace{1.5em}
	$\inferrule{m_1 \succcurlyeq_m m_2 \\ [\tau_{\arrg_1} \succcurlyeq \tau_{\arrg_2}]}{m_1\ A\ [\tau_{\arrg_1}] \succcurlyeq m_2\ A\ [\tau_{\arrg_2}]}$
\end{mathpar}
Note that if higher-order functions could be unique, we would have to account for covariance and contravariance in this definition.

\section{Intermediate Representation}\label{sec:ir}
For our model of the IR, we use a mixture of the IR described by \cite{ullrich_counting_2020} and the newly implemented LCNF, both detailed in \cref{sec:irs}. 

\newcommand{\Expr}{\mathrm{Expr}}
\newcommand{\FnBody}{\mathrm{FnBody}}
\newcommand{\Fn}{\mathrm{Fn}}
\newcommand{\Program}{\mathrm{Program}}

\begin{alignat*}{3}
  e &\in \Expr &\Coloneqq&\ \icode{c [y]}
    \sep \icode{pap c [y]}
    \sep \icode{x y}
    \sep \icode{(A [τ?]).ctorᵢ [y]} \\
    &&&\enspace\sep \icode{projᵢⱼ y} \\
  F &\in \FnBody &\Coloneqq&\ \icode{ret x}
    \sep \icode{let x := e; F}
    \sep \icode{case x of [F]} \\
    &&&\enspace\sep \icode{case' x of [ctorᵢ [y] ⇒ F]}\\
  f &\in \Fn &\Coloneqq&\ \lambda\ [y].\ F \\
  \delta &\in \Program &=&\ \Const \rightharpoonup \Fn
\end{alignat*}

Expr and FnBody are similar to Lean's IR, except for our definition of \icode{proj} and \icode{ctor}, as well as the addition of a new instruction \icode{case'}.

\icode{proj} is provided not just with the projection $j$ as in Lean's IR, but also the constructor $i$. As the code generation ensures that \icode{proj} calls always occur after \icode{case} within the same function if the type has multiple constructors or on its own if the type only has a single constructor, we can easily compute $i$ by walking back from the \icode{projⱼ y} call either to the start of the function to set $i = 0$ or to a \icode{case x of [...]} instruction, where we choose $i$ as the index of the branch that we are walking back from.

\icode{ctor} takes an additional vector of explicit attributed type arguments $[\tau?]$, where $?$ refers to each explicit argument being optional. Since users do not provide them, Lean can provide us with type arguments $[\kappa]$ for the constructor call, but not any of the attributes, and so we must infer them from the types of arguments provided in $[y]$. But since there may be type arguments to $A$ that occur only in the other constructors for $A$, we cannot infer all of them, and so they must be provided explicitly. However, type arguments that do not occur in $[y]$ are also not subject to any uniqueness constraints, and so we can instantiate them as strongly as possible. Subsequently, the attributes in $[\tau?]$ can be chosen arbitrarily: If the type argument occurs in $[y]$, we can infer the type together with its attributes, and if it does not occur in $[y]$, the type $\tau_e$ must be provided in $[\tau?]$, but the corresponding attributes can be chosen as given by $\strengthen(\tau_e)$.

The \icode{case} and \icode{proj} combination turns out to be unwieldy for substructural type systems: When we use \icode{let z := projᵢⱼ y; F} on a unique value to obtain another unique value, the contained value now exists both in $z$ and in $y$, i.e. uniqueness is violated. The solution to this issue would be that \icode{projᵢⱼ y} consumes our unique value $y$ so that it is not available in F any more. However, it is very common that we would like to access multiple fields of $y$ in succession, which we will not be able to do now that $y$ is consumed. So, instead, the typical solution to this issue in substructural type systems is not to access fields via projections, but using a single destructuring pattern match that yields all fields of the type in one go and consumes the variable associated with the type. This is exactly what \icode{case'} does as well, and while the instruction does not exist in Lean's IR, it does exist in Lean 4's LCNF. Regardless, even in LCNF, structures are still accessed via projections and not using a destructuring pattern match. To alleviate this final issue, we implement a compromise in \cref{sec:checking} which ensures that we can use multiple projections on $y$, but not use it in any other manner.

As in Lean's IR, a global and partial $\delta \in \Program$ assigns function declarations to constants. All $c \notin \dom(\delta)$ that occur in the program are assumed to be external functions.

Finally, there are a number of omissions from our IR compared to the IR implemented in Lean. Most notably, there are instructions to work with join points, which we could implement as functions in our IR. However, it is worth noting that join points, like auxiliary functions $c$ generated by the Lean compiler, do not necessarily have an associated user-provided type $\delta_\gamma(c)$. Unfortunately, we will not touch on the topic of type inference in this thesis.

\newcommand{\Tag}{\mathrm{Tag}}
\newcommand{\Escapee}{\mathrm{Escapee}}
\newcommand{\ExternFunEscapees}{\mathrm{ExternFunEscapees}}

\section{Escape Analysis}\label{sec:escapeanalysis}
As described in \cref{sec:borrowingbackground}, borrowing in functional languages is closely related to escape analysis; if nothing within a shared parameter escapes, then we do not have to make a unique argument to that parameter shared, as the caller is guaranteed to still hold the only reference to the object in question when the called function returns. Instead of unloading this additional burden of tracking the data flow of variables and fields to the user, we implement a data flow analysis, i.e. an instance of abstract interpretation.

\begin{alignat*}{3}
  n, m &\in \mathbb{N} \\
  s, t, v &\in \Tag &\Coloneqq&\ \#\textrm{const c} \sep \#\textrm{case i} \sep \#\textrm{app} \sep \#\textrm{param n} \\
  q &\in \Escapee\ &\Coloneqq&\ x_{[ij]} @ [t]? \\
  \delta_{q_e} &\in \ExternFunEscapees &=&\ \Const \rightharpoonup 2^{\Escapee}
\end{alignat*}

Escapees are the subject of our escape analysis and represent the elements of the sets that we compute. Each escapee has an associated variable $x$, a field index $[ij]$ represented by a vector of $\Ctor \times \Proj$ tuples and a vector of tags that describes the path to the parameter an escapee was spawned from, if the escapee came from a function call. The need for the vector of tags will become obvious later, and until then it can just be understood as an identifier that identifies the location in the code where escapees from function calls were spawned. Since external funtions do not have a function body that we can analyze, a global and partial function $\delta_{q_e}$ allows specifying the set of all escapees for these functions. For external functions $c \notin \dom(\delta_{q_e})$ we will assume all parameters and all fields thereof to escape.

\newcommand{\ecp}[2]{\llbracket {#1} \rrbracket_Q \left( {#2} \right)}

Using abstract interpretation, we compute a least fixed point of the following mutually recursive equations $\ecp{\cdot}{\cdot}$ and $\delta_Q$, which we will explain in detail along the way. The first parameter of $\ecp{\cdot}{\cdot}$ is the portion of the function body that we want to compute the escapees for, the second parameter denotes the vector of tags thus far from the start of the function to this portion of the function body.

\begin{align*}
  &\ecp{\cdot}{\cdot} : \mathrlap{\FnBody \times [\Tag] \to 2^{\Escapee}} \\
  &\ecp{\icode{ret x}}{[t]} =
    \left\{x_{[]}\right\} \\
  &\ecp{\icode{case x of [F]}}{[t]}\ =\
    \bigcup_n \ecp{[F]_n}{\#\text{case n} :: [t]} \\
  &\ecp{\icode{case' x of [ctorᵢ [y] ⇒ F]}}{[t]}\ =\
    \bigcup_n \ecp{[F]_n}{\#\text{case n} :: [t]} \\
    \mathclap{\hspace{39em}\cup \left\{x_{nm :: [kj]} @ [s]?
    \ \ |\ \ ([y]_m)_{[kj]} @ [s]? \in \ecp{[F]_n}{\#\text{case n} :: [t]}\ \land\ m \in [0, |[y]|) \right\}} \\
  &\ecp{\icode{let x = c [y]; F}}{[t]} = Q_F \\
  \mathclap{\hspace{34.7em}\cup \begin{cases}
  	Q & \exists [nm].\ x_{[nm]} @ [v]? \in Q_F \land c \in \dom(\delta_Q) \\
    \left\{y_{[]} \ | \ y \in [y]\right\} & \exists [nm].\ x_{[nm]} @ [v]? \in Q_F \land c \notin \dom(\delta_Q) \land c \notin \dom(\delta) \\
  	\emptyset & \exists [nm].\ x_{[nm]} @ [v]? \in Q_F \land c \notin \dom(\delta_Q) \land c \in \dom(\delta) \\
  	\emptyset & \lnot \exists [nm].\ x_{[nm]} @ [v]? \in Q_F
  \end{cases}} \\
  \mathclap{\hspace{34em} \text{where } Q := \left\{([y]_z)_{[ij]} @ (\#\text{param z} :: [t]) \ |\ z_{[ij]} @ [s]? \in \delta_Q(c) \right\}}\\
  \mathclap{\hspace{21.4em}\text{ and } Q_F := \ecp{F}{\#\text{app}::[t]}}\\
  &\ecp{\icode{let x = pap c [y]; F}}{[t]} = \ecp{F}{[t]} \\
  \mathclap{\hspace{26.5em}\cup \begin{cases}
  	\left\{y_{[]} \ | \ y \in [y]\right\} & \exists [nm].\ x_{[nm]} @ [v]? \in \ecp{F}{[t]} \\
  	\emptyset & \lnot \exists [nm].\ x_{[nm]} @ [v]? \in \ecp{F}{[t]}
  \end{cases}}\\
  &\ecp{\icode{let x = y z; F}}{[t]} = \ecp{F}{[t]} \\
  \mathclap{\hspace{24em}\cup \begin{cases}
      \left\{y_{[]}, z_{[]}\right\} & \exists [nm].\ x_{[nm]} @ [v]? \in \ecp{F}{[t]} \\
      \emptyset & \lnot \exists [nm].\ x_{[nm]} @ [v]? \in \ecp{F}{[t]}
  	\end{cases}}\\
  &\ecp{\icode{let x = (A [τ?]).ctorᵢ [y]; F}}{[t]} = \ecp{F}{[t]} \\
  \mathclap{\hspace{34em}\cup \begin{cases}
     	\left\{y_{[]} \ | \ y \in [y]\right\} & x_{[]} @ [v]? \in \ecp{F}{[t]} \\
     	\left\{([y]_j)_{[nm]} @ [s]? \ | \ x_{ij :: [nm]} @ [s]? \in \ecp{F}{[t]} \right\} & x_{[]} @ [v]? \notin \ecp{F}{[t]}
  \end{cases}} \\
  &\ecp{\icode{let x = projᵢⱼ y; F}}{[t]} = \ecp{F}{[t]} \\
  \mathclap{\hspace{36.5em}\cup \begin{cases}
    \left\{y_{ij :: [kl]} @ [s]? \ | \ x_{[kl]} @ [s]? \in \ecp{F}{[t]} \right\} & \exists [nm].\ x_{[nm]} @ [v] \in \ecp{F}{[t]} \\
    \emptyset & \lnot \exists [nm].\ x_{[nm]} @ [v]? \in \ecp{F}{[t]}
  \end{cases}}
\end{align*}
In \icode{ret x}, only $x$ itself escapes. For \icode{case x of [F]}, we determine the escapees of each $F \in [F]$ and compute the resulting union of all escapees. In \icode{case' x of [ctorᵢ [y] ⇒ F]}, we use the same idea as for \icode{case}, but must also transfer escapees concerning $[y]$ over to $x$: Each escapee $([y]_m)_{[kj]} @ [s]?$ in branch $n$ corresponding to constructor $n$ is converted to an escapee $x_{nm :: [kj]} @ [s]?$. 

For all \icode{let x = e; F} function bodies, we will always have a case stating that if $x$ does not escape in $F$, then neither do we need to compute any additional escapees for $e$.

Application \icode{let x = c [y]; F} is the most tricky since it is the spot where our analysis recurses with escapees for $c$. If we have already computed escapees for $c$ or they are specified in $\delta_{q_e}$, i.e. $c \in \dom(\delta_Q)$, we take all escapees $z_{[ij]}@[s]?$ for parameters $z$ from $\delta_Q$ and rename them to the corresponding arguments $[y]_z$. If $c \notin \dom(\delta_{q_e})$ is external, all $[y]$ are assumed to escape. Finally, if we have not already computed the escapees for $c$ but are expected to do so in the future because $c$ is not external, we yield the bottom element of our lattice $\bot = \emptyset$.

When creating a higher-order function using \icode{pap c [y]}, we assume that all $[y]$ escape if the resulting higher-order function escapes. The same is true for higher-order function application \icode{y z}: If the result escapes, then so may $y$ and $z$. Note that creating an escapee for $y$, i.e. the higher-order function itself, is important, because if the higher-order function containing all the previously-applied arguments or the return value of the function escapes, we need to know that the previously-applied arguments may escape too, and so we propagate this bit of information backwards using the escapee for $y$.

\icode{(A [τ?]).ctorᵢ [y]} and \icode{projᵢⱼ y} are once again fairly straight-forward. If the ADT resulting from a constructor call escapes, then so do all of its fields, and if only particular fields of constructor $i$ escape, then the respective escapees $x_{ij :: [nm]} @ [s]?$ must be translated to escapees for $[y]_j$ by removing the $ij$ field. Other escapees $x_{kj :: [nm]} @ [s]?$ for $k \neq i$ do not need to be translated.
For \icode{projᵢⱼ y}, we use the same idea as for \icode{case'} and translate the escapees for the projection to ones for $y$.

Next, we will define a couple of post-processing functions to make our application of abstract interpretation to the mutually recursive $\ecp{\cdot}{\cdot}$ and $\delta_Q$ terminate. 

\newcommand{\fd}{\mathrm{fd}}
\newcommand{\ct}{\mathrm{ct}}
\newcommand{\fs}{\mathrm{fs}}

\newcommand{\collapse}{\mathrm{collapse}}

\begin{alignat*}{3}
	\fd &: \mathrlap{\mathbb{N} \times 2^{\Escapee} \to 2^{\Escapee}} \\
	\fd&(\mathrm{arity}, Q) &&= \left\{x_{[ij]}@[t]? \ |\ x_{[ij]}@[t]? \in Q \land x \in [0, \mathrm{arity}) \right\}
\end{alignat*}
fd removes all dead escapees for a function of a particular arity by keeping only those that correspond to function parameters. This is mainly useful for performance because $\ecp{\cdot}{\cdot}$ accumulates escapees for all variables in a function, even local ones.

\begin{alignat*}{3}
	\fs &: \mathrlap{2^{\Escapee} \to 2^{\Escapee}} \\
	\fs&(Q) &&= \left\{q \ |\ q \in Q \land \lnot \exists q' \in Q.\ q \neq q' \land q' \subset q\right\}
\end{alignat*}
Here, $x_{[i_1j_1]} @ [t_1]? \subset x_{[i_1j_1]++[i_2j_2]} @ [t_2]?$ asserts that $x_{[i_1j_1]} @ [t_1]?$ subsumes $x_{[i_1j_1]++[i_2j_2]} @ [t_2]?$. Hence, fs removes all escapees which are subsumed by another escapee.

\begin{alignat*}{1}
	&\equiv_t : \Escapee \times \Escapee \to \mathbb{B} \\
	&x_{[ij]} @ [s]? \equiv_t y_{[kl]} @ [v]? :\Leftrightarrow [s?] = [v?]
\end{alignat*}
\begin{align*}
	\collapse &: \Escapee \times \Escapee \rightharpoonup \Escapee \\
	\collapse&(x_{[i_1j_1]} @ [t]?, x_{[i_2j_2]} @ [t]?) = x_{lcp([i_1j_1], [i_2j_2])} @ [t]?
\end{align*}
\begin{alignat*}{3}
	\ct &: \mathrlap{2^{\Escapee} \rightharpoonup 2^{\Escapee}} \\
	\ct&(Q) &&= \left\{\mathrm{fold}(\collapse, [x_{[ij]} @ [s]])\ |\ [x_{[ij]} @ [s]] \in Q/{\equiv_t} \right\}
\end{alignat*}
ct is the key post-processing function that makes our escape analysis terminate and finally makes use of the tags that we have been keeping track of in $\ecp{\cdot}{\cdot}$. The key idea is that we take escapees with the same vector of tags, i.e. equivalence classes in $Q/{\equiv_t}$, and collapse them so that we get an escapee with a field that is the longest common prefix of all the fields of escapees in an equivalence class. Note that in such an equivalence class, as all escapees have been created from the same parameter at the same call site, all these escapees must use the same variable.

Without ct, the corresponding lattice over $2^{\Escapee}$ does not have finite height: In an escapee $x_{[ij]} @ [t]?$, we can bound $x$ by all the variables that are possible in the program and $t?$ by every single call site in the program, but the field $[ij]$ may diverge, e.g. when attempting to run the abstract interpretation on a recursive function with a \icode{List}-type, in which case we will keep prepending fields to the respective escapee, and they will not subsume one another. 

Collapsing the escapees with the same tag effectively bounds the lattice: Because there are only finitely many call sites, if the abstract interpretation is executed on a program where it would otherwise diverge, it must necessarily eventually visit the same call site twice and yield an escapee with the same variable but a different field. Collapsing all these escapees from the same call site thus computes a more general escapee that subsumes all the previous escapees from that call site, ensuring that we can iteratively reduce the field in which we were diverging up to a bound of $[]$, where we are guaranteed to terminate.

\begin{align*}
	\delta_Q &: \mathrlap{\Const \rightharpoonup 2^{\Escapee}} \\
	\delta_Q&(c) = \begin{cases}
		\fs(|[y]|, \ct(\fd(\ecp{F}{[\#\text{const c}]}))) & c \in \dom(\delta) \land \delta(c) = \lambda\ [y].\ F \\
		\delta_{q_e}(c) & c \notin \dom(\delta) \land c \in \dom(\delta_{q_e})
	\end{cases}
\end{align*}

Finally, $\delta_Q$ computes the escapees of every function in the program and and uses $\delta_{q_e}$ to obtain escapees for some external functions.

In order to perform the abstract interpretation, we use Kosaraju's algorithm\footnote{\todo{citation}} to compute strongly connected components in the call graph of functions $c \in \dom(\delta)$, traverse the resulting graph of strongly connected components in reverse topological sort and then iteratively compute $\delta_Q(c)$ within each strongly connected component of mutually recursive functions until we reach a fixed point.

\section{Borrowing}\label{sec:borrowing}
When checking whether a parameter can be borrowed, we do not need to check whether fields that are always shared escape, only fields that are unique if their outer value is unique. In this section, we will define the function $\delta_\mathbb{B}$ that tells us which parameters of a function can be borrowed when the function is applied. Henceforth, we will use $[x] \leq_+ [y]$ to denote that $[x]$ is a prefix of $[y]$.

\newcommand{\ExternUniqueFieldResult}{\mathrm{ExternUniqueFieldResult}}
\newcommand{\ExternUniqueField}{\mathrm{ExternUniqueField}}
\newcommand{\ExternUniqueFields}{\mathrm{ExternUniqueFields}}

\begin{alignat*}{3}
	r_{*_e} &\in \ExternUniqueFieldResult\ &\Coloneqq& *_r \sep !_r \sep ?_r\ x^\tau\ [ij] \\
	f_{*_e} &\in \ExternUniqueField\ &\Coloneqq&\ [ij] (x^\tau)? \\
	\gamma_{*_e} &\in \ExternUniqueFields\ &=&\ \ADTConst \rightharpoonup 2^{\ExternUniqueField}
\end{alignat*}

For external types $A \notin \dom(\gamma)$, we cannot compute which fields are unique if their outer value is unique. Indeed, one might wonder what this even means for external types that are not managed by Lean, and the answer is that this is simply a mechanism that allows FFI-based libraries additional control over escape analysis, where a library specifying that a field of an external type escapes does not always have to block borrowing, e.g. when the uniqueness of the escapee depends on a type parameter that is always shared in particular function parameter. The other key property that we gain from the specification of unique fields on external types is that they provide a reasonable specification for which escapees are interesting to downstream external functions, the adherence to which could, in principle, even be checked automatically.

We assume that all $\gamma_{*_e}(A)$ are the leafs of a nonempty tree, i.e. there are no $[ij] (x^\tau)?,\ [kl] (y^\tau)? \in \gamma_{*_e}(A)$ s.t. $[ij] \neq [kl]$ but $[ij] \leq_+ [kl]$ or $[kl] \leq_+ [ij]$.

\newcommand{\eu}{\mathrm{eu}}
\newcommand{\paath}{\mathrm{path}}

\begin{alignat*}{3}
	\eu &: \mathrlap{\ADTConst \times [\Ctor \times \Proj] \rightharpoonup \ExternUniqueFieldResult} \\
	\eu&(A, \paath) &&= \\
	\mathclap{\hspace{34em}\begin{cases}
		?_r\ x^\tau\ [kl]	& A \in \dom(\gamma_{*_e}) \land \exists [ij] (x^\tau) \in \gamma_{*_e}(A).\ \exists [kl].\ [ij] ++ [kl] = \paath \\
		*_r	& A \in \dom(\gamma_{*_e}) \land \exists [ij] (x^\tau) \in \gamma_{*_e}(A).\ \paath \leq_+ [ij] \land \paath \neq [ij] \\
		*_r	& A \in \dom(\gamma_{*_e}) \land \exists [ij] \in \gamma_{*_e}(A).\ \paath \leq_+ [ij] \lor [ij] \leq_+ \paath \\
		!_r	& A \in \dom(\gamma_{*_e}) \land \lnot \exists [ij] (x^\tau)? \in \gamma_{*_e}(A).\ \paath \leq_+ [ij] \lor [ij] \leq_+ \paath \\
		*_r & A \notin \dom(\gamma_{*_e})
	\end{cases}}
\end{alignat*}

\newcommand{\isUnique}{\mathrm{isUnique}}
\newcommand{\rest}{\mathrm{rest}}

\begin{alignat*}{3}
	\isUnique &: \mathrlap{\AttrType \times [\Ctor \times \Proj] \rightharpoonup \mathbb{B}} \\
	\isUnique&(*\ x^\kappa, \paath) &&= \top \\
	\isUnique&(*\ \blacksquare, \paath) &&= \top \\
	\isUnique&(*\ A\ [\tau_\arrg], []) &&= \top \\
	\isUnique&(*\ A\ [\tau_\arrg], \paath@((i, j)::\rest)) &&= \\
	\mathclap{\hspace{28em}\begin{cases}
		\isUnique(\gamma(A)\{A, [\tau_\arrg]\}_{ij}, \rest)& A \in \dom(\gamma) \\
		\top & A \notin \dom(\gamma) \land \eu(A, \paath) = *_r \\
		\bot & A \notin \dom(\gamma) \land \eu(A, \paath) =\ !_r \\
		\isUnique([\tau_\arrg]_{x^\tau}, [kl]) & A \notin \dom(\gamma) \land eu(A, \paath) =\ ?_r\ x^\tau\ [kl]
	\end{cases}} \\
	\isUnique&(!\ x^\kappa, \paath) &&= \bot \\
	\isUnique&(!\ \blacksquare, \paath) &&= \bot \\
	\isUnique&(!\ A\ [\tau_\arrg], \paath) &&= \bot \\
	\isUnique&(!\ [\tau_\param] \to \tau_\ret, \paath) &&= \bot
\end{alignat*}

\begin{alignat*}{3}
	\gamma_* &: \mathrlap{\ADTConst \times [\AttrType] \to 2^{[\Ctor \times \Proj]}} \\
	\gamma_*&(A, [\tau_\arrg]) &&= \left\{ p \ |\ \isUnique(*\ A\ [\tau_\arrg], p) = \top \land p \neq [] \right\}
\end{alignat*}

\newcommand{\isBorrowed}{\mathrm{isBorrowed}}

\begin{alignat*}{3}
	\isBorrowed &: \mathrlap{\mathbb{N} \times \AttrType \times 2^{\Escapee} \to \mathbb{B}} \\
	\isBorrowed&(x, !\ A\ [\tau_\arrg], Q) &&= x_{[]}@[t]? \notin Q \land \forall x_{[ij]}@[t]? \in Q.\ [ij] \notin \gamma_*(A, \tau_\arrg) \\
	\isBorrowed&(x, \tau_\param, Q) &&= x_{[]}@[t]? \notin Q \land \attr(\tau_\param) =\ ! \qquad\text{otherwise}
\end{alignat*}

\begin{alignat*}{3}
	\delta_\mathbb{B} &: \mathrlap{\Const \to 2^\mathbb{N}} \\
	\delta_\mathbb{B}&(c) &&= \begin{cases}
		\left\{x \ |\ \isBorrowed(x, [\tau_\param]_x, \delta_Q(c)_x) \right\} & c \in \dom(\delta_Q) \\
		\emptyset & c \notin \dom(\delta_Q)
	\end{cases}\\
	&\mathrlap{\text{where}\ \delta_\tau(c) = ([\tau_\param], \tau_\ret)} \\
	&\mathrlap{\hspace{0.2em}\text{and}\ \delta_Q(c)_x \coloneqq \left\{y_{[ij]}@[t]? \ |\ y_{[ij]}@[t]? \in \delta_Q(c) \land y = x \right\}}
\end{alignat*}

\section{Type Checking}\label{sec:checking}

\newcommand{\ZeroedFields}{\mathrm{ZeroedFields}}
\newcommand{\Context}{\mathrm{Context}}

\begin{alignat*}{3}
	Z &\in \ZeroedFields\ &=&\ \Var \times \Ctor \times \Proj \to \mathbb{B} \\
	\Gamma &\in \Context  &\Coloneqq&\ [] \sep \Gamma, x : \tau
\end{alignat*}

We assume $\Gamma$ to be a multiset, i.e. we track duplicate judgements, but not the order of the context. Note that the latter would be required in dependent type theory, as the order of type dependencies must be retained.

\begin{mathpar}
	\boxed{\vdash \delta_\tau} \hspace{1.5em}
	$\inferrule[Program]{\forall c \in \dom(\delta_\tau) \cap \dom(\delta) \text{ s.t. } \delta(c) = \lambda [y] \ F \land \delta_\tau(\tau) = ([\tau_\param], \tau_\ret).\\ 
		\emptyset; [y : \tau_\param] \vdash F : \tau_\ret}
	{\vdash \delta_\tau}$
\end{mathpar}

\newcommand{\nz}{\mathrm{nz}}

\begin{alignat*}{3}
	\nz &: \mathrlap{\ZeroedFields \times \Var \to \mathbb{B}} \\
	\nz&(Z, x) &&= \lnot \exists i\ j.\ Z(x, i, j) = \top
\end{alignat*}

\newcommand{\inferVarsDash}{\mathrm{inferVars'}}

\begin{alignat*}{3}
	\inferVarsDash &: \mathrlap{\AttrType \times \AttrType \rightharpoonup (\Var \rightharpoonup \AttrType)} \\
	\inferVarsDash&(m\ x^\kappa, \tau) &&= \emptyset \\
	\inferVarsDash&(x^\tau, \tau) &&= \{ x^\tau \mapsto \tau \} \\
	\inferVarsDash&(m\ \blacksquare, m\ \blacksquare) &&= \emptyset \\
	\inferVarsDash&(m\ A\ [\tau_{\arrg_1}], m\ A\ [\tau_{\arrg_2}]) &&= \bigcupdot_i \inferVarsDash([\tau_{\arrg_1}]_i, [\tau_{\arrg_2}]_i) \\
	\inferVarsDash&(!\ [\tau_{\param_1}] \to \tau_{\ret_1}, !\ [\tau_{\param_2}] \to \tau_{\ret_2}) &&= \\
	&\mathrlap{\bigcupdot_i \inferVarsDash([\tau_{\param_1}]_i, [\tau_{\param_2}]_i) \cupdot \inferVarsDash(\tau_{\ret_1}, \tau_{\ret_2})}
\end{alignat*}

\newcommand{\inferVars}{\mathrm{inferVars}}

\begin{alignat*}{3}
	\inferVars &: \mathrlap{[\AttrType] \times [\AttrType] \rightharpoonup (\Var \rightharpoonup \AttrType)} \\
	\inferVars&([], []) &&= \emptyset \\
	\inferVars&(\tau_1 :: \rest_1, \tau_2 :: \rest_2) &&= \inferVarsDash(\tau_1, \tau_2) \cupdot \inferVars(\rest_1, \rest_2) \\
\end{alignat*}

\newcommand{\pickTypes}{\mathrm{pickTypes}}

\begin{alignat*}{3}
	\pickTypes &: \mathrlap{[\AttrType?] \times [\AttrType?] \rightharpoonup [\AttrType]} \\
	\pickTypes&([]) &&= [] \\
	\pickTypes&(\tau_e? :: \rest_e, \tau_i :: \rest_i) &&= \tau_i :: \pickTypes(\rest_e, \rest_i) \\
	\pickTypes&(\tau_e :: \rest_e, - :: \rest_i) &&= \strengthen(\tau_e) :: \pickTypes(\rest_e, \rest_i) \\
\end{alignat*}

\newcommand{\inferTypeArgs}{\mathrm{inferTypeArgs}}

\begin{alignat*}{3}
	\inferTypeArgs &: \mathrlap{\ADT \times \mathbb{N} \times [\AttrType] \times [\AttrType?] \rightharpoonup [\AttrType]} \\
	\inferTypeArgs&(a, i, [\tau_\arrg], [\tau_e?]) &&= \pickTypes([\tau_e?], [\mathrm{inferred}(y^\tau)])\\
		& \mathrlap{\text{where } a_i = [\tau_\field(x^\kappa_\adt, [y^\tau])] \to *x^\kappa_\adt} \\
		& \mathrlap{\text{and } \mathrm{inferred} = \inferVars([\tau_\field(x^\kappa_\adt, [y^\tau])], [\tau_\arrg]])}
\end{alignat*}

\newcommand{\tret}{\tau_\text{ret'}}

\begin{mathpar}
	\boxed{Z; \Gamma \vdash F : \tau}
\end{mathpar}
\begin{mathpar}
	$\inferrule[Duplicate]{Z; \Gamma, x :\ !\tau, x :\ !\tau \vdash F : \tret}{Z; \Gamma, x :\ !\tau \vdash F : \tret}$ \hspace{1.5em}
	$\inferrule[Forget]{Z; \Gamma \vdash F : \tret}{Z; \Gamma, x :\ !\tau \vdash F : \tret}$
\end{mathpar}
\begin{mathpar}
	$\inferrule[Downcast]{\tau \succcurlyeq \tau' \\ \nz(Z, x) \\ Z; \Gamma, x : \tau' \vdash F : \tret}{Z; \Gamma, x : \tau \vdash F : \tret}$
\end{mathpar}
\begin{mathpar}
	$\inferrule[$\blacksquare$-Cast]{Z; \Gamma, x : \weakenInner(\tau) \vdash F : \tret}{Z; \Gamma, x : \attr(\tau)\ \blacksquare \vdash F : \tret}$ \hspace{1.5em}
	$\inferrule[$\blacksquare$-Erase]{Z; \Gamma, x : \attr(\tau)\ \blacksquare \vdash F : \tret}{Z; \Gamma, x : \tau \vdash F : \tret}$
\end{mathpar}
\begin{mathpar}
	$\inferrule[Ret]{\nz(Z, x)}{Z; \Gamma, x : \tret \vdash \icode{ret x} : \tret}$ \hspace{1.5em}
	$\inferrule[Case]{\nz(Z, x) \\ [Z; \Gamma, x : m\ A\ [\tau_\arrg] \vdash F : \tret]}{Z; \Gamma, x : m\ A\ [\tau_\arrg] \vdash \icode{case x of [F]} : \tret}$
\end{mathpar}
\begin{mathpar}
	$\inferrule[Case'-!]
		{\nz(Z, x) 
				\\ A \in \dom(\gamma)
				\\ \gamma(A)\{A, [\tau_\arrg]\} = \mu\ x^\kappa_\adt.\ [[\tau_\field] \to *x^\kappa_\adt]
				\\ [Z; \Gamma, [y :\ !\tau_\field] \vdash F : \tret]}
		{Z; \Gamma, x : \ !\ A\ [\tau_\arrg] \vdash \icode{case' x of [ctorᵢ [y] ⇒ F]} :  \tret}$
\end{mathpar}
\begin{mathpar}
	$\inferrule[Case'-*]
	{\nz(Z, x) 
		\\ A \in \dom(\gamma)
		\\ \gamma(A)\{A, [\tau_\arrg]\} = \mu\ x^\kappa_\adt.\ [[\tau_\field] \to *x^\kappa_\adt]
		\\ [Z; \Gamma, [y : \tau_\field] \vdash F : \tret]}
	{Z; \Gamma, x : *\ A\ [\tau_\arrg] \vdash \icode{case' x of [ctorᵢ [y] ⇒ F]} :  \tret}$
\end{mathpar}
\begin{mathpar}
	$\inferrule[Let-App]{[\nz(Z, y)] 
		\\ \delta_\tau(c) = ([\tau_\param], \tau_\ret)
		\\ Z; \Gamma, \{ [y]_x : [\tau_\param]_x \ |\ x \in \delta_\mathbb{B}(c) \}, z : \tau_\ret \vdash F : \tret
	}
	{Z; \Gamma, [y : \tau_\param] \vdash \icode{let z := c [y]; F} : \tret}$
\end{mathpar}
\begin{mathpar}
	$\inferrule[Let-Pap-Full]{[\nz(Z, y)] 
		\\ \delta_\tau(c) = ([\tau_\param], \tau_\ret)
		\\ |[y]| = |[\tau_\param]|
		\\ Z; \Gamma, z :\ !\tau_\ret \vdash F : \tret
	}
	{Z; \Gamma, [y :\ !\tau_\param] \vdash \icode{let z := pap c [y]; F} : \tret}$
\end{mathpar}
\begin{mathpar}
	$\inferrule[Let-Pap-Part]{[\nz(Z, y)] 
		\\ \delta_\tau(c) = ([\tau_\param], \tau_\ret)
		\\ |[y]| = |[\tau_{\param_1}]| < |[\tau_\param]|
		\\ [\tau_{\param_1}] ++ [\tau_{\param_2}] = [\tau_\param]
		\\ Z; \Gamma, z :\ !\ [\tau_{\param_2}] \to \tau_\ret \vdash F : \tret
	}
	{Z; \Gamma, [y :\ !\tau_{\param_1}] \vdash \icode{let z := pap c [y]; F} : \tret}$
\end{mathpar}
\begin{mathpar}
	$\inferrule[Let-VarApp-Full]{\nz(Z, y) 
		\\ Z; \Gamma, z :\ !\tau_\ret \vdash F : \tret
	}
	{Z; \Gamma, x :\ !\ \tau_\param \to \tau_\ret, y :\ ! \tau_\param  \vdash \icode{let z := x y; F} : \tret}$
\end{mathpar}
\begin{mathpar}
	$\inferrule[Let-VarApp-Part]{\nz(Z, y) 
		\\ |[\tau_{\text{param'}}]| \geq 1
		\\ Z; \Gamma, z :\ !\ [\tau_{\text{param'}}] \to \tau_\ret \vdash F : \tret
	}
	{Z; \Gamma, x :\ !\ (\tau_\param :: [\tau_{\text{param'}}]) \to \tau_\ret, y :\ ! \tau_\param \vdash \icode{let z := x y; F} : \tret}$
\end{mathpar}
\begin{mathpar}
	$\inferrule[Let-Ctor]{[\nz(Z, y)]
		\\ A \in \dom(\gamma)
		\\ (\gamma(A), i, [\tau], [\tau_\arrg?]) \in \dom(\inferTypeArgs)
		\\ [\tau_\arrg'] = \inferTypeArgs(\gamma(A), i, [\tau], [\tau_\arrg?])
		\\ \gamma(A)\{A, [\tau_\arrg']\}_i = [\tau_\field] \to *x^\kappa_\adt
		\\ [\tau_\field] = [\tau]
		\\ Z; \Gamma, z : *\ A\ [\tau_\arrg'] \vdash F : \tret
	}
	{Z; \Gamma, [y : \tau] \vdash \icode{let x = (A [$\tau_\arrg$?]).ctorᵢ [y]; F} : \tret}$
\end{mathpar}
\begin{mathpar}
	$\inferrule[Let-Proj-*]{Z(y, i, j) = \bot
		\\ A \in \dom(\gamma)
		\\ Z[(y, i, j) \mapsto \top]; \Gamma, z : \gamma(A)\{A, [\tau_{\arrg}]\}_{ij} \vdash F : \tret
	}
	{Z; \Gamma, y : *\ A\ [\tau_{\arrg}] \vdash \icode{let z = projᵢⱼ y; F} : \tret}$
\end{mathpar}
\begin{mathpar}
	$\inferrule[Let-Proj-!]{Z(y, i, j) = \bot
		\\ A \in \dom(\gamma)
		\\ Z; \Gamma, z :\ !\gamma(A)\{A, [\tau_{\arrg}]\}_{ij} \vdash F : \tret
	}
	{Z; \Gamma, y :\ !\ A\ [\tau_{\arrg}] \vdash \icode{let z = projᵢⱼ y; F} : \tret}$
\end{mathpar}